\chapter*{Avant-propos de l'éditeur-traducteur}
\addcontentsline{toc}{chapter}{Avant-propos de l'éditeur-traducteur}

Ce livre est la deuxième partie d'un premier livre de propositions politiques pour réformer le droit d'auteur. Comme lui, il est un livre de remix. Comme dans le premier livre et pour les mêmes raisons, j'enjoins le lecteur à essayer de découvrir les différents chapitres sans préjuger immédiatement de leur auteur original et dans le désordre s'il le souhaite.

Les témoignages de ce livre se répondent et expliquent les mesures proposées dans le premier livre. Tous les personnages croient qu'Internet doit permettre un monde meilleur, mais chacun aborde ce monde à sa manière. 

En page \pageref{cycliste}, le cycliste s'intéresse avant tout aux usages concrets qu'il peut faire de son vélo et constate un monde totalitaire. L'amoureux des droits de l'homme dénonce ce monde. Le propriétaire considère en page \pageref{proprio} que la propriété fait partie des droits naturels et la défend obstinément. Pour lui, la propriété intellectuelle est l'instrument de contrôle totalitaire idéal. Le conservateur désespère de la lenteur du progrès ou plutôt de la rapidité du regrès. La culture y est aux mains des plus riches pour lui en page  \pageref{conserv}. Le développeur se réjouit des opportunités à venir. Le chercheur feuillette avec joie l'abondance de travaux de ses collègues à sa disposition. 

À l'inverse, dans le monde dont rêve le poète en page \pageref{poete}, que le développeur développe en page \pageref{devadmin} et que le chercheur pratique en page \pageref{cherch} en feuillettant avec joie l'abondance de travaux produits par ses collègues, l'État remplit sa tâche première. Il sert ses citoyens.

La vue d'ensemble qui se dégage de la lecture des différents argumentaires doit permettre de comprendre quels sont les problèmes essentiels que pose le droit d'auteur à l'ère d'Internet. Elle doit convaincre de la nécessité pour l'État d'adopter le type de mesures promues dans le premier livre.

Ce deuxième livre est une deuxième bouteille à la mer~: le lira qui voudra pour en tirer les enseignements qu'il voudra. Nous, internautes qui avons financé l'impression et l'envoi à nos députés du premier livre de mesures, nous enjoignons nos députés à le lire avec attention et à transformer nos propositions en lois. 
