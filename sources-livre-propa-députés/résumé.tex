% rubber: set program lualatex

\RequirePackage[l2tabu, orthodox]{nag} % Avertissements contre les mauvaises pratiques et les packages obsolètes
\documentclass[12pt,liststotoc,bibtotoc]{scrbook}
\usepackage[paperwidth=16cm, paperheight=24cm]{geometry}
\usepackage{polyglossia}
	\setdefaultlanguage{french}
	\newcommand{\ieme}{\textsuperscript{e}}
	
%\usepackage{luatextra} % Wrapper pour les utilitaires luatex
\usepackage{fontspec} % Permet l'utilisation des polices OpenType avec luatex
	\setmainfont[Ligatures=TeX]{Linux Libertine O}
	\setsansfont[Ligatures=TeX]{Linux Biolinum O}

% Deux-trois francisations supplémentaires
\addto\captionsfrench{\renewcommand{\contentsname}{Sommaire}}

% Typo plus propre et règles de typo françaises
\usepackage{xspace}
\usepackage{microtype}
\usepackage[hyphenation,parindent,lastparline]{impnattypo}
\usepackage[all]{nowidow}

%pour énumérer différement
\usepackage{enumitem}

% ajoute (entre autre) la bibliographie dans la table des matieres 
\usepackage[nottoc]{tocbibind}
\setcounter{secnumdepth}{1}
\setcounter{tocdepth}{2}
\addto\captionsfrench{\renewcommand{\bibname}{Lectures complémentaires en français}} 

% urls internes et externes en noir cliquables 
\usepackage[urlcolor=blue,linkcolor=blue,colorlinks,breaklinks=true]{hyperref}
\usepackage[anythingbreaks]{breakurl}

\newcommand{\livre}[1]{\emph{#1}}
\newcommand{\site}[1]{\emph{#1}}

\usepackage{framed}
\usepackage[hyperref,framed]{ntheorem}
\theoremstyle{break}
\newframedtheorem{mesure}{Mesure}

%titre
\title{Il faut réformer le droit d'auteur}
% \subtitle{}
\author{Édité par Lionel Maurel et Xavier Gillard}
\date{\today}
\publishers{Soutenu par SavoirsCom1 et…}
% \uppertitleback{}
\lowertitleback{\centering

Livre sous licence Creative Commons CC-BY-SA :

Le texte complet de la licence est disponible à
\url{http://creativecommons.org/licenses/by-sa/2.0/fr/}

N'hésitez pas à me contacter à xavier@sploing.fr}


\bibliographystyle{plain-fr}


\begin{document}
\renewcommand{\labelitemi}{$\bullet$}
%\renewcommand{\bibsection}{\chapter{\refname}}

\maketitle

\chapter*{Avant-propos de l'éditeur-traducteur}
\addcontentsline{toc}{chapter}{Avant-propos de l'éditeur-traducteur}

Ce livre est un livre de remix. Il compile des argumentaires et revendications portés par différentes associations et collectifs français. Il n'est pas nécessaire de le lire dans l'ordre. Les plus pressés iront directement à l'index des mesures proposées situé en page \pageref{index}. J'enjoins au lecteur d'essayer de découvrir les différents chapitres sans préjuger immédiatement de leur source originelle.

La croyance fondamentale qui guide les mesures proposées est que la libre circulation de l'information qu'Internet a facilitée devrait permettre un monde meilleur. C'est pourquoi le droit d'auteur dans sa forme actuelle est un frein au progrès. L'État doit abandonner une législation obsolète pour entrer avec fracas dans le nouveau millénaire. Ouverture et liberté doivent être les maîtres mots de la nouvelle société de l'information pour que nous en tirions le meilleur parti. 

Ce discours a déjà été répété à l'envi par de multiples personnes. Parmi les parlementaires et l'exécutif français comme européen, le nombre de gens qu'il convainc ne cesse de croître. En même temps, les industries de l'ancien monde continuent de lutter âprement pour retarder leur chute et éviter de changer de modèle économique. Elles sont soutenues par quelques politiques et industriels qui n'ont pas grandi dans un monde où l'information veut être libre et ont peur des nouvelles libertés qui s'offrent à nous. 

Pour ceux qui ne sont pas convaincus par les arguments développés dans ce livre, nous avons compilé en ligne un recueil\footnote{disponible en ligne à \url{http://fichiers.sploing.fr/contexte.pdf} au format PDF ou comme page web à \url{http://rda.sploing.fr/partie-2-les-temoignages}} de témoignages de personnages qui vise à mieux leur faire comprendre les enjeux de la nouvelle ère et à leur donner un avant-goût des bienfaits que ces libertés nous apporteront. Pour ceux qui doutent même de la pertinence de remettre en cause le droit d'auteur à l'heure actuelle, nous avons inclus une courte allégorie cycliste au tout début du livre.

Ces deux livres, le recueil de propositions et celui de témoignages, se veulent une bouteille à la mer~: les lira qui voudra pour en tirer les enseignements qu'il voudra. Nous, internautes qui avons financé l'impression et l'envoi à nos députés du premier livre de mesures, nous enjoignons nos députés à lire les deux livres avec attention et à transformer nos propositions en lois. 

\chapter*{Introduction}\label{premintro}
\addcontentsline{toc}{chapter}{Introduction}

Le système du droit d’auteur est aujourd’hui déphasé. Il criminalise une génération entière dans une tentative désespérée d’arrêter le progrès technologique. Pourtant le partage de fichiers continue à s’accroître et les remix continuent à fleurir. Ni la propagande ni les techniques d’intimidation ni le durcissement des lois n’ont pu arrêter cette évolution.

Il n’est plus possible de renforcer les mesures de protection du droit d'auteur sans violer des droits humains fondamentaux. Tant que les individus pourront communiquer en privé, ils s’en serviront pour partager des contenus soumis au droit d’auteur. Le seul moyen de limiter le partage de fichiers c’est de supprimer le droit à la communication privée.

Nous voulons une société où la culture prospère, où les artistes et les créateurs ont une chance de vivre de leur art. Heureusement, il n’y a aucune contradiction entre le partage et la culture. Une décennie de partage intensif nous l’a prouvé. L'Histoire nous l'a prouvé.

Quand les bibliothèques publiques sont apparues en Europe il y a 150 ans, les éditeurs y étaient extrêmement opposés. Leurs arguments étaient les mêmes que ceux dont on se sert aujourd’hui dans le débat sur le partage. Si le peuple pouvait accéder gratuitement aux livres, les auteurs ne pourraient plus vivre de leur art. C'était la mort du livre.

Nous savons à présent que les arguments contre les bibliothèques publiques étaient faux. Les livres prêtés par les bibliothèques n'ont jamais fait s'écrouler les ventes des éditeurs et toute la société bénéficie de l'accès libre à la culture. 

Internet est la plus merveilleuse bibliothèque publique jamais créée. Pour tous, y compris ceux aux moyens économiques limités, l’accès à toute la culture de l’humanité n’est plus qu’à un simple clic. Cette liberté de circulation de l'information est une révolution. L'État doit l'encourager en la protégeant et en participant lui-même à la production et à la diffusion de cette culture. C'est dans son intérêt comme dans celui de ses citoyens. 

L'État doit accepter le progrès pour la diffusion de la connaissance qu'a apporté Internet. Il doit abandonner l'idée qu'interdire des pratiques majoritaires comme le partage à but non lucratif est porteur d'avenir. Dans une démocratie les lois sont d'abord ce que les citoyens en font. Il est absurde qu'un État démocratique lutte contre la majorité de ses citoyens.

Notre État doit aussi accompagner et encourager pour le bien de tous le progrès. Une vraie réforme du droit d'auteur n'est pas qu'une adaptation marginale de quelques restrictions obsolètes et dangereuses dans les lois actuelles sur le droit d'auteur. Elle implique l'ajout de nouvelles contraintes pour les administrations et les artistes en faveur de la diffusion du maximum d'informations possible. 

Ces contraintes doivent lui permettre de co-créer avec ses citoyens les services dont ils ont besoin. Pour prendre les décisions les plus avisées et créer les œuvres les plus innovantes, ses citoyens ont besoin de données libres. Les données sont la matière première de l'économie de la connaissance. C'est pourquoi l'État doit fournir à ses citoyens autant d'informations et d'œuvres culturelles libres qu'il peut. Des mécanismes de gestion collective assainis peuvent permettre de favoriser ces nouvelles pratiques.

Les propositions que nous faisons sont résumées par les lignes directrices suivantes~:

\begin{enumerate}
\item Conserver l'essentiel des droits moraux et garder l'exclusivité commerciale pour permettre aux modèles
économiques actuellement viables de le rester.
\item Laisser s'épanouir la culture du remix et de l'échange à but non-lucratif.
\item Diminuer la durée de protection.
\item Enregistrer régulièrement les œuvres pour constituer un catalogue des métadonnées. 
\item Protéger et promouvoir le domaine public et l'accès libre à des œuvres et données libres.
\end{enumerate}

Il reste aux politiques à s'emparer de ces idées pour les transformer en lois. Les plus pressés trouveront un index des réformes proposées en page \pageref{index}.


\tableofcontents
\chapter{L'allégorie du cycliste}\label{cycliste}

Je me suis acheté un vélo. Un beau modèle. Je l’ai attendu longtemps.

Aux États-Unis, ça fait déjà longtemps qu’il est sur le marché. Pas en Allemagne. Et l’importer aurait été illégal. Le mois dernier, on pouvait le louer auprès d’une grande chaîne de télé pendant une
semaine et faire une virée. Ça m’a plu.

Mais à la fin de la journée, il était de nouveau verrouillé. Je devais attendre.

Par la suite le vélo est devenu disponible dans les arrières-cours de mon quartier pour pas un rond. Ça m’a paru un peu louche.

Mais je m’en fiche. Maintenant, j’ai mon vélo. Il est beau.

Il est marqué jusque sur les autocollants du cadre que je ne dois pas voler ou reconstituer de vélo. Logique. Pourquoi d’ailleurs~? Je l’ai bel et bien acheté.

Avant le premier démarrage j’ai dû appeler le fabricant et lui expliquer quels étaient les trois quartiers de la ville dans lesquels je voulais utiliser mon vélo. Lorsque je circule dans un quartier
non autorisé les freins s’enclenchent tout seul. Je n’ai rien à faire. Ça fait partie du service. Je peux alors appeler le fabricant et reconfigurer le vélo. De la sorte je circule dans toute la
ville.

Si je voulais louer mon vélo ça ne me serait pas permis. La selle envoie des petites décharges dans le corps et signifie son désaccord. C’est à la répartition des masses à l’arrière qu’elle reconnaît
qui s’assoit sur le vélo. Si ce n’est pas moi la sonnette carillonne. Du coup je fais attention à mon régime. Sinon mon vélo ne me reconnaît plus.

Il y a peu j’ai voulu le repeindre. Je trouvais que le kaki faisait vieux jeu. En grande surface on m’a ri au nez. Ça serait tout à fait illégal. Est-ce que j’avais demandé au fabricant~? Il aurait
sûrement dû prévoir quelque chose pour la couleur.

La ville vient de construire de nouvelles pistes cyclables et je trouvais qu’elle avait raison. Mais j’ai entendu une rumeur~: Mon vélo ne peut plus rouler dessus. Les pneus sont trop minces. Ils ne
passent plus sur le nouveau revêtement.

Mais une nouvelle génération arrive. Avec des chenilles. Ils seront beaucoup plus sûrs.

Maintenant il y a des postes de police sur les pistes. Pour contrôler qui est sur quel vélo. Quand on perd le contact visuel avec tous les postes le vélo éjecte son passager. Par temps de
brouillard on voit souvent des hommes joncher la route comme des fruits trop mûrs.

Si on me vole mon vélo, ça peut devenir encore plus cher. Parce que je l’ai diffusé. Le constructeur ne peut plus en vendre un directement au voleur. J’en suis responsable.

Tout ça m’est devenu trop périlleux. Maintenant je veux donner mon vélo à d’autres. Mais on chuchote que ce ne serait pas permis. Mon vélo ne serait qu’à moi. Je l’ai donc simplement
supprimé.\footnote{\url{http://www.137b.org/?p=2445} pour la version allemande sous CC-BY. Traduction sous CC-BY-SA}

\chapter{Conserver ce qui est utile du droit moral}\label{pater}

\begin{mesure}[Préservation du droit à la paternité]
 «~Rendre à César ce qui est à César~» est une maxime qui met tout le monde d’accord.
\end{mesure}

Dans les faits, les convenances sont souvent
plus strictes sur le sujet que n’importe quelle législation relative au droit d’auteur.

Les scientifiques ou les blogueurs ont tendance à citer leurs sources d’une façon qui fait bien plus que respecter le
minimum légal. Il y a plusieurs raisons à cela. Cela rend votre article plus crédible si vous donnez
les liens vers vos sources afin que vos lecteurs puissent en vérifier l’origine s'ils le
souhaitent. Les personnes que vous citez sont contentes, elles seront donc plus enclines à citer
vos propres articles si l’occasion se présente et votre influence augmentera. 

Le droit d’être reconnu en tant qu’auteur sur Internet n’est pas menacé. Nous proposons donc de
laisser inchangé ce point de la législation du droit d’auteur. Nous proposons également de laisser inchangé le point suivant~:

\begin{mesure}[Préservation du droit à la divulgation]
Le mode et le moment de la première diffusion ne peuvent être décidés que par les auteurs des œuvres. Une personne qui ne souhaite pas que sa création soit portée à la connaissance du public ne doit pas être contrainte à le faire.  
\end{mesure}

Il est également important pour des raisons commerciales que l’auteur et les intermédiaires auxquels il peut éventuellement faire appel soient maîtres du calendrier de diffusion de son œuvre. C’est ce que vise à garantir le droit à la divulgation.

En revanche, pour permettre l'émergence d'un droit au remix à but non lucratif développé dans le chapitre \ref{remix}, il faut permettre la modification et l'adaptation des oeuvres  pour ne permettre à l'auteur de ne porter plainte que lorsque les transformations non-commerciales de ses œuvres nuisent à sa réputation. C'est ce que propose déjà la Convention de Berne~:

\begin{quotation}
Indépendamment des droits patrimoniaux d’auteur, et même après la cession des dits droits, l’auteur conserve le droit de revendiquer la paternité de l’œuvre et de s’opposer à toute déformation, mutilation ou autre modification de cette œuvre ou à toute autre atteinte à la même œuvre, \textbf{préjudiciables à son honneur ou à sa réputation}.
\end{quotation}

C’est à travers la jurisprudence que le droit moral est devenu en France aussi étendu qu’aujourd’hui,en consacrant un droit quasi-absolu au respect de l'intégrité des oeuvres au bénéfice de l'auteur, mais il n'en est pas ainsi dans les conventions internationales.

La directive européenne 2001/29 sur le droit d'auteur dans la solution pourrait être modifiée dans le sens de la Convention de Berne et le législateur français pourrait également intervenir en ajoutant un article L-210 au Code de la propriété intellectuelle :

\begin{mesure}[Restriction du droit à l'intégrité de l'œuvre]
L’auteur jouit du droit au respect de l’intégrité de son oeuvre. Il peut s’opposer à toute déformation, mutilation ou autre modification de cette œuvre, dans la mesure où elles sont préjudiciables à son honneur ou à sa réputation.
\end{mesure}


\chapter{Bannir les verrous numériques}\label{verrous}

Les MTP ou Mesures Techniques de
Protection, plus
connues sous le sigle anglais DRM pour Digital
Rights Management, visent à restreindre les usages possibles des consommateurs d’œuvres «achetées»
légalement et sur lesquelles ils devraient donc pouvoir exercer tous leurs droits. 

Nous proposons deux mesures les concernant. La première n'est réalisable qu'à long terme et la deuxième peut être immédiatement débattue et votée par le Parlement.

\begin{mesure}[Interdire les mesures de protection technique]
 Il devrait être systématiquement légal de passer outre les MTP et nous devrions bannir les MTP qui
empêchent des usages légaux. Les grandes multinationales ne devraient pas avoir le droit d’écrire
leurs propres lois d’utilisation des fichiers.
\end{mesure}

La législation doit bénéficier à la société toute entière, y compris les consommateurs. En même temps avoir le droit
de faire quelque chose selon la loi n’a que peu de valeur en soi si vous n’avez pas les moyens
pratiques de le faire.

Il n’y a aucun intérêt à ce que nos parlements introduisent une législation équilibrée et
raisonnable sur le droit d’auteur si en même temps nous permettons aux multinationales d’écrire
leurs propres lois et d’obliger à leur respect par des moyens techniques.

Dans son livre \livre{Culture Libre}, le professeur de droit Lawrence Lessig donne l’exemple d’un livre
numérique publié par Adobe. Le livre était \livre{Alice au Pays des Merveilles}. Il a été publié la première
fois en 1865 et son copyright a expiré depuis longtemps. Puisqu’il n’est plus sous copyright,
chacun devrait pouvoir faire ce qu’il veut du texte de Lewis Carroll.

Mais un éditeur auquel il a acheté le texte a décidé de régler les verrous DRM de telle sorte qu'il ne pouvait pas en extraire une copie ni l’imprimer ni le louer et encore moins le donner à un ami.

Les aveugles et malvoyants qui ont besoin de convertir les livres numériques dans des formats audio
qui leurs soient accessibles sont souvent entravés par les verrous DRM. Même s’ils ont légalement le
droit de changer de format, les verrous les en empêchent en pratique.

Un autre exemple est le zonage régional sur les DVDs qui empêche de regarder des films légalement achetés s'ils sont achetés dans une autre zone que celle où a été acheté le lecteur.

\begin{mesure}[Autorisation de contournement des verrous]
Si des verrous existent, ils doivent pouvoir être légalement cassés s'ils empêchent la jouissance pleine et entière de l'œuvre et de ses usages légaux. 
\end{mesure}

Les verrous DRM ont été juridiquement sanctuarisés par les traités de l'OMPI de 1996 et la directive européenne 2001/29. La France ne peut donc espérer entraîner un changement rapide de la législation européenne et des traités qu'elle a signés. De plus il est possible qu'à l'issue des négociations la question reste ouverte et que certains pays décident de soutenir les verrous au contraire de la France. Il faut donc en attendant un consensus international autoriser le contournement des verrous au lieu de simplement les interdire. 

Il existe déjà un régime d'exceptions aux mesures techniques de protection dans la loi française même s'il est peu et difficilement appliqué pour des questions de lourdeur administrative. Une telle exception existe aussi aux États-Unis via le Digital Millenium Copyright Act qui permet de contourner les verrous pour remixer les morceaux. 


\chapter{Préserver et promouvoir le domaine public}\label{dompub}

\section{Le domaine public est un bien commun qui a besoin d'être défendu}
Depuis des siècles, les bibliothèques, les archives et les musées ont été, partout en Europe, les
gardiens de notre riche et diversifié patrimoine culturel. Qu’il s’agisse de sculpture, de peinture,
de musique ou de littérature, ces institutions ont su préserver nos trésors de connaissance, de
beauté et d’imagination et en donner l’accès au plus grand nombre.

La numérisation apporte un souffle nouveau aux œuvres du passé et les transforme non
seulement en une source d’intérêt pour les utilisateurs individuels mais aussi en matériau précieux
pour construire l’économie numérique de demain.

Nous sommes convaincus que la mission de rendre accessible en ligne notre patrimoine culturel
et de le préserver pour les générations futures est avant tout du ressort des
institutions publiques. Il est inconcevable d’abandonner cette responsabilité à un ou plusieurs
acteurs privés, au risque de le(s) voir imposer une forme de contrôle. Mais cela ne signifie pas que
les entreprises privés ne doivent pas s’impliquer en matière de numérisation : bien au contraire,
nous considérons qu’elles ont un rôle et souhaitons qu’elles renforcent leurs investissements dans
le cadre de partenariats équilibrés et profitables.

Or ces partenariats sont rarement équilibrés faute de politique nationale forte. Il est donc essentiel de le consacrer et de le protéger par la loi. On ne peut plus laisser une question aussi essentielle relever du ressort des seuls établissements culturels et des collectivités dont ils dépendent, qui sont souvent mal armés pour aborder la question et engagés dans des logiques de dégagement de ressources propres qui peuvent les pousser à marchandiser le domaine public. Le domaine public doit être le même pour tous les citoyens en France, car derrière cette notion, c’est la liberté fondamentale d’accès à la Culture et le droit de créer à partir des œuvres du passé qui sont en jeu. Le défendre impliquera donc la création de nouvelles peines pour infraction aux règles du domaine public.

\section{Quatre mesures pour que rien ne puisse être soustrait au domaine public}
\begin{mesure}[Inscription du domaine public dans la loi]
Inscrire le domaine public dans la loi. Au terme du délai de validité du droit d'auteur, l'œuvre est réputée appartenir au domaine public.
\end{mesure}

Cette précision permet de faire référence par la suite à des règles particulières attachées aux œuvres dans le domaine public pour le protéger.

\begin{mesure}[Protection du domaine public numérisé]
Les reproductions fidèles d’œuvres appartenant au domaine public doivent aussi être dans le domaine public.
\end{mesure}

Ce point permet d'interdire les pratiques des très nombreux musées, bibliothèques et services d’archives en France qui estiment qu’ils bénéficient d’un droit d’auteur sur les reproductions numériques d’œuvres élevées dans le domaine public. Ce principe a déjà été consacré dans la jurisprudence aux États-Unis à l’occasion de la décision Bridgeman Art library v. Corel Corp. La mesure s'appliquera principalement aux œuvres numérisées ou intégrées à des bases de données.

\begin{mesure}[Le domaine public n'appartient pas à l'État]
Exclure que les œuvres du domaine public puissent être considérées comme des informations appartenant à l'État.
\end{mesure}

Plusieurs institutions culturelles considèrent qu’en numérisant des œuvres du domaine public, elles produisent des données (des suites de 0 et de 1) relevant du champ d’application de la loi du 17 juillet 1978 sur les informations publiques.

Cette interprétation a un effet redoutable, car cette loi de 1978 , si elle n’autorise pas en principe les administrations à s’opposer à la réutilisation des informations, leur permet de la soumettre au paiement d’une redevance, notamment pour les usages commerciaux (exemple). La loi de 1978 permet d'installer un système de domaine public payant. De plus, les institutions culturelles bénéficient d’un régime dérogatoire complexe, dit exception culturelle, qui leur donne une plus grande marge de manœuvre pour poser des restrictions à la réutilisation.

\begin{mesure}[Élévation volontaire dans le domaine public]
Permettre aux auteurs de renoncer à tous leurs droits sur leurs œuvres pour enrichir le domaine public.
\end{mesure}

On devrait permettre aux auteurs qui le souhaitent de verser par anticipation leurs œuvres dans le domaine public. La renonciation inclurait les droits patrimoniaux comme moraux. 

Enfin les œuvres créées par les agents publics dans le cadre de leurs missions de service pourraient être versées automatiquement dans le domaine public, comme c’est le cas actuellement aux États-Unis pour les œuvres produites par les agents fédéraux. 








\chapter{Reconnaître et promouvoir les licences libres}\label{licencelibre}

\section{La possibilité d'inscrire la définition d'une licence libre dans la loi}
La définition des licences libres relèvent actuellement purement du droit contractuel puisque la loi ne définit pas ce qu'est une licence libre. Il n'est pas nécessaire pour l'État de définir lui-même ce que des associations comme l'Open Knowledge Foundation ont déjà défini collaborativement. Il est en revanche nécessaire que les services de l'État se réfèrent à une définition précise des licences libres, qui est la suivante~:

\begin{quotation}
Une œuvre est réputée libre lorsque sa licence confère à toute personne morale ou physique, en tout temps et en tout lieu, les quatre possibilités suivantes~:
\begin{itemize}
\item La possibilité d'utiliser l'œuvre, pour tous les usages ;
\item La possibilité d'étudier l'œuvre ;
\item La possibilité de redistribuer des copies de l'œuvre ;
\item La possibilité de modifier l'œuvre et de publier ses modifications.
\end{itemize}
\end{quotation}

Des exemples de licences libres qui peuvent être appliquées à des œuvres ou données sont les licences Creative Commons CC-BY ou CC-BY-SA, la licence Art Libre, les licences Cecill, la General Public Licence, la licence BSD ou encore les licences gouvernementales Open Data (Licence Ouverte en France, Open Government Licence au Royaume-Uni). 

Pour les oeuvres et données produites par les administrations publiques, une préférence devrait être accordée aux licences comportant une clause de partage à l'identique (Share Alike) si ces administrations désirent limiter les risques de réappropriation exclusive. 

\section{Libérer toutes les données et œuvres qui peuvent l'être}
Nous proposons de rendre obligatoire la libération des données et des œuvres produites ou subventionnées par les services de l’État ou des collectivités locales. Par exemple, un logiciel développé sur commande de l’administration devrait être libre, tout comme des cartes, des travaux de recherche ou un catalogue des métadonnées des œuvres enregistrées. Le rôle des personnes publiques est de favoriser la diffusion de la connaissance et l’initiative individuelle, non de marchander cette connaissance déjà financée par les impôts des citoyens ou d’accorder des monopoles à des entreprises. C'est particulièrement le cas dans le domaine de la recherche scientifique et les travaux des chercheurs produits sur des fonds publics devraient faire l'objet d'une publication en libre accès sous licence libre.

\begin{mesure}[Libération des œuvres subventionnées]
Toutes les œuvres ou données immatérielles produites sur commande des personnes morales de droit public ou co-financées par celles-ci doivent être publiées sous licence libre gratuitement ou pour un coût d'accès marginal. Ce passage sous licence libre devra aussi s'appliquer pour toutes les oeuvres divulguées par les administrations. Les données confidentielles ou critiques pour la sécurité publique sont les seules à ne pas devoir être publiées.
\end{mesure}

L'État a tout intérêt à travailler en co-création avec ses citoyens plutôt que contre eux. Le service de l'État Étalab a déjà publié une licence libre aux termes définis ci-dessus, appelée simplement \textit{Licence ouverte}. Cette licence pourrait être la licence standard des publications des bases de données crées par l'administration, sauf précisions spécifiques. 

Les œuvres qui relèvent du code de la propriété intellectuelle pourront être placées sous des licences libres comme la \textit{Creative Commons Paternité} ou \textit{Art libre} par exemple. La \textit{Creative Commons Paternité}, plus communément désignée CC-BY, n'impose comme condition à la réutilisation que la reconnaissance de la paternité, tandis que la licence \textit{Art libre} exige en plus que les œuvres dérivées soient publiées sous la même licence.

\subsection{La sécurité des citoyens comme limite}
Si nous insistons sur la nécessité de publier le plus grand nombre de données possibles, c'est parce qu'il n'est pas possible à l'État de prévoir quelles données seront utiles ou non. Bien souvent, c'est justement en reliant plusieurs bases de données individuellement anodines que l'on extrait des informations utiles. L'État ne peut pragmatiquement pas imaginer tous les liens qui peuvent être faits avec les données publiées. De plus, imposer que \emph{toutes} les données doivent être publiées interdit de publier des bases de données volontairement incomplètes et oblige à publier les données sources. 

La révision de la directive européenne sur les informations publiques a étendu le droit à la réutilisation dont bénéficie les citoyens européens, mais elle ne va toujours pas jusqu'à imposer aux administrations une obligation de publier en ligne les données qu'elles produisent, ce qui limite fortement la dynamique de l'ouverture des données.


La seule limite à cette logique d'ouverture de l'État doit être la protection de la vie privée ou de la sécurité des citoyens. Dans le cas de base de données brutes il faut prendre en compte la possibilité d'identifier les citoyens même si les données sont anonymisées. Par exemple, si la publication de données médicales comporte toujours le code postal et la date de naissance exacte des patients et permet à des assureurs de retrouver facilement quels clients sont affectés par des maladies graves, alors cette publication doit être être interdite en l'état. Le niveau de granularité de la base de données doit être augmenté de sorte que cibler les individus ne soit plus possible.

\subsection{Libérer les travaux de recherche}

La libération des œuvres et données produites par les services de l’État concernerait en particulier les chercheurs et instaurerait une obligation nationale de publication en accès libre des travaux scientifiques financés par le biais de l'argent public. Une telle obligation existe déjà au niveau européen dans le cadre du programme OpenAIRE par exemple et plusieurs Etats européens étudient en ce moment la possibilité de mettre en place des politiques de libre accès au niveau national.

A minima, l'obligation faite aux chercheurs produisant des travaux sur financement public devrait porter sur le versement de leurs articles dans des dépôts librement accessibles en ligne. Les chercheurs pourraient continuer à publier dans des revues commerciales, à la condition que les cessions de droits consenties aux éditeurs ne les empêchent pas de déposer par ailleurs leurs travaux en libre accès. Au-delà, il importe que les États soutiennent le développement de revues en accès libre, en leur assurant des moyens suffisants pour garantir leur pérennité sans avoir à reporter l'intégralité de leurs coûts sur les auteurs d'articles. 

Des sites de dépôts en accès libre gérés par des universités ou des centres de recherche existent déjà en France, ainsi que des initiatives de revues en accès libre. Il est donc possible de s’appuyer sur l’existant pour poursuivre le mouvement.


 \section{Rendre accessibles les données libérées}
Libérer les données n'est pas suffisant si celles-ci ne sont pas facilement accessibles parce que leur accès est onéreux, parce que le format de diffusion n'est pas utilisable informatiquement ou est non-standard et nécessite de payer des licences logicielles élevées pour être lu.

\begin{mesure}[Conditions d'accessibilité des données libérées]
Les données produites et publiées par l'administration sous licence libre doivent respecter les contraintes suivantes~:

\begin{itemize}
\item Entières~: Les bases de données sont intégralement publiées.
\item Brutes~: Leur format est directement utilisable par un ordinateur.
\item Documentées~: Elles sont accompagnées de leurs métadonnées dans un format documenté.
\item Interopérables~: La documentation du format de fichier est aisément accessible et complète. 
\item Actuelles~: Elles sont les plus récentes possibles.
\item Permanentes~: Leurs adresses d'accès sont durables.
\item Gratuites ou peu coûteuses~: Le coût d'accès est nul ou marginal.
\end{itemize}
\end{mesure}

Le moyen technique le plus simple de remplir ces contraintes est de mettre en place via Internet des portails de dépôt qui les recensent et les mettent à disposition via des protocoles standards comme ceux utilisés pour afficher les pages web. Plusieurs gouvernements en Europe ont lancé de tels portails de diffusion des informations, comme le site data.gouv.fr en France. Mais les ressources humaines, juridiques ou informatiques dont disposent ces portails doivent être renforcées pour améliorer la qualité de service.


\chapter{Créer un droit à la transformation des œuvres}\label{remix}

La législation très restrictive d’aujourd’hui est un obstacle majeur pour les créateurs qui veulent embrasser toutes les possibilités offertes par le numérique.

\begin{mesure}[Valoriser les œuvres transformatives]
Nous voulons que soient reconnus trois droits à la création:

\begin{itemize}
\item Le droit de réutiliser toute œuvre dans une autre et de publier le résultat. 
\item Le droit d’utiliser toute œuvre existante pour la modifier et de publier ces modifications. 
\item Le droit d’utiliser lucrativement ces œuvres dérivées en échange d’un paiement équilibré aux auteurs originaux. 
\end{itemize}
\end{mesure}

Le premier droit concerne par exemple la réutilisation d'une musique dans une vidéo amateur, le deuxième l'utilisation des images d'un film pour en faire une fausse bande-annonce et le troisième la vente de mashups sur iTunes.

Dans la création originelle classique, l’ancien disparaissait dans le nouveau. La culture du remix contemporaine se différencie en ce que l’ancien reste visible derrière le nouveau. Le remix est une nouvelle copie créative que l’on reconnaît comme telle. Dans cette mesure, puisque la copie créative fait partie du quotidien de nos interactions à travers toutes les couches sociales, la reconnaissance d’un droit au remix est une condition essentielle pour la liberté de création et d'expression de notre société.

Il faut protéger la culture du remix car elle nous fait passer d’une culture de la consommation passive à une culture de la création massive.

À l'heure actuelle les majors revendiquent
la propriété sur des sons individuels et de très courts extraits. Si vous êtes un musicien hip-hop,
attendez vous à payer des centaines de milliers d’euros par avance pour avoir le droit d’utiliser
des samples si vous souhaitez toujours rendre votre musique accessible au public.

C’est clairement une restriction du droit de créer de nouvelles cultures. C'est aussi un frein à une juste rémunération des auteurs. Si un droit au remix était reconnu, les auteurs des œuvres remixées pourraient être rémunérés en touchant une partie des bénéfices que génèrent les remix en échange de leur renonciation à leur droit moral à l'intégrité de leurs œuvres. En revanche, il serait totalement illégitime de rémunérer l'usage transformatif des oeuvres dans un cadre non-commercial, car celui-ci relève fondamentalement de la liberté d'expression. 

Une telle réforme peut être conduite de plusieurs manières et notamment par le biais d'un élargissement de l'exception de citation. C’est ce que recommande le rapport Lescure en France. La Commission européenne avait auparavant proposé d'introduire une nouvelle exception dans son livre vert sur le droit d'auteur dans l'économie de la connaissance publié en 2008. Le Canada a de son côté introduit en 2012 une exception en faveur du remix.  

La directive européenne de 2001 sur le droit d’auteur est suffisamment ouverte, s'agissant du droit de citation, pour s'appliquer aux usages transformatifs. En France les contours de l'exception sont encore beaucoup trop étroits pour s'appliquer aux mashups et au remix. C'est pourquoi l’article 122-5 du Code de la propriété intellectuelle pourrait être modifié de la manière suivante :

\begin{mesure}[Étendre le droit de citation]
Les analyses et citations concernant une oeuvre protégée au sens des articles L. 112-1 et L. 112-2 du présent Code, justifiées par le caractère critique, polémique, pédagogique, scientifique, d’information, \textbf{ou de création sans but lucratif} de l’oeuvre à laquelle elles sont incorporées et effectuées dans la mesure justifiée par le but poursuivi.
\end{mesure}

Nous précisons dans cette mesure qu'elle s'applique à toutes les sortes d'œuvres pour empêcher que la jurisprudence ne la restreigne aux textes comme c'est actuellement le cas en France.



\chapter{Enregistrer les œuvres tous les cinq ans}\label{registre}

Les œuvres orphelines sont un vrai problème. Bien souvent il est difficile de localiser le
propriétaire d’une œuvre mais celui-ci se manifeste quand l’œuvre dérivée est publiée. De plus la
majorité des œuvres orphelines ont peu ou aucune valeur commerciale mais il est quand même
impossible de les diffuser sans risquer des poursuites.

\begin{mesure}[Création d'un registre des œuvres protégées]
 La protection du droit d’auteur doit être accordée automatiquement dès la publication comme
aujourd’hui, mais si les propriétaires veulent continuer à jouir de l'entièreté de leurs droits après les cinq
premières années de publication, ils doivent se manifester régulièrement de sorte qu’ils soient facilement
trouvables. 

Dans le cas où l'ayant-droit ne se manifeste pas et dans l'attente de son réenregistrement, ses travaux sont soit placés sous une licence libre soit versés dans le domaine public. Il récupére ses droits à partir de la date de réenregistrement.

Pour simplifier sa mise en œuvre, cette mesure peut ne s'appliquer qu'aux œuvres à venir et non aux œuvres déjà publiées.
\end{mesure}

\section{Qu'est-ce qu'une œuvre orpheline~?}

Une œuvre orpheline est une œuvre encore protégée par le droit d’auteur mais pour laquelle le
détenteur des droits n’est pas connu ou ne peut être retrouvé. Cela peut être un livre, une chanson,
un film, une photo ou tout autre création qui tombe sous le coup de la législation relative au droit
d’auteur.

Les œuvres orphelines représentent un important problème pour quiconque souhaite les utiliser. Si
vous le faites sans en avoir obtenu la permission, vous courez le risque que le détenteur des droits
s’en souvienne soudainement, vous intente un procès et vous réclame beaucoup d’argent. Comme nous le
savons tous, les tribunaux peuvent être assez enclins à attribuer des réparations même pour des
violations mineures de droits d’auteur et à condamner à des sommes astronomiques. Dans la plupart
des cas, le risque n’est tout simplement pas acceptable.

Puisqu’il n’y a pas de détenteur de droits à qui s’adresser pour demander une licence, vous ne
pouvez rien faire. Peu importe combien vous trouvez important de partager cette œuvre avec le reste
du monde, il n’y a aucun moyen de le faire sans enfreindre la loi et sans vous exposer à un grand
risque financier. Les œuvres orphelines sont de fait bloquées par le droit d’auteur.

\section{Le trou noir du vingtième siècle}

Ce n’est pas un problème marginal. Une grande partie de notre héritage culturel commun du 20\ieme
siècle tombe dans cette catégorie. Environ 75\% des livres que Google souhaite numériser dans le
cadre de leur «~Google Books initiative~» sont épuisés mais toujours sous droit d’auteur.

Même s’il est théoriquement possible de retrouver le détenteur des droits pour beaucoup de ces
livres en entreprenant une investigation pour chaque cas individuel, cela devient en pratique
infaisable lorsque vous voulez numériser en masse.

Google Books n’est pas le seul projet qui numérise des œuvres et les rend disponibles même si
c’est celui qui a attiré le plus d’attention dernièrement. Il y a un projet européen appelé
Europeana avec un objectif similaire, ainsi que
l’initiative ouverte du Projet Gutenberg.
Tous ces
projets sont freinés par le problème des œuvres orphelines.

Si nous n’agissons pas, une grande part de notre héritage culturel commun du 20\ieme siècle risque de
se retrouver perdue avant qu’il ne soit légal de la sauver pour la postérité.

\section{Un registre bénéficierait à tous}

Dans le même temps, l’existence de bases de données publiques, dans lesquelles les détenteurs des
droits en questions peuvent facilement être retrouvés par tous ceux intéressés par l’obtention d’une
licence commerciale d’une œuvre, sera bien évidemment bénéfique aux détenteurs de droits. Si vous
voulez vendre quelque chose, rendre votre identité connue de potentiels acheteurs est bien sûr dans
votre propre intérêt.

Inversement, cette base de donnée permettra de savoir quelles œuvres appartiennent au domaine public ou de connaître les licences sous lesquelles les auteurs veulent placer leurs œuvres. Elle permettra donc aux auteurs de savoir rapidement quelles sont les œuvres qu'ils peuvent réutiliser facilement. Cela facilitera les processus de création.

Dans la dynamique de libération des données recommandée plus bas, cette base de donnée devrait être sous une licence libre et facilement cherchable via un portail sur Internet ou indexable par des moteurs de recherche externes. 

De telles bases de données existent déjà partiellement pour les œuvres du domaine public. Europeana par exemple se veut être un portail d'accès libre au domaine public européen.

\section{Une réponse conforme à la Convention de Berne}
La Convention de Berne interdit de poser des conditions préalables à la jouissance du droit d'auteur mais elle n'interdit pas de mettre sous tutelle la gestion des œuvres au cas où les auteurs sont introuvables en attendant leur réapparition. C'est pourquoi la présente proposition préserve les droits à rémunération des titulaires introuvables en cas de réapparition.

\section{Des accords actuellement défavorables aux auteurs}

Certains accords comme ReLIRE actuellement en vigueur ou en négociation sont défavorables aux auteurs, car les éditeurs souhaitent que l'exploitation des œuvres orphelines soit nécessairement gérée commercialement et ils sont chargés de contacter tous les auteurs pour les avertir de la prochaine perte de leurs droits s'ils ne se signalent pas. 

Or d'une part, si ce sont les éditeurs qui sont chargés de contacter les auteurs, ceux-ci pourraient avoir intérêt à ne pas le faire ou à ralentir le processus pour pouvoir rééditer leurs catalogues sans verser de droits aux auteurs. D'autre part imposer systématiquement des conditions commerciales réduit l'usage de l'œuvre et le bénéfice en notoriété de l'auteur. On ne saurait autrement préjuger de la volonté des auteurs surtout en imposant des modalités de paiement a priori qui limitent les usages des œuvres orphelines alors que l'objectif même de la législation est précisément inverse.

C'est pourquoi permettre aux auteurs de récupérer leurs droits dès qu'ils se signalent voire de poursuivre les éditeurs qui ne les ont pas contacté dans les temps rééquilibre la loi en faveur des auteurs tout en favorisant la réappropriation de leurs œuvres par le public. 






\chapter{Légalisation du partage non-marchand}\label{depen}

Jusqu’il y a 20 ans, le droit d’auteur concernait à peine le commun des mortels. Les réglementations visaient les acteurs commerciaux, comme les labels, les chaînes de télévision ou les maisons d’édition.

Les citoyens qui voulaient copier un poème et l’envoyer à un de leur proche ou enregistrer une chanson
sur une cassette et la donner à un ami n’avaient pas à s’inquiéter des poursuites judiciaires.

Mais le droit d'auteur n'a pas évolué depuis et impose de graves restrictions dans
la vie quotidienne des individus. Alors que la technologie a rendu le partage de plus en plus
simple, la législation protégeant le droit d'auteur a évolué dans le sens inverse, vers une criminalisation croissante de ce
partage.

\begin{mesure}[Dépénalisation du partage sans but lucratif]
 Nous voulons que le droit d’auteur redevienne ce pourquoi il a été conçu et rendre clair qu’il ne
doit réguler que les échanges commerciaux. Publier un travail protégé sans but lucratif
ne devrait jamais être interdit. La persistance du piratage d'œuvres protégées sans but lucratif est une bonne raison pour cette
légalisation.
\end{mesure}

Contrairement à ce qu'affirment les représentants des industries culturelles, une telle réforme est possible, dans le cadre des traités internationaux tels que la Convention de Berne ou ceux de l’Organisation Mondiale de la Propriété Intellectuelle.

Le partage des oeuvres ne constituent pas un préjudice qui devrait faire l'objet d'une compensation au profit des titulaires de droits. L'usage d'une oeuvre ne la déprécie pas, mais au contraire augmente toujours sa valeur. De nouvelles pistes pour le financement de la création peuvent et doivent être explorées, notamment parce que le nombre des créateurs augmentent considérablement à mesure que le numérique met en capacité de créer un plus grand nombre de citoyens. La légalisation du partage ne doit pas s'accompagner de systèmes de compensation à l'image de la copie privée. 

Pour éviter tous risques de dérives, il convient au maximum d'éviter d'importer des notions liées au droit d'auteur dans la consécration des échanges non-marchands. Notamment il ne paraît pas opportun de recourir à de nouvelles exceptions ou à des systèmes de gestion collective obligatoire, qui impliqueront ensuite nécessairement une logique compensatoire de rémunération. 

La solution la plus efficace pour légaliser les échanges non-marchands consiste sans doute à étendre la notion d'épuisement du droit d'auteur, déjà consacrée au niveau européen, en la rendant applicable dans l'environnement numérique, mais seulement aux échanges effectués sans but lucratif. Une telle réforme peut être opérée en révisant la directive 2001/29 sur le droit d'auteur dans la société de l'information.  

\section{Il faut s'adapter au sens de l'histoire}

Une telle réforme est~:

\begin{itemize}
 \item \emph{inéluctable}
 
On peut penser que ce serait une bonne chose si tous les échanges illégaux de fichiers
disparaissaient. Mais ça
ne change rien à la réalité. 

La limitation du partage de fichiers par les lois et la répression ne
fonctionnent pas. Le partage de fichier est là pour durer, avec notre accord ou sans.
\item \emph{indispensable}

Les tentatives d’imposer
l’interdiction du partage de fichiers mettent en danger les droits fondamentaux.

Cela serait une solution inacceptable même si la répression fonctionnait ou si le secteur de la culture était réellement en train de mourir. Ni l'un ni l'autre ne sont vrais.
\item \emph{inoffensive}

Les artistes et le secteur de la
culture se portent bien malgré le partage de fichiers (ou peut être grâce à
lui), il n’y a donc pas de réel problème à résoudre.

\item \emph{facile à mettre en place}

La raison est très simple. «~Suivre l’argent~» suffit aux autorités pour leur permettre de garder une trace des activités
commerciales.

Si un entrepreneur souhaite gagner de l’argent, la première des choses qu’il doit faire, c’est faire connaître au plus grand nombre possible ce qu’il a à proposer. Mais s'il propose quelque
chose d’illégal, cela arrivera aux oreilles de la police avant qu’il ait eu le temps d’attirer une
clientèle importante.

Aucune restriction supplémentaire des droits fondamentaux n’est nécessaire. Les systèmes de contrôle
déjà en place pour d’autres raisons suffisent pour garder une trace des activités commerciales.

\end{itemize}

\section{La différence entre le commercial et le non-commercial}
La distinction entre la sphère des échanges non-marchands et la sphère marchande n'est pas facile à tracer, notamment sur Internet où interviennent de multiples intermédiaires lors des échanges (fournisseurs d'accès à Internet, plateformes de stockage ou de partage, moteurs de recherche).  
 
Il existe déjà un arsenal juridique développé par les tribunaux, pour distinguer les entreprises à but lucratif de celles qui ne poursuivent pas de telles fins. Il existe aussi plusieurs licences basées sur le droit d’auteur, comme les licences Creative Commons Attribution Non Commerciale, qui s’appuient sur cette distinction.

Néanmoins, afin de garantir un maximum de sécurité juridique, il convient de définir de manière stricte la notion d'échange non-marchand en la restreignant aux formes de partage décentralisées entre individus, n'impliquant pas le recours à des plateformes centralisées de stockage des fichiers. Par ailleurs, le bénéfice de la légalisation des échanges non-marchands ne pourrait être invoqué que si aucun usage commercial des oeuvres n'est effectué, y compris des usages indirects de type recettes publicitaires.  

En revanche, le fait de référencer et de signaler des liens vers des sites mettant à disposition des oeuvres doit rester libre. Cela permettra la mise en place d'annuaires de liens permettant l'accès aux oeuvres pour les internautes. Dans l'esprit de notre mesure, ces annuaires ne pourront être gérés que par des associations à but non lucratif qui ne dépendent pas d'une activité commerciale pour se financer.

De telles propositions favoriseraient des formes d'échanges décentralisés, utilisant l'architecture du réseau Internet en conformité avec sa nature et elles empêcheraient la reconstitution d’une industrie rentable du piratage comme Megaupload en avait donné l'exemple.

Tout comme actuellement, les entreprises ou associations qui proposeraient au téléchargement des fichiers protégés par le droit d'auteur et en généreraient un bénéfice substantiel resteraient obligées de reverser une partie de leurs bénéfices aux ayants-droits. 

\section{Une exception facilitant les usages pédagogiques et de recherche}

Prescripteurs de culture, les enseignants jouent un rôle fondamental en matière de sensibilisation à la création
culturelle et artistique. Avec leurs élèves, ils utilisent de plus en plus souvent des ordinateurs et Internet dans le cadre de leurs cours pour diffuser des œuvres protégées par le droit d'auteur. Cependant l'exception pédagogique actuelle repose sur des accords sectoriels complexes et prête à confusion. L’enchevêtrement de dispositions spécifiques conduit les enseignants à se situer aux marges du droit de la propriété littéraire et artistique.

Les chercheurs qui doivent fréquemment utiliser des oeuvres protégées dans le cadre de leurs travaux ont les mêmes soucis. Des usages innovants de recherche comme le datamining (fouille de bases de données) peuvent nécessiter d'utiliser à grande échelle des oeuvres protégées.  

La complexité des règles, source de lourdeur bureaucratique et d’insécurité juridique, est d’autant moins compréhensible que les enjeux financiers en cause sont limités.

La dépénalisation du partage non-marchand aurait comme conséquence immédiate de permettre aux enseignants et professeurs de l'Éducation nationale et du monde de la recherche de pouvoir librement utiliser et diffuser des travaux protégés par le droit d'auteur à leurs élèves ou collègues. En revanche les entreprises d'enseignement privé sans contrat avec l'État ne seraient protégées que par l'exception pédagogique actuelle. 

L'obligation que nous préconisons de placer sous licence libre et de publier les travaux des agents publics aurait aussi pour effet de favoriser le développement des ressources pédagogiques libres.

Une exception spécifique introduite au niveau européen et couvrant l'ensemble des usages d'oeuvres protégées effectués sans but commercial au sein des établissements d'enseignement et de recherche constituerait toutefois une garantie bien meilleure. S'agissant d'usages aussi légitimes que l'éducation et de la recherche, il importe que cette exception ne fasse pas l'objet d'une compensation financière. C'est déjà le cas au Canada depuis une réforme opérée en 2012 et cela devrait aussi l'être en Europe et en France.  



\chapter{Vingt ans de monopole commercial}\label{dur}

L’essentiel de l’industrie du divertissement actuelle est bâtie sur l’exclusivité commerciale des
travaux protégés. Nous voulons sauvegarder cette activité. Mais les durées d’exclusivité actuelles
sont absurdes. Aucun investisseur ne voudrait attendre un retour sur investissement aussi long.

\begin{mesure}[Réduction de la durée de protection]
 Nous souhaitons raccourcir les durées de protection à quelque chose de raisonnable à la fois du
point de vue de la société et des investisseurs et nous proposons vingt années à partir de la
publication.

Nous souhaitons la même période de protection pour tous les types de création.
\end{mesure}

\section{À productions différentes, durées différentes~?}

Ne serait-il pas judicieux d’avoir des durées de protection différentes pour les différents types de
création~? Vingt années de protection pour un programme informatique a certainement différentes
implications que vingt années pour un morceau de musique ou un film. Ne serait-il pas mieux
d’adapter les durées de protection selon ce qui est raisonnable pour chaque type de création~?

Le chiffre sur lequel il faut se prononcer est arbitraire. Cela pourrait être quinze ans ou vingt-et-un ans ou dix-huit ans sans changer grand chose. C'est donc partiellement une affaire de sensibilité personnelle et chacun va vouloir une durée de protection longue pour le type de création qui est le plus proche de sa sensibilité personnelle. Devoir définir des valeurs semi-arbitraires pour
chaque catégorie de production réduit donc les chances de trouver une
solution que l’on peut défendre de façon objective.

\section{Une durée rationnelle pour un investisseur}

Si l'on regarde la question du point de vue d’un investisseur, les choses deviennent
différentes. L’industrie de la musique a beau être très différente du secteur du logiciel, ils ont
quelque chose en commun. L’argent c’est de l’argent, quelque soit le secteur dans lequel on choisit d’investir.

Lorsqu’un investisseur prend la décision d’investir dans un projet, quelle que soit l’industrie – cela
peut être la musique, le cinéma, le logiciel grand public, ou tout autre chose – cet investisseur
établira sa stratégie avec une limite de temps pour obtenir un retour sur investissement. Si le
projet se développe selon les prévisions, il est supposé couvrir ses coûts et dégager des bénéfices
dans les x années. Si tel n’est pas le cas, c’est un échec.

X est toujours petit dans ce genre de prévisions. Que quelqu’un établisse une stratégie de
développement concernant un projet culturel dont le délai de retour sur investissement est supérieur
à trois ans est hautement improbable. Les personnes qui construisent des ponts, des réacteurs
nucléaires et autres infrastructures effectuent évidemment des investissements à plus long terme,
mais en dehors de ces industries les stratégies de développement de plus de trois ans ne sont
vraiment pas courantes.

C’est encore plus vrai dans le domaine de la culture. Qui peut prédire ce qui sera à la mode dans
deux ou trois ans, dans un paysage aussi changeant que celui de la culture~? On attend de la plupart
des projets culturels qu’ils génèrent des bénéfices dans l’année.

En considérant les durées de protections du point de vue d’un investisseur, on peut justifier le
fait d’avoir les mêmes durées pour toutes les créations. Le but de l’exploitation exclusive du droit
d’auteur est d’attirer les investisseurs vers le marché de la culture. Les investisseurs pensent
la même chose sans tenir compte de ce dans quoi ils sont en train d’investir.

Un projet doit dégager des bénéfices dans l’année ou les suivantes, autrement
c’est un échec. La faible probabilité que le projet que vous avez financé se révèle indémodable et
continue de générer des profits pendant des décennies est une chance pour l’investisseur, mais ça
n’a pas sa place dans un projet de développement sérieux.

\section{Pourquoi pas moins~?}

L’important c’est de se débarrasser des durées de protection actuelles d’une vie ou plus. Ces
longues périodes sont clairement néfastes pour la société puisqu’elles gardent la plupart de notre
héritage culturel commun bloqué même longtemps après que la majorité des productions aient perdu
toute valeur commerciale pour les ayants-droits. C’est une perte sèche économiquement parlant et un
scandale culturellement parlant.

Si les durées de protections étaient réduites à 20 ans, cela résoudrait la plupart des problèmes «
du trou noir du 20\ieme siècle~», et permettrait aux bibliothécaires et archivistes de commencer
l’urgente tâche de préservation des créations du 20\ieme siècle qui se dégradent dans les archives, en
les numérisant. Cinq ou dix ans seraient plus appropriés pour favoriser l’archivage, mais 20 ans
devraient convenir.

Dans le même temps, 20 ans est encore suffisant pour nourrir le rêve plaisant (mais hautement
improbable) de créer un succès majeur indémodable qui génère des revenus durant des décennies. Si
votre prochain projet trouve le bon filon et vous propulse soudainement sous les feux des
projecteurs pour longtemps tel Paul Mc Cartney ou ABBA, 20 ans devraient
être plus que suffisants pour que vous deveniez très riche et que vous n’ayez plus jamais jamais à
vous soucier d’argent.

\chapter{Remodeler le financement de la culture}\label{financement}

\section{Trois sources d'investissement principales}
Les investissements privés, les subventions publiques et le mécénat individuel ou collectif sont les principales sources de financement de la culture.

Les investissements privés comme ceux opérés par les maisons d'édition ou les studios d'enregistrement doivent continuer. C'est pourquoi nous défendons au chapitre \ref{depen} qu'il ne faut pas dépénaliser la diffusion à but lucratif. Il faut permettre aux modèles économiques existants de perdurer.

Le rôle des subventions publiques dans la création aujourd'hui est essentiel. L'État, les régions et les collectivités locales investissent dans la culture car ils considèrent qu'il est du ressort des institutions publiques de soutenir l'offre culturelle à destination des citoyens. Nous soutenons pleinement cette idée. Nous considérons de plus qu'il faut que ces institutions exigent en retour de leurs partenaires que les contenus ainsi produits soient plus facilement et rapidement diffusables et réutilisables par les citoyens qui les ont financés. C'est le sens des propositions des chapitres \ref{licencelibre} et \ref{dompub}. 

Le mécénat, qu'il soit individuel ou collectif, doit surtout être réglementé par le droit des contrats pour éviter les fraudes fiscales ou les arnaques au don. Ainsi, les plates-formes de financement participatif doivent être encouragées et non freinées par des incertitudes juridiques sur le statut fiscal des dons ou investissements. Cela peut être fait en leur accordant des statuts particuliers auprès des autorités de régulation de contrôle prudentiel ou de régulation des marchés. Par ailleurs la loi doit prendre en compte ces nouveaux types de financements et ne pas honorer les ventes ou investissements reçus via ces plates-formes doit pouvoir être facilement sanctionné. De telles mesures ne sont cependant pas centrales pour le financement de la création. Le financement participatif se développe rapidement même en l'absence d'un cadre juridique plus précis.

\section{Les sociétés de gestion des droits}

La logique des sociétés de gestion relève de l'investissement privé. Un artiste décide individuellement de déléguer la gestion de ses droits à une entité qui lui reverse une partie des droits perçus au prorata de la diffusion et de la réutilisation de ses œuvres. L'utilité de telles entités est certaine car elle simplifie la collecte des droits. 

Avec les réformes que nous proposons, leur rôle est amené à s'élargir. Il serait logique qu'elles veillent au respect des licences libres amenées à se développer selon le chapitre \ref{licencelibre}, qu'elles perçoivent au nom de leurs sociétaires les revenus issus de la commercialisation des œuvres dérivées des leurs comme proposé au chapitre \ref{remix} et qu'elles participent au maintien du registre défendu au chapitre \ref{registre}. De plus, dans le cas où l'État déciderait de financer la culture via le prélèvement d'une redevance qui s'ajouterait à un abonnement à Internet, les sociétés de gestion pourraient aider à répartir une partie des revenus issus de cette redevance.

Cependant, les sociétés de gestion sont actuellement entachées par plusieurs soucis majeurs, dont:

\begin{itemize}
\item Les données de répartition ne sont pas publiées par les sociétés de gestion.
\item Elles s'arrogent trop facilement des droits de perception sur des œuvres même quand elles n'en détiennent pas.
\item Elles se basent parfois sur la possibilité de diffusion des œuvres au public pour établir une redevance et non sur la diffusion concrète.
\item Elles ne sont pas démocratiques en raison de votes censitaires ou d'organisations collégiales.
\item Elles monopolisent la gestion des droits des artistes qui y sont inscrits même quand ceux-ci voudraient s'y opposer pour une partie de leurs œuvres ou voudraient accorder des dérogations.
\end{itemize}

Ces travers dérivent à notre avis d'obligations légales de transparence trop faibles ainsi que d'un manque de concurrence entre les sociétés de gestions qui conduit à des abus de type monopolistique. Chaque société de gestion monopolise l'exploitation des droits patrimoniaux dans un domaine spécifique de la vie culturelle comme le cinéma ou la musique. 

\begin{mesure}[Sociétés de gestion]
Nous proposons de~:

\begin{description}
\item soit
	\begin{itemize}
	\item ouvrir au marché la redistribution des droits patrimoniaux des auteurs
	\item imposer aux sociétés de gestion la transparence des critères de répartition et des statistiques de répartition selon ces critères.
	\item sanctionner plus sévèrement les perceptions abusives
	\end{itemize}

\item soit 
	\begin{itemize}
	\item imposer la règle «~un sociétaire pour un vote~»
	\item laisser aux sociétaires le choix de décider œuvre par œuvre des règles d'inclusion dans le catalogue des sociétés de gestion
	\item imposer aux sociétés de gestion la transparence des critères de répartition et des statistiques de répartition selon ces critères.
	\item sanctionner plus sévèrement les perceptions abusives
	\end{itemize}
\end{description}
\end{mesure}

Les sociétés de gestion devraient dans tous les cas continuer à être agréées par l'État car leurs agents doivent être assermentés pour pouvoir constater la matérialité des infractions au droit d'auteur. L'existence d'un registre comme proposé au chapitre \ref{registre} doit permettre de limiter les perceptions abusives.

Dans le cas de la libéralisation du marché de la perception des droits d'auteurs, un éclatement complet du marché est impossible car une société de gestion ne peut correctement faire respecter les droits de ses sociétaires que lorsqu'elle obtient une taille suffisante. En revanche la scission des sociétés de gestion actuelles en plusieurs sociétés moins prônes aux abus et plus à l'écoute de leurs sociétaires et des besoins du public serait le résultat souhaité de la libéralisation.



%Annexes
\appendix
\chapter{Index des mesures}\label{index}

\listtheorems{mesure}

\chapter{Sources}


Les œuvres que j'ai utilisées sont~:

\begin{itemize}
\item les billets \livre{I Have A Dream : une loi pour le domaine public en France~!} et \livre{Pour un droit au mashup, mashupons la loi !} disponibles à \url{https://scinfolex.wordpress.com/2012/10/27/i-have-a-dream-une-loi-pour-le-domaine-public-en-france/} et \url{http://scinfolex.wordpress.com/2013/06/20/pour-un-droit-au-mashup-mashupons-la-loi/} dans le domaine public de Lionel Maurel alias Calimaq pour les chapitres \ref{dompub} et \ref{remix}.
\item les \livre{Eléments pour la réforme du droit d'auteur et des politiques culturelles liées} disponibles à \url{https://www.laquadrature.net/fr/elements-pour-la-reforme-du-droit-dauteur-et-des-politiques-culturelles-liees} de la Quadrature du Net pour les chapitres \ref{verrous}, \ref{dompub}, \ref{licencelibre}, \ref{registre}, \ref{depen} et \ref{financement}.
\item l'article \livre{Licence libre} de Wikipedia disponible à \url{http://fr.wikipedia.org/wiki/Licence_libre} sous licence \href{http://creativecommons.org/licenses/by-sa/2.0/fr/}{CC-BY-SA} et l'article \livre{Définition} de Actions Open Data disponible à \url{http://actionsopendata.org/l-open-data/definition/} sous licence \href{http://creativecommons.org/licenses/by-sa/2.0/fr/}{CC-BY-SA} pour le chapitre \ref{licencelibre}.
\item le rapport Lescure disponible à \url{http://culturecommunication.gouv.fr/Actualites/A-la-une/Culture-acte-2-75-propositions-sur-les-contenus-culturels-numeriques} pour les propositions des chapitres \ref{verrous}, \ref{licencelibre} et \ref{depen}.
\item le rapport Bernard Lang \livre{L'exploitation des œuvres orphelines dans les secteurs de l'écrit et de l'image fixe} disponible à \url{http://bat8.inria.fr/~lang/orphan/oeuvres-orphelines-BLang.pdf} pour le chapitre \ref{remix}.
\item les positions de l'AFUL sur la question des œuvres orphelines et sur celle des licences libres disponibles à \url{http://aful.org/droit-auteur/index/oeuvres-orphelines/} et \url{http://aful.org/association/positions} pour les chapitres \ref{remix} et \ref{licencelibre}.
\item le pacte du logiciel libre proposé par l'April disponible à \url{http://www.april.org/pacte-du-logiciel-libre} pour le chapitre \ref{licencelibre}.
\item ma traduction du billet de blog \livre{Mein Rad} de Marcel-André Casasola Merkle disponible à \url{http://traductions.sploing.fr/politique/2012/05/09/mon-velo/} sous licence \href{http://creativecommons.org/licenses/by-sa/2.0/fr/}{CC-BY-SA} pour le chapitre \ref{cycliste}.
\item ma traduction du livre du Parti Pirate Suédois \livre{The Case for Copyright Reform} disponible à \url{http://reformedroitauteur.sploing.fr/} sous licence \href{http://creativecommons.org/licenses/by-sa/2.0/fr/}{CC-BY-SA} pour l'introduction générale en page \pageref{premintro} et les chapitres \ref{pater}, \ref{verrous}, \ref{remix}, \ref{registre}, \ref{depen} et \ref{dur}. 
\item ma traduction du site de la campagne de la Digitale Gesellschaft pour le droit au remix disponible à \url{http://politiquedunetz.sploing.fr/2013/06/campagne-de-la-digitale-gesellschaft-pour-le-droit-au-remix/} pour le chapitre \ref{remix}.
\end{itemize}

\nocite{lessig2004culture}
\nocite{florent2004bon}
\nocite{benkler2009richesse}
\nocite{benkler2009richesse}
\nocite{manach2010vie}
\nocite{lessig2005avenir}
\nocite{raymond1998cathedrale}
\nocite{stallman2010richard}
\nocite{sagot2002propriete}
\nocite{perline2004bataille}
\nocite{aigrain2008internet}
\nocite{hadopi}
\nocite{opendata}
\nocite{smiers2011monde}


\bibliography{biblio}
\end{document}
