\chapter*{Introduction}\label{premintro}
\addcontentsline{toc}{chapter}{Introduction}

Le système du droit d’auteur est aujourd’hui déphasé. Il criminalise une génération entière dans une tentative désespérée d’arrêter le progrès technologique. Pourtant le partage de fichiers continue à s’accroître et les remix continuent à fleurir. Ni la propagande ni les techniques d’intimidation ni le durcissement des lois n’ont pu arrêter ce développement.

Il n’est plus possible de renforcer les mesures de protection du droit d'auteur sans violer des droits humains fondamentaux. Tant que les individus pourront communiquer en privé, ils s’en serviront pour partager des contenus soumis au droit d’auteur. Le seul moyen de limiter le partage de fichiers c’est de supprimer le droit à la communication privée.

Nous voulons une société où la culture prospère, où les artistes et les créateurs ont une chance de vivre de leur art. Heureusement, il n’y a aucune contradiction entre le partage et la culture. Une décennie de partage intensif nous l’a prouvé. L'Histoire nous l'a prouvé.

Quand les bibliothèques publiques sont apparues en Europe il y a 150 ans, les éditeurs y étaient extrêmement opposés. Leurs arguments étaient les mêmes que ceux dont on se sert aujourd’hui dans le débat sur le partage. Si le peuple pouvait accéder gratuitement aux livres, les auteurs ne pourraient plus vivre de leur art. C'était la mort du livre.

Nous savons à présent que les arguments contre les bibliothèques publiques étaient faux. Les livres prêtés par les bibliothèques n'ont jamais fait s'écrouler les ventes des éditeurs et toute la société bénéficie de l'accès libre à la culture. 

Internet est la plus merveilleuse bibliothèque publique jamais créée. Pour tous, y compris ceux aux moyens économiques limités, l’accès à toute la culture de l’humanité n’est plus qu’à un simple clic. Cette liberté de circulation de l'information est une révolution. L'État doit l'encourager en la protégeant et en participant lui-même à la production et à la diffusion de cette culture. C'est dans son intérêt comme dans celui de ses citoyens. 

L'État doit accepter le progrès pour la diffusion de la connaissance qu'a apporté Internet. Il doit abandonner l'idée qu'interdire des pratiques majoritaires comme le partage à but non lucratif est porteur d'avenir. Dans une démocratie les lois sont d'abord ce que les citoyens en font. Il est absurde qu'un État démocratique lutte contre la majorité de ses citoyens.

Notre État doit aussi accompagner et encourager pour le bien de tous le progrès. Une vraie réforme du droit d'auteur n'est pas qu'une adaptation marginale de quelques restrictions obsolètes et dangereuses dans les lois actuelles sur le droit d'auteur. Elle implique l'ajout de nouvelles contraintes pour les administrations et les artistes en faveur de la diffusion du maximum d'information possibles. 

Ces contraintes doivent lui permettre de co-créer avec ses citoyens les services dont ils ont besoin. Pour prendre les décisions les plus avisées et créer les œuvres les plus innovantes, ses citoyens ont besoin de données libres. Les données sont la matière première de l'économie de la connaissance. C'est pourquoi l'État doit fournir à ses citoyens autant d'informations et d'œuvres culturelles libres qu'il peut. Des mécanismes de gestion collective assainis peuvent permettre de favoriser ces nouvelles pratiques.

Les propositions que nous faisons sont résumées par les lignes directrices suivantes~:

\begin{enumerate}
\item Conserver l'essentiel des droits moraux et garder l'exclusivité commerciale pour permettre aux modèles
économiques actuellement viables de le rester.
\item Laisser s'épanouir la culture du remix et de l'échange à but non-lucratif.
\item Diminuer la durée de protection.
\item Enregistrer régulièrement les œuvres pour constituer un catalogue des métadonnées. 
\item Protéger et promouvoir le domaine public et l'accès libre à des œuvres et données libres.
\end{enumerate}

Il reste aux politiques à s'emparer de ces idées pour les transformer en lois. Les plus pressés trouveront un index des réformes proposées en page \pageref{index}.
