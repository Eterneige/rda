\chapter{Le manifeste du poète}\label{poete}

Il n’y a pas de concept plus rabâché dans les discours médiatiques que celui de génération. J’ai déjà essayé une fois de compter les «~générations~» autoproclamées dans les journaux ces 10 dernières années. Je pense en avoir trouvé au moins douze. Mais elles avaient toutes une chose en commun~: elles n’existent que sur le papier. Dans la réalité personne n’a vécu
cette impulsion à la fois unique en son genre, tangible et inoubliable, cette expérience commune à travers laquelle nous resterions différentiables de toutes les générations précédentes. Nous
cherchions loin à l’horizon, mais la mutation fondamentale était passée inaperçue. Elle se cachait dans les câbles grâce auxquels la télévision a embrassé le pays, dans l’éclipse du réseau fixe par
celui mobile et avant tout dans l’accès à Internet généralisé. Ce n’est que maintenant que nous comprenons tout ce qui a changé dans les 15 dernières années.

Nous, les enfants d'Internet, qui avons grandi avec Internet et sur Internet, nous sommes une génération qui satisfait aux critères du concept. Il n’y a pas eu de déclic mais plutôt une
métamorphose de la vie. Ce qui nous unit à présent n’est pas un contexte culturel commun et déterminé mais le sentiment que nous pouvons définir librement ce contexte et ses cadres.

Pendant que j’écris, je sais bien que j’abuse du mot «~Nous~». Parce que notre «~nous~» est changeant, flou. Avant on aurait dit~: temporaire. Quand je dis «~nous~», je pense «~beaucoup d’entre nous~»
ou «~quelques uns d’entre nous~». Quand je dis «~nous sommes~», je pense «~il arrive que nous soyons~». C’est pourquoi je dis «~nous~» pour pouvoir parler de nous.

\section{Nous avons grandi avec Internet et sur Internet}
C’est pourquoi nous sommes différents. C’est le point crucial et à vrai dire pour nous l’étonnante différence~: Nous ne «~surfons~» pas sur Internet, et
Internet n’est pas pour nous un «~lieu~» ou un «~espace virtuel~». Internet n’est pas pour nous une extension externe de notre réalité, mais en fait partie~: une couche invisible mais toujours
présente qui s’entrelace à notre environnement physique, une sorte de seconde peau.

Nous n’utilisons pas Internet, nous vivons dedans et avec. Si nous devions vous expliquer à vous, les analogiques, notre «~Bildungsroman~»\footnote{«~Roman initiatique~». Style littéraire allemand du
18\ieme siècle qui thématise l'apprentissage de la vie d'un jeune personnage.}, nous vous dirions que toutes les
expériences essentielles que nous avons faites avaient le réseau en commun. En ligne, nous nous sommes créés des amis et des ennemis, nous avons préparé nos anti-sèches, nous avons planifié des
soirées et des sessions de travail, nous sommes tombés amoureux et nous nous sommes séparés.

Internet n’est pas pour nous une technologie que nous devions apprendre et que nous avons intégrée d’une manière ou d’une autre. Le réseau est avant tout un processus continu qui évolue en permanence
sous nos yeux, avec nous et à travers nous. Les technologies se créent et disparaissent dans notre environnement, les sites web naissent, se déploient et meurent, mais le réseau subsiste parce que
nous
sommes le réseau, nous qui communiquons bien plus efficacement que jamais dans l’histoire de l’humanité.

Nous avons grandi sur Internet, c’est pourquoi nous pensons différemment. Pouvoir trouver une information est pour nous aussi évident que pour vous pouvoir trouver une gare ou une poste dans une ville
inconnue. Quand nous voulons quelque chose, comme les premiers symptômes de la varicelle, les raisons du naufrage de l’«~Estonie~» ou savoir pourquoi notre facture d’eau semble aussi suspicieusement
haute, nous prenons les devants avec la sûreté d’un automobiliste guidé par un GPS.

Nous connaissons beaucoup d’endroits où trouver les informations désirées, nous savons comment on y arrive et nous pouvons juger de leur fiabilité. Nous avons appris à accepter que nous trouverons
rarement une réponse mais bien plutôt plusieurs. Nous extrayons de cette pluralité l’option la plus vraisemblable pour ignorer les autres. Nous sélectionnons, filtrons, nous souvenons et sommes
prêts à échanger ce que nous savons déjà contre quelque chose de neuf, de meilleur, quand nous butons contre un obstacle.

Pour nous Internet est une sorte de disque dur externe. Nous ne retenons pas de définition précise~: les dates, les montants, les formules, les paragraphes et les définitions exactes. Un résumé avec
le cœur de l’affaire nous suffit, et nous le travaillons pour le relier avec d’autres informations. Si nous avons besoin de détails, nous les cherchons dans les secondes qui suivent.

Nous n’avons pas besoin d’être des experts dans tout ce que nous connaissons parce que nous trouvons les hommes qui en ont fait leur spécialité et que nous pouvons croire. Les autres hommes ne
partagent pas leur expertise avec nous pour de l’argent mais plutôt parce qu’ils sont comme nous convaincus que l’information connaît un flux continuel et veut être libre, que nous profitons tous de
l’échange. Et ce tous les jours~: pendant nos études, au travail, lors de la résolution de problèmes quotidiens ou lorsque ça nous intéresse. Nous savons comment la concurrence fonctionne et l’aimons.
Mais notre compétition, notre vœu d’être différent, se base sur la capacité de manipuler et interpréter les informations, pas sur leur monopolisation.

\section{La participation à la culture n’est pas notre occupation des jours de fête}
La culture globale est le socle de notre identité, plus importante que notre compréhension particulière comme
tradition, les histoires de nos aînés, le statut social, l’origine ou même notre langue. Dans l’océan des événements culturels nous pêchons ce que bon nous semble, nous interagissons avec, notons et
sauvegardons nos évaluations sur des sites web dédiés qui proposent d’autres albums, films ou jeux qui pourraient nous plaire.

Nous regardons avec d’autres collègues certains de ces films, de ces séries ou vidéos, ou alors avec des amis du monde entier. Pour certains contenus notre appréciation ne sera jamais partagée qu’avec
un petit nombre de personnes que parfois nous ne verrons peut-être jamais dans la vie réelle. C’est pourquoi nous avons le sentiment que notre culture devient à la fois globale et individuelle. C’est
la raison pour laquelle nous avons besoin d’y accéder librement.

Cela ne veut pas dire que nous exigeons un accès gratuit à tous les biens culturels, même si quand nous créons quelque chose, bien souvent nous le mettons simplement en circulation. Nous comprenons
que la créativité demande toujours des efforts et de l’investissement, et ce malgré la démocratisation des techniques de montage audio ou vidéo. Nous sommes prêts à payer, mais les renchérissements
gigantesques des intermédiaires nous paraissent bêtement et simplement inadaptés. Pourquoi devrions nous payons pour la copie d’une information qui peut pourtant être copiée parfaitement très
rapidement, sans changer seulement d’un iota la valeur de l’original~? Si nous ne recevons que l’information brute, nous demandons un prix adapté. Nous sommes prêts à payer plus, mais alors nous
attendons aussi plus~: un emballage intéressant, un gadget, une meilleure qualité, la possibilité de pouvoir le regarder tout de suite, ici et maintenant, sans attendre la fin du téléchargement. Nous
pouvons même montrer de la gratitude et donner à l’artiste (puisque l’argent ne correspond plus qu’à des suites de chiffres sur un écran, payer est presque devenu un acte symbolique duquel les deux
partis devraient profiter), mais les objectifs de vente de quelque sorte que ce soit ne nous intéressent pas du tout. Ce n’est pas notre faute si votre modèle économique ne fait plus aucun sens dans
sa forme traditionnelle et si vous vous décidez à défendre votre modèle daté au lieu d’accepter les nouvelles exigences et d’essayer de nous fournir plus que ce que nous pourrions avoir autrement.

Encore une chose~: Nous ne voulons pas payer pour nos souvenirs. Les films qui datent de notre enfance, la musique qui nous a bercée pendant 10 ans~: Dans une mémoire mise en réseau, ce ne sont plus
que des souvenirs. Les rappeler et les échanger, les redévelopper, c’est pour nous aussi normal que pour vous les souvenirs de «~Casablanca~». Nous trouvons sur Internet les films que nous avons vus
enfants. Pouvez-vous vous imaginer que quelqu’un vous poursuive pour ça en justice~? Nous non plus.

\section{Ce qui nous importe le plus, c’est la liberté}
Nous sommes habitués à payer automatiquement nos factures tant que l’état de notre compte bancaire le permet. Nous savons que nous devons seulement remplir en ligne un formulaire et signer un contrat
livré par la poste quand nous ouvrons un compte ou voulons changer d’opérateur téléphonique. C’est pourquoi, en tant qu’utilisateur de l’État, nous sommes de plus en plus énervés par son interface
utilisateur archaïque. Nous ne comprenons pas pourquoi nous devrions remplir plusieurs formulaires papiers où le plus gros peut comporter plus de cent questions. Nous ne comprenons pas pourquoi nous
devons justifier d’un domicile fixe (il est absurde de devoir en avoir un) avant de pouvoir entreprendre d’autres démarches, comme si les administrations ne pouvaient pas régler ces choses sans que
nous intervenions.

Nous avons perdu la conviction née dans la crainte de nos parents que les trucs administratifs sont d’une importance capitale et que les affaires réglées par l’État sont sacrées. Ce respect ancré dans
la distance entre le citoyen solitaire et la hauteur majestueuse dans laquelle réside la classe dominante, à peine visible là-haut dans les nuages, nous ne l’avons plus. Notre compréhension de la
structure sociale est différente de la leur~: La société est un réseau, pas une pyramide. Nous sommes habitués à pouvoir adresser la parole à presque n’importe qui, qu’il soit journaliste, maire,
professeur d’université ou star de la pop, et nous n’avons pas besoin de qualifications qui iraient de pair avec notre statut social. Le succès d’une interaction tient uniquement à l’appréciation par
les autres de l’importance du contenu de notre message et de la pertinence d’y répondre. Et puisque nous avons le sentiment, grâce à la collaboration et à des disputes incessantes où nous défendons
contre la critique nos arguments, que nos opinions sont les meilleures, pourquoi ne pourrait-on pas attendre de dialogue sérieux avec le gouvernement~?

Nous ne sentons pas de respect religieux pour les «~institutions démocratiques~» dans leur forme actuelle, nous ne croyons pas à l’irrévocabilité de leurs rôles comme tous ceux qui considèrent que les
institutions démocratiques sont des objets de vénération qui se construisent d'eux-mêmes et à leur propre fin. Nous n’avons pas besoin de monuments. Nous avons besoin d’un système qui réponde
à nos
attentes, d’un système transparent et en état de marche. Et nous avons appris que le changement est possible, que tout système difficile à manier peut être remplacé par un plus efficace, qui soit
mieux adapté à nos exigences et laisse plus de marge de manœuvre.

Ce qui nous importe le plus, c’est la liberté. La liberté de s’exprimer, d’accéder à l’information et à la culture. Nous croyons qu’Internet est devenu ce qu’il est grâce à cette liberté et nous
pensons que c’est notre devoir de défendre cette liberté. Nous devons cela aux générations futures comme nous leur devons de protéger l’environnement.

Il est possible qu’aucun nom adapté n’existe pour désigner ce que nous voulons, ou que nous ne soyons pas encore tout à fait conscient qu'il s'agit là d'une vraie et réelle démocratie. Une
démocratie qui n’a peut-être jamais
été rêvée par vos journalistes.
