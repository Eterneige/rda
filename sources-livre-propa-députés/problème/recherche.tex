\chapter{L'enthousiasme du chercheur}\label{cherch}

\section{La recherche de la vérité n'est pas celle du profit financier}
Tout le monde peut comprendre pourquoi le piratage de la musique peut être polémique. Les musiciens gagnent leur vie en vendant leurs œuvres. Partager des copies de leur travaux semble devoir les priver d'une source de revenus. Les pirates y répondent en faisant remarquer qu'il est moins rentable d'être inconnu que d'être piraté. Mais les chercheurs ne gagnent pas leur vie en vendant leurs articles scientifiques. Je n'entrerai donc pas dans le débat du piratage ici. 

Imaginons qu'un groupe d'auteurs ne gagnent pas leur vie en vendant leurs œuvres et en autorise le partage. Ce n'est pas parce qu'ils sont riches qu'ils suivent ce chemin étrange. C'est parce que leurs sujets d'intérêts, leurs motivations personnelles et les circonstances institutionelles les amènent à écrire pour être lus. L'important est l'impact, l'influence. Pas l'argent. Leurs carrières dépendent plus de la taille de leur lectorat que de l'existence de lecteurs payants.

Il n'y a pas beaucoup de journalistes ou de romanciers dans ce groupe. Mais c'est là qu'on trouve de plus en plus de chercheurs des sciences dures ou humaines.

Les chercheurs ne gagnent pas leur vie en vendant leurs articles à des revues académiques où ceux-ci sont vérifiés par leurs pairs. La plupart du temps, ces revues ne versent aucune redevance aux auteurs, mais leur font au contraire payer pour que leur article soit examiné. S'il arrive que les chercheurs veuillent gagner de l'argent en écrivant des livres,  ce n'est clairement pas le cas des articles de revues. De plus en plus de chercheurs ont compris cela et encouragent le partage de leurs articles publiés. 

La plupart des chercheurs travaillent dans des institutions qui les payent pour chercher. Les salaires que celles-ci leur payent leur permettent tout d'abord de se nourrir. Ils leur permettent aussi de ne pas être dépendant du marché du livre. Des musiciens qui n'enregistreraient que quelques morceaux tous les ans qu'ils n'enverraient qu'à quelques amis sélectionnés sur le volet mourraient de faim. Des chercheurs qui peaufinent pendant deux mois un seul article sur un sujet inexploré et inconnu continue à vivre. Ces chercheurs peuvent défendre des opinions peu populaires ou s'intéresser à des sujets spécialisés sans dommage financier. Ils peuvent s'intéresser à ce qui leur paraît vrai sans se préoccuper de savoir si ça se vend.

Ce système fonctionne très bien comme cela. Si les universités ne permettaient pas aux chercheurs de vivre décemment, ceux-ci ne défendraient que des opinions populaires. La recherche de la vérité serait remplacée par la recherche du profit. 

\section{La libre diffusion et réutilisation permet une meilleure reconnaissance}
Les premières revues scientifiques qui furent lancées à Paris et Londres il y a 350 ans supplantèrent rapidement les lettres et les livres utilisés jusque là pour diffuser la recherche. Les revues permettaient d'atteindre un public plus large et étaient imprimées plus vite que les livres. La réputation de certaines revues a permis de hiérarchiser les chercheurs entre eux. C'étaient les chercheurs qui étaient en demande de publication. Cela explique que les revues ne leur payent rien. 

Le système a bien fonctionner jusqu'à ce que le prix des revues finisse par augmenter plus vite que l'inflation à partir de 1970. Ces prix n'ont cessé d'augmenter exponentiellement depuis quarante ans. Même les universités les plus riches sont obligées d'annuler certains abonnements chaque année pour raisons budgétaires. Harvard par exemple a récemment demandé à ses facultés de limiter le nombre de leurs abonnements à cause des prix démentiels. Ainsi Elsevier, un éditeur majeur du marché, a déclaré une marge de 36\% en 2010, alors qu'ExxonMobil ne déclarait \emph{que} 28\%.  Certains prennent cela pour une crise du marché des revues mais c'est bien plutôt une crise de l'accès à la recherche.

C'est pourquoi en 2002 l'Initiative pour le Libre Accès de Budapest a donné un nom au partage libre de la recherche~: Accès libre. Elle a décrit quelques stratégies pouvaient être poursuivies pour améliorer l'accès libre. Voici la définition de l'accès libre selon cette initiative~:


\begin{quotation}Par «~accès libre~» aux articles de revues académiques, nous entendons sa mise à disposition gratuite sur l'Internet public, permettant à tout un chacun de lire, télécharger, copier, transmettre, imprimer, chercher ou faire un lien vers le texte intégral de ces articles, les disséquer pour les indexer, s'en servir de données pour un logiciel ou s'en servir à toute autre fin légale, sans barrière financière, légale ou technique autre que celles indissociables de l'accès et l'utilisation d'Internet. La seule contrainte sur la reproduction et la distribution et le seul rôle du droit d'auteur dans ce domaine devraient être de garantir aux auteurs un contrôle sur l'intégrité de leurs travaux et le droit à être correctement reconnus et cités.
\end{quotation}

En un mot, la seule obligation pour l'utilisation d'un article en accès libre est de citer son auteur. 

Pourquoi donc cette restriction~? L'objectif de l'accès libre est d'abattre les barrières à l'utilisation académique de la littérature académique sans porter préjudice aux chercheurs. Académiquement, réutiliser les travaux des autres sans les citer n'est pas honnête. Les carrières et influences des auteurs dépendent de la bonne attribution des bons travaux aux bonnes personnes. 

Inversement, quel est le but de la suppression des restrictions classiquement imposées par le droit d'auteur au nom du bien de ces auteurs~? La réponse est que partager la connaissance accélère la recherche. Cela aide les chercheurs à trouver les travaux dont ils ont besoin pour progresser et à se faire connaître pour leurs propres travaux. La connaissance a toujours été un bien commun dans le sens théorique du terme~: c'est un bien idéel et non-rival qui peut être consommé sans dommage par tous. Une politique du libre n'est que la traduction concrète de cette idée. 

Ce sont les auteurs qui décident de mettre leurs livres en accès libre. Nous chercheurs n'attendons pas des musiciens et romanciers qu'ils décident de mettre leurs œuvres sous licence libre, même si certains acceptent de céder leurs droits de temps à autre. Ce qui nous intéresse sont les travaux académiques et notre salaire ne dépend pas de la vente de nos articles. 

\section{Les œuvres en accès libre sont en croissance exponentielle}

Il y a deux manières répandues de promouvoir l'accès libre. Soit via les revues elles-même (accès doré dans le jargon), soit via des sites de dépôt (accès vert). 

Les revues d'accès libre se financent de multiples manières, car bien sûr accès libre ne veut pas dire édition gratuite. Les éditeurs sont parfois à but lucratif et parfois non. Certains utilisent une méthode de revue par les pairs classique mais d'autres sont plus innovants. Le répertoire des revues en accès libre compte à présent plus de 8000 revues en accès libre.

Les sites de dépôt sont des simples bases de données de contenu numérique. En France, le site HAL héberge par exemple plus de 220~000 travaux. Certains sites ne diffusent qu'un seul champ, d'autres sont multi-disciplinaires. Certains sont ouverts à tous les chercheurs, d'autres ne publient que les travaux d'une université ou d'un centre de recherche. Plus de 60\% des revues donnent leur accord pour que les articles validés soient déposés dans de tels sites. Ces sites peuvent par ailleurs héberger des images, des présentations, de l'audio, du code source de logiciel, etc… Il en existe plus de 2100 actuellement.

30\% des revues modérées actuelles sont en accès libre. C'est énorme par rapport à il y a dix ans et le taux de conversion des indécis décolle. Mais c'est toujours une petite partie de l'ensemble de l'offre de revues. C'est pourquoi imposer aux chercheurs de publier dans des revues en accès libre leur limiterait nettement le choix.

Cependant les sites d'accès libre vert sont en pleine expansion. La plupart des universités encouragent cette pratique si elles n'y obligent pas. À Harvard ou au MIT, non seulement la faculté oblige à ce dépôt mais impose que tous les travaux à venir le soient et préempte ainsi les contrats des éditeurs. Ainsi les politiques universitaires peuvent obliger à des changements massifs de comportement. Il existe à présent plus de 150 dépôts de travaux en accès libre purement universitaires dans le monde. 

Certains pourraient croire que cette passion pour la publication en accès libre a été imposée par les administrations centrales dans les universités. Rien n'est plus faux. Au MIT et à Harvard, toutes les facultés ont voté à l'unanimité les politiques de ces facultés. 

Les politiques nationales de certains pays comme l'Angleterre ou le Danemark sont plus avancées. Le Danemark a purement et simplement imposé que tous les travaux de recherche financés par l'argent public soient diffusés en accès libre. En Angleterre cela ne saurait tarder, car le Higher Education Funding Council for England a annoncé que tous les travaux publiés dans le cadre du prochain Research Excellence Framework de 2014 seront en accès libre. La Commission Européenne a annoncé que les travaux qu'elle financeraient serviront une politique d'accès libre et a appelé les États-membres à faire de même.

La plupart des gens qui surfent sur Internet ne réalisent pas que ce réseau a initialement été développé par des chercheurs pour partager leurs recherches mais que cet objectif a été laissé de côté avec l'arrivée du commerce. Il n'a cependant pas été laissé de côté et l'accès libre aux publications scientifiques est en train de devenir la norme des communications académiques. 

La prochaine étape de ce processus, déjà en cours, sera la réutilisation de tous ces travaux de recherche nouvellement publiés à des fins industrielles. Les produits et services innovants de l'avenir sont déjà souvent construits à partir de recherches financés par l'État. Augmenter le nombre de travaux librement disponibles pour le grand public grâce à Internet donnera une nouvelle dynamique à ce cercle vertueux. Plus que jamais, la recherche publique remplira alors complètement ses buts~: améliorer nos connaissances et grâce à celles-ci améliorer notre qualité de vie.


