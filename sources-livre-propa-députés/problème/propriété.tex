\chapter{La colère du propriétaire}\label{proprio}

\section{Biens tangibles et intangibles}

Les biens que l'on peut posséder sont a minima les biens tangibles, ceux que l'on peut toucher. Ce sont les meubles et immeubles. Les immeubles comprennent les biens tangibles qui ne peuvent être déplacés, comme les maisons et les terres. Les meubles concernent tout ce qui peut l'être, comme les meubles de la maison, les voitures, les montres…

Tous les objets tangibles, qu'il s'agisse de biens acquis ou créés, de meubles ou d'immeubles ou de nos propres corps, peuvent être soumis à un contrôle légitime par des individus précis appelé droit de propriété.

Lorsque l'on passe des objets tangibles et corporels à des objets intangibles la situation est moins claire. Est-ce que nous avons des droits sur nos créations intellectuelles~? Est-ce que le système légal devrait protéger ces droits~? Nous verrons que les droits patrimoniaux amènent surtout une réduction de nos droits de propriété sur des objets tangibles. Ils restreignent donc nos libertés. À l'ère d'Internet, cette restriction ne peut pas être justifiée. Pour cette raison ils doivent être abrogés ou diminués. 

\section{Droit de propriété et rareté}

Jetons un coup d'œil sur l'idée du droit de propriété pour les biens tangibles. Qu'est-ce qui est intéressant dans les biens tangibles et rend leur appropriation par certains et non d'autres intéressante~? 

C'est leur rareté qui est responsable de la situation. C'est un fait que des conflits d'exclusivité d'usage existent ou peuvent exister autour des biens tangibles. C'est cette possibilité de conflit autour d'eux qui rend les biens tangibles rares, et incite à vouloir créer des règles pour gérer leur usage. Ainsi, la fonction sociale fondamentale et éthique des droits de propriété est de réguler les conflits que les biens engendrent. 

Si nous vivions dans un jardin d'Éden où la terre et les biens étaient infiniment abondants, il n'y aurait aucun besoin d'avoir des règles de propriété. Le concept même de propriété serait vicié car il n'y aurait aucun conflit. Si vous pouvez recréer immédiatement une tondeuse à gazon lorsque je vous prends la vôtre, je ne vous la «volerais» pas. 

Or la nature contient des choses qui sont économiquement rares. Mon utilisation de ces choses exclue que les utilisiez en même temps ou après, et vice-versa. La fonction des droits de propriété est donc d'allouer à chacun un droit exclusif d'utilisation de certaines ressources. Pour que cela soit possible, il faut que ces droits soient à la fois visibles et justes. Clairement, afin que les individus puissent éviter d'utiliser les biens des autres, les frontières des biens en question doivent être objectives, id est certifiables par tout un chacun, intersubjectivement. Elles doivent être visibles. En d'autres termes, «les bonnes barrières font les bons voisins». 

Les droits de propriété doivent aussi être raisonnablement justes parce qu'ils ne pourront servir à éviter les conflits s'ils ne sont pas reconnus comme loyaux par ceux qu'ils affectent. Si les droits de propriété sont alloués de manière déloyale ou par la force, ils sont nuls et non avenus, c'est le règne de la force et la disparition du droit. 

\section{Le versant patrimonial du droit d'auteur}

Le droit d'auteur comprend des droits moraux inaliénables et des droits patrimoniaux. Les droits moraux obligent essentiellement à citer l'auteur d'un travail que l'on réutilise sans laisser croire qu'il soutient votre travail. Ces droits moraux sont des convenances inscrites dans la loi qui ne menacent en rien les droits fondamentaux ou la créativité.

Les droits d'auteur patrimoniaux sont des droits donnés aux auteurs de «travaux originaux» comme les livres, les articles, les films et les programmes d'ordinateur. Ils donnent à leurs possesseurs le droit exclusif de reproduire leurs œuvres, d'en créer des œuvres dérivées, de les mettre en scène ou de les diffuser au public. Ces droits ne protègent que la forme ou l'expression des idées, pas les idées sous-jacentes elles-même.

Ces droits patrimoniaux apparaissent automatiquement dès que l'œuvre est fixée dans un médium tangible d'expression et dure pour la vie entière de l'auteur plus soixante-dix ans.

Le droit d'auteur est un droit sur un objet idéel. La propriété d'une idée ou d'un objet idéel donne de fait à son propriétaire des droits sur les instanciations physiques de l'œuvre ou invention. Dans le cas d'un livre protégé, le détenteur du droit de copie A a un droit sur l'objet idéel. Chaque livre n'en est rien de plus qu'un exemplaire. Le système donne des droits sur l'arrangement même des mots. Par conséquent A a un droit sur toute réalisation concrète de cet arrangement comme une impression papier ou un affichage sur un écran. 

Ainsi, si A écrit un roman, il a un droit de copie sur l'œuvre même. S'il vend une copie physique du roman à B, sous forme d'un livre, alors B ne possède que la copie physique du roman, il ne possède pas le roman lui-même, et n'a pas le droit de faire une copie du roman, même avec sa propre encre et son propre papier.

Les droits patrimoniaux d'un auteur sur ses œuvres intangibles imposent donc des restrictions des droits de propriété des autres sur leurs biens tangibles. Nous montrons que ces restrictions sont injustifiées.

\section{Pourquoi un droit de propriété sur les idées~?}

\subsection{Les idées ne sont pas rares}

De même que la tondeuse magique du jardin d'Éden, les idées ne sont pas rares. Si j'invente une technique pour filer le coton, que vous copiez cette technique ne m'empêchera pas de l'utiliser. J'aurai toujours et ma technique et mon coton. Il n'y a là aucune rareté économique. Aucun conflit ne peut suivre de la rareté de l'objet. Il n'y a donc aucun besoin d'exclusivité. Cela vaut pour les livres aussi par exemple. Si vous copiez un livre que j'ai écrit, j'ai toujours l'original avec moi et votre copie de mon organisation des mots n'exclue pas que je continue à la posséder. C'est pourquoi les écrits ou les inventions ne sont pas rares dans le même sens que des hectares de terre ou des voitures. Pour reprendre les mots célèbres de Thomas Jefferson, lui-même inventeur et membre du premier bureau d'enregistrement des États-Unis. Jefferson était partisan d'une approche utilitariste du système de brevet. C'est pourquoi il est resté tout sa vie sceptique à son propos.

\begin{quotation}
Celui à qui je donne une idée s'instruit sans nuire à mon instruction, comme celui qui allume une lampe m'éclaire sans en être moins éclairé. 
\end{quotation}

Puisque l'utilisation de nos idées par les autres ne nous en prive pas, il n'y a aucun conflit d'usage possible et aucun intérêt à mettre en place des droits de propriété. Pourrait-on cependant justifier autrement que par la rareté des ressources un droit de propriété sur les idées~?

\subsection{Créer ne suffit pas pour s'approprier}
La règle générale pour les biens tangibles est que si le bien n'appartient à personne le premier à s'en emparer en l'occupant ou l'utilisant le possède. Cela permet d'établir une règle claire qui ne soit pas biaisée ou arbitraire. Il y a plusieurs manières de s'emparer d'un objet qui n'appartient à personne. Je peux par exemple cueillir une pomme dans une forêt qui n'appartient à personne pour me l'approprier, ou planter une barrière dans un territoire vierge. 

Cependant, quel serait l'équivalent de cette règle pour les biens idéels~? Le candidat le plus évident est que toute idée appartient à son créateur.

Or parler de création n'est pertinent tant que la création révèle une primo-occupation ou est en soi un acte de primo-occupation. Cependant, la création n'est pas par elle-même une condition suffisante pour primo-posséder un bien. Il n'est pas possible de créer des objets tangibles sans utiliser d'abord des matières premières, et ces matières premières doivent être rares. Soit je les possède, soit je ne les possède pas. Si je ne les possèdais pas et si elles appartenaient à quelqu'un lorsque j'ai créé le nouvel objet, alors je ne possédais pas ce nouvel objet non plus, et je devrai peut-être même dédommager son propriétaire pour avoir utilisé ses biens. Par exemple, si je forge une épée avec un métal qui ne m'appartient pas, le métal ne m'apppartient pas plus après avoir forgé l'épée qu'avant. Seule la primo-appropriation de ce métal est un critère valable pour primo-posséder un objet, non sa création. 

\subsection{Protéger l'originalité c'est empêcher la créativité}
On peut croire qu'à la différence des métaux les idées n'existent pas avant que leurs créateurs leur donnent vie. Elles seraient originales. Mais qui serait assez fou pour se croire génial au point d'avoir des idées complètement originales~? Les auteurs sont toujours perchés sur les épaules de maîtres auxquels ils payent leur tribut.

Sir Arthur Conan Doyle écrivait :
\begin{quotation}
Si chaque auteur qui est redevable de sa production à Poe et reçoit des honoraires pour une histoire devait en donner un dixième pour l’érection d’un monument à la mémoire du maître, la pyramide serait aussi grande celle de Kéops.
\end{quotation}

Justifier les droits d'auteur patrimoniaux par l'originalité des créations, c'est donc les étouffer dans l'œuf. Personne n'ayant jamais une idée absolument originale, personne ne devrait complètement posséder une idée. 

Il serait quand même possible de posséder partiellement une idée~: on possèderait son «~agencement~». Si aucune idée n'est complètement originale, les réorganisations d'anciennes idées pourraient justifier une règle d'appropriation. 

Seulement cette justification est viciée car il est difficile de savoir dans quelle mesure de nouveaux agencements sont vraiment nouveaux. Définir l'originalité n'est pas chose aisée, même pour un juge. Une règle de propriété basée sur cette règle ne respecte donc pas l'impératif de clarté et d'univocité de la loi. Son existence entraînera des conflits qui sinon n'auraient jamais existé. Ce n'est pas ce que l'on attend d'une loi. 

Par conséquent il est masochiste pour un auteur de défendre des droits patrimoniaux fondés sur une exigence d'originalité. Est-ce vraiment le but d'un auteur de passer son temps à calculer ce qu'il doit à chaque personne à laquelle il a emprunté une idée~? Est-ce que les auteurs sont tellement enchantés que cela de payer des redevances à tous ceux qui les menacent de procès pour infraction au droit d'auteur par peur des frais judiciaires, quand bien même les requêtes seraient infondées~? Nous-mêmes, voulons-nous favoriser un monde où seuls les plus riches peuvent créer~?

Pour les auteurs, un système juridique axé sur l'originalité ressemble plus à un cauchemar bureaucratique destiné à empêcher toute création qu'à autre chose. Au contraire ne pas protéger les idées revient à créer un jardin d'Éden pour la créativité. Cependant certains semblent penser le contraire. Ils défendent que le droit d'auteur est utile pour le progrès des arts.

\section{Peut-on justifier cette infraction à mes droits de propriété par son utilité~?}

Le droit d'auteur redistribue les régimes de propriété en prenant à ceux qui possèdent des biens tangibles pour donner aux créateurs et inventeurs. Prima facie, le droit d'auteur est donc une infraction aux droits de propriété. C'est cette redistribution invasive qui doit être justifiée. Le plus souvent ce sont des arguments utilitaristes qui sont utilisés. 

Les utilitaristes soutiennent que le but des droits d'auteur patrimoniaux est l'encouragement à la création et l'innovation. C'est pourquoi les moyens apparemment immoraux que sont les restrictions des droits de propriété des individus sur leurs biens physiques sont justifiés. Mais il y a quelques problèmes fondamentaux qui découlent d'une justification purement utilitariste de la loi ou de droits fondamentaux. 

Tout d'abord, supposons que le bien-être ou l'utilité soient maximisées en adoptant certaines règles légales~: la «taille du gâteau» est agrandie. Est-ce que ces règles sont pour autant justifiées~?

Par exemple, on pourrait dire que le bien-être net de la population totale s'améliore lorsque nous donnons la moitié de la fortune du pour-cent le plus riche aux dix pour-cents les plus pauvres, et que ce vol de la propriété de A pour le donner à B améliore plus le bien-être de B que cela ne diminue celui de A. Cela ne veut pas dire que le vol des biens de A soit justifié. Le but de la loi n'est pas nécessairement de maximiser le bien-être des gens, mais d'être de donner à chacun ce qui lui revient de droit. Même si le droit d'auteur arrivait à augmenter notre bien-être total, cela ne voudrait pas dire que la violation de certains droits fondamentaux comme le droit à la propriété serait pleinement justifiée. Plus caricaturalement, ce n'est pas non plus parce que tuer quelques personnes peut permettre d'augmenter le bien-être général que c'est justifié.

Outre ces problèmes éthiques, une approche purement utilitariste est incohérente. Elle implique nécessairement de faire des comparaisons d'utilité interpersonnelles illégitimes, comme lorsque le «coût» des lois sur le droit d'auteur est soustrait des «bénéfices» de ces lois pour déterminer s'il y a un bénéfice net. Mais parler de valeur ne veut pas dire tout restreindre à des valeurs marchandes, des prix. De fait, cela n'a rien à voir. La libre circulation du savoir a une valeur sociale inestimable qui dépasse largement le cadre marchand. Est-ce que fixer un prix objectif au savoir par la loi est vraiment intellectuellement honnête~?

Finalement, même si nous laissons de côté les problèmes de comparaison interpersonnelle et de justice de la redistribution pour continuer, en utilisant des techniques de mesure utilitaristes standards, il n'est pas du tout clair que le droit d'auteur amène réellement quelque changement que ce soit au bien-être général. Aucune étude économétrique n'est arrivée à montrer des gains de bien-être net de manière convaincante. Il paraît même évident que le droit d'auteur limite la diversité des œuvres disponibles en empêchant la culture du remix. Celle-ci s'épanouit sur Internet car elle méprise les lois sur le droit d'auteur ou de copie.

Ce qui est certain, c'est que les poursuites pour droit d'auteur engendrent des coûts juridiques et administratifs élevés. Hadopi, les chasses internationales aux pirates, l'installation d'infrastructures de censure ou de surveillance et les procès inutiles ont un coût bien réel mais n'ont jamais arrêté le piratage. Or est-ce que ceux qui appellent à l'utilisation de la force pour réglementer l'usage de leurs biens par les autres ne devraient pas pouvoir montrer que cet usage de la force est vraiment utile~?

Il faut rappeler que lorsque nous appelons à l'existence de certaines lois ou de certains droits et que nous analysons leur légitimité, nous analysons la légitimité et l'éthique de l'usage de la force. Se demander si une loi doit exister ou non est se demander s'il est juste d'utiliser la force contre ses contrevenants. Il n'est pas étonnant qu'un simple calcul de maximisation des gains ne suffit pas à répondre à la question. 

En résumé, l'analyse utilitariste est largement inefficace, car il est méthodologiquement douteux de ne s'intéresser qu'à la taille réelle du gâteau, sans compter que déterminer cette taille est un vrai pari et qu'il est empiriquement improbable qu'elle augmente réellement.

\vspace{3em}
Les justifications des droits d'auteur patrimoniaux sont invalides ou pour le moins douteuses. Le droit d'auteur ne peut donc justifier aucune restriction de nos droits de propriétés sur des objets que nous avons achetés. Il ne peut pas non plus justifier le déploiement totalitaire d'outils de surveillance de nos moyens de communication, de nos ordinateurs et in fine de notre vie privée.
