\chapter{L'intérêt du développeur et la satisfaction de l'administration}\label{devadmin}

\begin{quotation}
\emph{Les données libres sont la matière première de la nouvelle révolution industrielle}

Francis Maude, Ministre pour le Bureau du Cabinet en Angleterre
\end{quotation}
\section{La liberté et les données publiques}
Les données publiques ne sont vraiment utiles que lorsqu'elles sont disponibles et référencées dans un format brut et sous une licence libre.

Des données sont brutes lorsque chaque élément de la base de donnée peut être immédiatement isolé comme une case dans un tableur pour être traité informatiquement. Publier des PDFs qui contiennent des tableaux est mieux que rien mais n'est pas très utile car le PDFs est un format d'impression lisible surtout par des humains et non modifiable. Il n'est pas évident voire impossible pour un ordinateur de réorganiser de tels données automatiquement.

Elles sont libres quand elles sont toujours à jour et qu'il est possible de les télécharger et de les réutiliser gratuitement pour tout usage, y compris commercial. Elles ne doivent donc pas être protégées par des brevets et comme le droit d'auteur est par défaut restrictif, elles doivent être accompagnées d'une licence qui spécifie clairement les libertés accordées. Il ne doit pas non plus être nécessaire d'acheter des logiciels exorbitants pour pouvoir les lire et les réutiliser. Le moyen de diffusion le plus efficace est bien sûr Internet.

En outre, comme il est impossible de rassembler toutes les données existantes dans une seule base de données géantes, il est nécessaire de favoriser leur indexation par les moteurs de recherche voire de mettre soi-même à disposition du public un moteur de recherche.

Concrètement, des données publiques sont donc libres lorsque~:

\begin{itemize}
\item tout objet numérique élémentaire a un identifiant unique et peut être accessible et cherchable via Internet via des protocoles et formats standards pour un coût nul ou marginal
\item chaque base de donnée est sous une licence permettant sa réutilisation et sa diffusion par tout un chacun et pour tous les buts imaginables
\end{itemize}

\section{Les bases de données sont les nouvelles matières premières}

Dans tous les domaines, les données brutes ne sont pas un but en soi. Elles ont de la valeur si et seulement si elles permettent de mieux agir, maintenant ou à l'avenir. Plus les décisions que l'on prend grâce à elles sont nombreuses et utiles, plus les données elles-mêmes deviennent importantes. C'est donc leur usage qui leur donne de la valeur. Pour valoriser les données publiques, il faut favoriser la création de produits et de services les réutilisant. 

Beaucoup de bases de données ont déjà été produites ou numérisées par les administrations publiques. Les récréer ex nihilo prend énormément de temps et est un gachis de ressources, pourtant c'est parfois ce qui arrive. Par exemple le projet OpenStreetMap recartographie gratuitement sous des licences libres la France entière parce que les cartes produites par l'Institut national de l’information géographique et forestière (IGN) ne peuvent pas être réutilisées librement par les citoyens français. 

Dans le cas de l'IGN, la justification est que c'est à l'IGN de rentabiliser les cartes que les citoyens ont déjà payé avec leurs impôts et pas aux citoyens. Cependant, est-ce que cette exclusivité de l'utilisation commerciale par les administrations est vraiment la meilleure méthode pour créer de la richesse~? Nous allons tout de suite voir que non.

\section{La valeur économique des données libres}

Les données libres créent de la valeur de deux manières distinctes~:

\begin{itemize}
\item Elles permettent à l'administration d'économiser de l'argent en externalisant certains services.
\item Elles permettent à des entreprises de fournir des services performants et innovants à moindre coût.
\end{itemize}

Dans le secteur privé, avoir des données libres permet bien souvent de ne pas perdre de l'argent dans la négociation de licences coûteuses ou la longue et coûteuse création de bases de données. Cela facilite la création de nouveaux services par tout un chacun et favorise donc une saine concurrence tout en évitant aux citoyens de payer deux fois ou plus la création de ces bases de données. Comme beaucoup de services intéressants pour les citoyens sont avant tout locaux, de nouvelles opportunités d'emploi apparaissent. Cela permet aussi aux investisseurs étrangers de disposer facilement d'indicateurs leur permettant d'évaluer le sérieux des administrations locales ou de sonder l'état d'un potentiel marché plus rapidement.

Prenons par exemple le cas des données météorologiques. Dans son étude \livre{Public Information wants to be free}, James Boyle évalue à 9,5 milliards par an l'investissement public européen dans la météo et à 19 milliards celui américain. Selon lui, l'Europe en tire un bénéfice économique d'environ 68 milliards via l'amélioration des décisions des individus et des entreprises tandis que les États-Unis, qui ont décidé de complètement libérer ces données, en retirent 750 milliards par an, c'est-à-dire 39 fois plus.

Un tel taux de multiplication n'est pas seulement possible parce le secteur privé s'est emparé de ces données pour créer des services qui n'auraient sinon jamais existé, mais aussi parce que des volontaires ou des entreprises aident l'administration à mieux organiser ses propres données. Comme le remarque un rapport de l'institut londonienne de libération des données~: 

\begin{quotation}
Nous avons simplement lancé un appel à contribution sur Twitter pour nous aider à libérer les données publiques londoniennes et une petite armée de volontaires nous a aidé à améliorer la qualité de nos données. S'il y a une leçon à retenir, c'est qu'il faut utiliser l'expertise qui est déjà là, prête à aider.
\end{quotation}

Il faut enfin noter que la libération des données publiques créent des opportunités entrepreunariales même pour ceux qui n'ont pas accès à Internet. En Ouganda, la Grameen Foundation a développé la Question Box, une sorte de téléphone qui permet aux Ougandais d'avoir accès par le réseau mobile à de nombreuses informations médicales, agricoles ou éducatives. C'est une sorte de Google pour ceux qui n'ont pas Internet. En Europe, un tel outil pourrait intéresser ceux qui sont laissés de côté par la révolution du numérique à cause de leur manque de familiarité aux ordinateurs ou de leur mauvaise compréhension de la langue, comme les plus âgés ou certains nouveaux immigrants.  

\section{Améliorer l'administration par la libération des données}

Les pays et les villes ont sans cesse besoin de s'améliorer et de se mettre à jour. Ils doivent savoir quels services sont utiles aux citoyens, où et comment. La libération des données permet de satisfaire aux exigences de renouvellement d'une manière assez indolore, car elle permet de personnaliser à moindre coût les services aux usagers. Par exemple, la ville de Rennes a vu fleurir les applications permettant de faciliter l'utilisation des transports en commun dès qu'elle a libéré les plans et horaires des lignes de bus et métro.

Dans le modèle administratif classique, l'administration délivre le même service impersonnel à tous. Mais si l'administration ne fait plus que fournir des données co-créées avec les citoyens et des entreprises, alors les citoyens peuvent adapter à leurs propres besoins des services qui auraient été autrement standards. Cela favorise la participation citoyenne et la confiance dans l'administration et diminue les coûts pour l'administration comme pour l'usager tout en améliorant le service. Cette réalité est certes en partie cynique mais indéniablement elle est pragmatique.

Cependant il faut distinguer cette sorte de restructuration de l'action publique avec une privatisation classique. Dans les dérégulations habituelles, ce qui se passe est que le monopole de l'État est remplacé par un ou plusieurs monopoles privés. La libération des données publiques, au contraire, donne une chance égale à chacun de créer son propre service. Il existe donc une saine concurrence entre les acteurs et le risque de monopolisation d'activités autrefois publiques par des acteurs privés est nettement moins élevé.

\section{Les derniers freins à l'essor d'une nouvelle ère dans les administrations}

Les principales raisons pour lesquelles les données publiques ne sont pas aussi souvent libérées qu'elles devraient l'être sont les suivantes~:

\begin{itemize}
\item La pure et simple ignorance de l'importance du mouvement en cours. Beaucoup d'administrations n'ont toujours pas terminé de numériser correctement leurs données quand bien même les directives gouvernementales l'exigent.
\item Les craintes légales. Dans l'état actuel du droit d'auteur, le statut juridique par défaut de toute œuvre est «~Tous droits réservés~». Lorsque plusieurs acteurs ont participé ensemble à la création d'une base de donnée, il n'est pas toujours évident d'identifier qui est titulaire de quels droits ni de négocier la libération des données.
\item La crainte de publier des données de mauvaise qualité. Parfois les sources des données sont inconnues ou mal identifiées et les administrations ne savent pas elles-mêmes si celles-ci sont fiables ou non.
\item L'argent. Il existe un investissement initial pour libérer des données qui n'est pas toujours négligeable à cause des points ci-dessus. Qui plus est les administrations qui voient les retombées économiques positives de la libération des données, comme le Trésor Public ou les mairies, sont rarement exactement les mêmes que celles qui ont procédé aux investissements. 
\end{itemize}

