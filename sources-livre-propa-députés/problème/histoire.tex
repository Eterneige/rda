\chapter{Le commentaire de l'historien}

\section{15\ieme siècle~: L'invention de l'imprimerie met fin aux moines-copistes}

\subsection{Le prix du livre}
Commençons avec la Peste Noire dans les années 1350 en Europe. L'Europe fut durement touchée, comme le reste. Elle mit 150 ans à s'en remettre. Les gens fuyaient l'Empire Byzantin et amenaient en
Europe la peste.

Les institutions religieuses furent les plus lentes à récupérer. Elles ne furent pas frappées durement seulement parce que les moines vivaient dans des environnements confinés, mais aussi parce qu'ils
ne se reproduisaient pas eux-mêmes, et que les parents manquaient à présent de main d'œuvre.

À l'époque, si vous vouliez un livre, il fallait le demander à un moine. Qui le recopierait à la main.
Aucune copie n'était parfaite. Les moines corrigeaient des fautes tout en en introduisant d'autres.

Seuls certains livres étaient publiés. Non seulement parce que le prix des matières premières était exorbitant (170 peaux de veau ou 300 de mouton)
mais aussi parce que l'Église n'autorisait pas les livres qui contrariaient sa doctrine.

En 1450, les monastères n'étaient toujours pas repeuplés et copier un livre coûtait très cher à cause des matières premières et du manque de main d'œuvre. 

En 1451, Gutenberg parfit la
presse à imprimer, avec des caractères en métal amovible, une impression à base d'encre grasse et de blocs en bois. En même temps, un nouveau type de papier bon marché à fabriquer et résistant est
copié des Chinois. Dans les décennies qui suivirent, les moines-scribes devinrent obsolètes.

\emph{L'invention de la presse a révolutionné la société en permettant de transmettre rapidement, facilement et efficacement l'information.}
\subsection{L'Église perd le contrôle de l'information}
L'Église catholique, qui contrôlait auparavant totalement la diffusion de l'information et avait fait un marché de niche de cette rareté de l'information, éructa. Elle ne contrôlait plus ni
l'information ni les esprits et fit pression sur les rois pour bannir cette technologie.

De nombreux arguments furent inventés à l'occasion pour rétablir l'ancien régime. L'un d'entre eux était \begin{quotation}
     «~Comment paierons-nous les moines-copistes~?~»                                                                                                    
                                                                                                        \end{quotation}

Finalement, l'Église catholique ne réussit pas à empêcher la propagation de l'imprimerie,  laissant la voie libre à la Renaissance et au protestantisme, mais beaucoup de sang coulat pour empêcher la
circulation rapide et efficace des idées, de la culture et de la connaissance.

L'aboutissement des efforts de l'Église peut être marqué par la parution d'une loi en France le 13 janvier 1535 qui ordonnait la fermeture de toutes les librairies et condamnait à mort par pendaison
quiconque utilisait une imprimerie.

Cette loi fut largement inefficace. Les frontières du pays fourmillèrent de librairies et la littérature pirate se répandit en France via des contrebandiers ravitaillant les gens normaux en quête de
nouvelles choses à lire.

\section{17\ieme siècle~: Marie la Sanguinaire censure grâce au droit de copie}
\subsection{Une bâtarde répudiée}
Le 23 mai 1533, la fille de 17 ans qui serait Marie 1\ieme d'Angleterre fut officiellement déclarée bâtarde par un archevêque protestant. Sa mère, Catherine, qui était catholique et une protégée du
Pape,
avait
été mise à la porte par son père Henri, qui s'était converti au protestantisme pour se débarrasser d'elle. Marie tenterait de redresser cette injustice toute sa vie.

Le roi Henri VIII voulait un fils pour lui léguer le trône d'Angleterre mais son mariage n'avait pas réussi. Sa femme Catherine d'Aragon ne lui avait donné qu'une fille. Pire, le Pape ne voulait
pas le laisser divorcer.

La solution d'Henri fut assez radicale mais innovante. Il convertit toute l'Angleterre au protestantisme et fonda l'Église d'Angleterre afin de renier le Pape. Il fit déclarer son mariage nul le 23
mai 1533 puis se maria à plusieurs femmes par la suite. Il eut une deuxième fille avec sa seconde femme puis enfin un garçon avec la troisième. La demi-sœur et le demi-frère de Marie étaient
protestants. 

Édouard succéda à Henri VIII en 1547, à neuf ans. Il mourut avant d'atteindre l'âge adulte. Marie était deuxième dans l'ordre de succession, même si elle était une bâtarde. Elle devint donc reine en
1553.

Elle n'avait pas parlé à son père pendant des années. Elle voulut rendre l'Angleterre au catholicisme. Elle persécuta sans relâche les protestants en en exécutant publiquement des centaines et
s'acquit le surnom de Marie la Sanguinaire\footnote{Bloody Mary en anglais}.
\subsection{Rétablissement du catholicisme}
Marie Tudor partageait les préoccupations de l'Église catholique à propos de la presse. Que le grand public puisse faire rapidement circuler l'information était dangereux pour le rétablissement du
catholicisme à cause de l'existence de textes hérétiques. Vu que la France avait misérablement échoué à bannir l'imprimerie, même en menaçant les contrevenants de pendaison, elle chercha une autre
solution. Une qui serait aussi autant bénéfique pour l'industrie de l'imprimerie que pour elle.

Elle accorda un monopole à la corporation des imprimeurs de Londres en échange du contrôle des ouvrages imprimés. Ce fut un accord très lucratif pour la corporation, qui travailla dur pour maintenir
le monopole en place. 

Cette coopération des pouvoirs corporatistes et gouvernementaux fut très efficace pour réduire la liberté d'expression et étouffer les dissensions politico-religieuses. Aussi
longtemps que rien de politiquement dérangeant ne circulait, tous les divertissements étaient autorisés. L'accord était gagnant-gagnant.

Le monopole fut accordé le 4 mai 1557 à la Compagnie londonienne des Libraires. Il fut appelé copyright (droit de copie en français).

Les Libraires travaillait comme un bureau de censure privée en brûlant les livres interdits, en détruisant les presses clandestines et en empêchant la diffusion de tous les textes politiquement
incorrects. Peu d'affaires remontaient à la Reine. Après quelques consultations, les censeurs savaient ce qu'il fallait censurer.

Marie Tudor mourut en 1558. Sa sœur protestante Élisabeth lui succéda. La tentative de restauration du catholicisme par Marie échoua mais l'invention du copyright lui survécut. 

\subsection{Le droit de copie se maintient grâce aux éditeurs}
À la mort de Marie, ni la Couronne ni la corporation ne désiraient abolir le copyright. L'accord dura 138 ans sans interruption. Élisabeth se servit de cet accord pour empêcher la propagation des
matériaux catholiques.

Pendant le 17\ieme siècle, le Parlement essaya de prendre progressivement le contrôle de la censure des mains de la Couronne. 

En 1641, il abolit dans la foulée du vote de
l'Habeas Corpus l'infâme Chambre étoilée, véritable tribunal d'inquisition qui jugeait les
cas
d'infractions
à la censure. Cela rendit ces infractions inoffensives, de même que la traversée en dehors des clous est tout à fait tolérée concrètement parlant. Dès lors, la créativité en Grande-Bretagne fleurit.

Malheureusement, ce n'était pas du tout ce que le Parlement avait prévu.

En 1643, le monopole du copyright fut restauré par le Licensing Order of 1643 avec une subtilité supplémentaire. Tous
les auteurs, imprimeurs et éditeurs devaient s'enregistrer auprès des Libraires pour demander une licence
d'exercice et de publication pour chaque ouvrage. Les Libraires avaient de plus le droit de détruire les presses et les livres non autorisés, et d'infliger de sévères sanctions aux contrevenants.

Accélérons un peu. Il y eut la Glorieuse Révolution en 1688, et la composition du Parlement changea radicalement. En faisaient désormais majoritairement partie des gens qui avaient été du mauvais côté de la
censure et n'avaient aucune envie de la voir continuer. Le monopole des Libraires fut abrogé en 1695.

À partir de 1695, il n'y eut plus de copyright. La créativité fleurit de nouveau, et les historiens soutiennent que les documents publiés dans ce vent de liberté menèrent à la fondation des
États-unis.

Malheureusement, les Libraires étaient peu ravis d'avoir perdu leur travail et leur monopole lucratif. Ils rassemblèrent leurs familles devant le Parlement en demandant la restauration de leur
monopole.

Le Parlement, qui venait juste d'abolir la censure, n'avait aucune envie de la restaurer. Les Libraires suggérèrent donc que les auteurs devraient «~posséder~» leurs travaux. Ainsi, ils
faisaient d'une pierre trois coups. 

\begin{enumerate}
\item 
Le Parlement s'assurait qu'il n'y aurait pas de censure centralisée. \item
Les éditeurs gardaient le monopole de leurs impressions, et les
auteurs ne pouvaient publier sans eux. \item
Le monopole entrait dans la Common Law, ce qui lui offrait des protections plus fortes que s'il faisait partie des lois purement
jurisprudentielles.\end{enumerate}

Le lobby des éditeurs obtint gain de cause et le nouveau monopole du droit de copie parut en 1709 pour prendre effet au début 1710. Voilà la première grande victoire du lobby de l'édition.

Dès lors, les Libraires continuèrent à détruire et brûler les travaux des autres presses pendant longtemps même si légalement ils n'en avaient plus le droit. Cela dura jusqu'au jugement Entick vs.
Carrington en 1765, qui concernait un de ces raids contre les auteurs «~non licenciés~» (c'est-à-dire non désirés).
Les juges décidèrent que la poursuite des presses et auteurs illégaux était du
ressort de l'État, non des Libraires.

Ce qu'il faut retenir ici est que le copyright n'a jamais été inventé pour profiter aux auteurs. Ce sont des justifications a posteriori. Le copyright bénéficiait avant tout aux censeurs, aux éditeurs
et aux imprimeurs. Personne ne défendit que le copyright était essentiel à
l'écriture.

Rien de nouveau sous le soleil donc. Les toutes premières fondations de la démocratie en Europe de l'Ouest devaient déjà se battre contre les monopoles de l'édition.

\section{19\ieme siècle~: De Jefferson à Hugo}
\subsection{Le copyright américain sert le progrès de la science et des arts utiles}
\begin{quotation}
 Lire sans payer~? C'est du vol !
\end{quotation}

À la fondation des États-Unis, le concept de monopole sur les idées fut apporté dans le nouveau monde, et intensément débattu. Thomas Jefferson s'opposa
fermement à ce monstre~:
\begin{quotation}
Si la nature a rendu une chose moins susceptible que les autres de propriété exclusive, c'est l'action de penser appelée idée, qu'un individu peut posséder pour autant qu'il la garde pour lui-même,
mais qui ne lui appartient plus dès qu'elle est divulguée aux autres, sans que les autres puissent s'en débarrasser. Ce trait tout à fait particulier des idées fait que l'on ne les possède pas moins
si les autres les possèdent, parce que tous les possèdent entièrement. Instruire quelqu'un ne diminue pas mon instruction. Il est illuminé sans me faire pour autant de l'ombre. Que les idées puissent
circuler librement d'un point à l'autre du globe pour l'instruction morale mutuelle des hommes et pour l'amélioration de leur condition semble avoir été volontairement conçu par la nature quand
elle nous a rendu … incapable de les enfermer ou de nous les approprier.
\end{quotation}

Un compromis fut cependant conclu. Les États-Unis ont été les premiers à considérer qu'il y avait une raison à l'existence du copyright et des brevets. Cette raison est très lapidaire dans la loi
américaine~:

\begin{quotation}
 …pour promouvoir le progrès des sciences et des arts utiles…

Article I, Section 8, Clause 8 de la Constitution américaine, dite clause du copyright
\end{quotation}

Il doit être noté que l'intention du monopole n'est pas qu'une profession gagne sa vie, comme celle des distributeurs ou des imprimeurs. Elle est exemplaire dans sa clarté~: la seule justification du
monopole est \emph{la maximisation de la culture et de la connaissance disponibles dans la société}.

Le copyright américain découle donc d'un équilibre entre l'accès du public à la culture et l'intérêt de ce même public à la création de la culture. C'est essentiel. Le public
est la seule aune de l'intérêt du copyright. 

Les détenteurs de monopole, bien que bénéficiant eux aussi du copyright, ne sont pas des intéressés légitimes, et n'ont pas leur mot à dire dans
l'interprétation de la loi, de même qu'un régiment obéit mais ne dicte pas la politique de sécurité nationale. 

Ce point doit être souligné. Beaucoup croient que la Constitution des États-Unis justifie l'existence d'un monopole du droit de copie
pour que les artistes puissent gagner leur vie. Littéralement
parlant, là n'est pas la question ou l'intérêt du copyright.

\subsection{Invention des bibliothèques publiques en Angleterre}
Pendant ce temps au Royaume-Uni, les livres devenaient relativement chers à cause du copyright. Seuls les riches pouvaient collectionner les livres, et certains commencèrent à en louer à d'autres.

Les éditeurs devinrent furieux en s'en rendant compte et firent pression sur le Parlement pour rendre illégale la lecture d'un livre sans en avoir payé son propre exemplaire. Ils essayèrent de rendre
illégales les bibliothèques publiques avant même que celles-ci soient inventées.

Mais le Parlement ne suivit pas leur avis en voyant que la diffusion des livres était bénéfique pour le public. Pour le Parlement, le vrai problème était surtout que les riches pouvaient maintenant
contrôler ce que les pauvres lisaient et qui lisait. Il décida donc de créer des bibliothèques accessibles au public, à tous les publics.

Les défenseurs du copyright fulminèrent en entendant l'idée. \begin{quotation}
  «~Plus personne ne pourra vivre de ses écrits ! Plus personne n'écrira de livre ! Plus aucun livre ne se vendra !~»                                                            
                                                             \end{quotation}
                                                             
Cependant, le Parlement anglais de l'époque fut bien plus avisé que ne le sont les parlementaires européens d'aujourd'hui et prit le pétage de plomb des monopolistes du droit de copie pour ce qu'il
était. En 1849, une loi instituant l'existence de bibliothèques publiques fut passée et la première bibliothèque ouvrit en 1850.

Bien sûr, depuis, aucun livre n'a plus été écrit. Ou alors, les tirades des monopolistes du copyright étaient injustifiées et tout aussi fausses que leurs tirades actuelles.

On peut même noter que dans certains pays européens, depuis le début du 20\ieme siècle, certaines rentes versées aux auteurs et traducteurs sont indexées sur leur succès en bibliothèque.

\subsection{L'industrialisation allemande dépasse l'anglaise}
À la même période, le droit d'auteur ou le droit de copie n'existaient pas en Allemagne. Plusieurs historiens soutiennent
que c'est cette différence qui a expliqué que
l'Allemagne a si rapidement
rattrapé et dépassé industriellement le Royaume Uni. La connaissance se répandait vite et bien. D'une certaine manière, le succès allemand prouvait que les vues de Jefferson ou du Parlement anglais
étaient correctes~: le public a intérêt à la diffusion rapide de la culture.

\subsection{Les droits moraux se répandent sur le continent}
À la fin du 19\ieme, siècle, les renforcements du droit de copie exigés par les éditeurs avaient enterré les possibilités pour les créateurs de gagner leur vie grâce
à leur travail. Basiquement, l'argent allait aux distributeurs et aux éditeurs, et les créateurs vivotaient sans le sou (tout comme aujourd'hui). 

En France, une personne nommée Victor Hugo essaya d'équilibrer les règles du jeu en internationalisant une tradition française nommée droit d'auteur et en l'incluant dans le droit de
copie.

Il essaya aussi d'internationaliser le droit de copie même si l'époque était plutôt aux lois de libéralisation des marchés à travers l'Europe. À l'époque, les droits
restaient nationaux. Un écrivain
français vendait son monopole à un éditeur français mais les éditeurs allemands et anglais faisaient ce qu'ils voulaient de son œuvre.

Curieusement, les monopoles des brevets et du droit de copie furent oubliés dans le mouvement de libéralisation des marchés pour empêcher la «~concurrence déloyale~», réminiscence de l'époque
où les corporations dictaient les prix des produits et les salaires. Aujourd'hui encore, les poursuites judiciaires à coup de brevets rappellent l'époque où les corporations
vandalisaient les commerces non autorisés.

L'idée de Victor Hugo était de compenser les pouvoirs exorbitants des éditeurs en accordant des droits inaliénables aux créateurs, avec l'effet de bord de continuer à priver le public de ses droits.

Il y réussit partiellement, même s'il ne vit pas son succès de son vivant, avec la signature de la convention de Berne en 1886. Cette convention dit que les pays doivent mutuellement respecter les
droits d'auteur et de copie établis dans les autres pays et instaure une autorité de surveillance, la BIRPI (Bureaux internationaux réunis pour la protection de la propriété intellectuelle). Cette
agence est devenue l'Organisation mondiale de la propriété intellectuelle ou OMPI et la convention de Berne a déjà été
plusieurs
fois réécrite depuis 1886.

On peut noter que, dans toutes ses péripéties, les principaux perdants ont toujours été le public et les
créateurs, le plus souvent mal organisés et représentés lors de l'élaboration des lois, à
l'inverse des éditeurs et distributeurs.

\section{Les quatre droits essentiels d'un auteur}
\begin{enumerate}
\item 
\textbf{Le monopole commercial de la reproduction d'une œuvre}. C'est le monopole originellement accordé à la corporation des imprimeurs londoniens pour organiser la censure.
\item
\textbf{Le monopole commercial de la représentation vivante d'une œuvre}. Si quelqu'un met en scène une œuvre pour en tirer de l'argent, le détenteur du monopole a le droit de demander une
rétribution.
\item
\textbf{Le droit moral de la reconnaissance comme auteur}, afin de protéger contre la contrefaçon ou le plagiat.
\item
\textbf{Le droit moral d'interdire une réinterprétation de son œuvre}. Si un artiste pense qu'une mise en scène détruit son œuvre ou son image, il peut refuser sa diffusion.
\end{enumerate}

Les droits moraux, issus du droit d'auteur français défendu par Victor Hugo à travers l'Europe, sont très différents des monopoles commerciaux en ce qu'ils ne sont pas transférables. Ils ne sont pas
du tout justifiés par les raisons données par le Parlement britannique en 1709. 

Il est aussi notable que ces quatre aspects sont souvent délibérément confondus pour défendre le monopole le plus dommageable à la société, le monopole commercial sur la reproduction. 

Vous
entendrez souvent des gens de l'industrie du droit de copie défendre leur monopole en demandant «~Voudriez-vous que quelqu'un prenne votre travail et se l'attribue~?~». Malheureusement, ce troisième
point peu controversé, le droit moral à l'attribution et au crédit, n'a rien à voir avec les monopoles d'exploitation commerciale.

Les États-Unis ne voulurent pas reconnaître les droits moraux et restèrent donc en dehors de la convention de Berne jusqu'à ce qu'elle se révèle utile pour contrer Toyota un siècle plus tard. Nous y
reviendrons.

\section{Années 1930~: l'industrie musicale entre dans la danse}
Pendant la plus grande partie du vingtième siècle, les conflits sur les droits d'auteur et de copie se focalisèrent sur la musique, non sur le livre. La bataille fit rage entre les musiciens et leurs
labels. Au début du vingtième siècle, les musiciens étaient
regardés à juste titre comme les plus légitimes dans ce conflit. Cependant, l'industrie de la musique réussit à mettre sous sa coupe l'essentiel de la production musicale. Cela commença en Italie. 

Dans les années 1930, beaucoup de musiciens perdirent leur emploi à cause de la crise économique et de l'arrivée du cinéma parlant.

Dans cet environnement, deux initiatives furent prises parallèlement. 

De leur côté, les syndicats de musiciens essayèrent de garantir des revenus stables à leurs chômeurs tout en réglementant l'arrivée de la «~musique mécanique~», c'est-à-dire de celle qui
ne demande
pas de musiciens en chair et en os pour être reproduite. La question fut soulevée dans une assemblée de l'Organisation internationale des travailleurs de l'époque.

En même temps, l'industrie musicale essaya aussi de contrôler cette «~reproduction technique~» de la musique en contrôlant la radio et les musiciens. Cependant, le monde politique et économique de
l'époque les regardait comme des intermédiaires vers les musiciens. En faire plus était essayer de courir la faillite vu qu'ils n'étaient pas plus importants que quiconque, sauf pour Mussolini.

\subsection{L'Italie fasciste et la naissance de l'IFPI}

En 1933, l'industrie du phonographe fut invitée à Rome par la  Confederazione Generale Fascista dell’Industria Italiana. Lors d'une conférence entre le 10 et le 14 novembre, la fédération
internationale de l'industrie phonographique fut fondée. Elle serait plus tard connue sous le nom d'IFPI. Il fut conclu que l'IFPI essayerait d'obtenir les mêmes droits pour les producteurs que pour
les musiciens et les artistes qui vendaient habituellement leurs droits aux éditeurs et aux diffuseurs.

L'IFPI continua à se rencontrer dans un pays qui favorisait son agenda corporatiste, c'est-à-dire l'Italie, comme l'année suivante à Stresa en 1935. L'année 1935 et les suivantes furent chaotiques
en Europe mais l'Italie mit en place les droits corporatistes que l'IFPI réclamait dès 1937.

Essayer de négocier un monopole attaché à la convention internationale de Berne et du même style que le droit de copie possédé normalement par les auteurs mais pour les producteurs était trop tentant
pour l'IFPI. 

Vu que l'Italie de 1950 leur était politiquement hostile, elle se réunit dans le Portugal para-fasciste d'alors. La conférence mit sur pied un projet qui donnerait aux producteurs des droits
identiques à ceux du droit de copie, appelés «~droits voisins~», sur les reproductions techniques ou vivantes des œuvres. Ce monopole serait à peu près
identique à celui du monopole
commercial du droit de copie.

Les droits voisins furent ratifiés par l'Organisation mondiale de la propriété intellectuelle d'alors en 1961 dans la convention de Rome, qui donna aux producteurs et aux artistes-interprètes des droits beaucoup
plus importants qu'avant.

Depuis 1961 l'IFPI défend corps et âme le droit de copie même si les producteurs eux-même ne jouissent pas directement du monopole qu'il confère. Ils jouissent seulement des dits «~droits voisins~».

\subsection{Lumières et puissance financière}
L'industrie du divertissement confond tous ces monopoles sur la reproduction des œuvres à dessein. Elle défend «~son droit de copie~» qu'elle n'a en fait pas. Elle parle avec
nostalgie d'une tradition qui remonterait à l'époque des Lumières et aurait été créée par la sagesse des anciens de cette époque [insérez les chatons ici] . Ces monopoles n'ont été créés qu'en Europe
en 1961 et ont toujours été polémiques. Ils n'ont jamais été produits par la sagesse des Lumières.

Si les musiciens avaient été les seuls à réussir à faire reconnaître leurs droits sur les copies enregistrées de leurs performances, les producteurs seraient restés des bureaux
d'intermédiaires comme ils
l'avaient toujours été. Cela aurait été le cas si les gouvernements fascistes ne s'étaient pas mêlés de l'affaire entre temps.

\section{1980~: deuxième piratage des droits de copie et d'auteur, par Pfizer}
La dernière partie du vingtième siècle est marquée par deux faits. 

D'un côté les labels s'effrayent de la possibilité qui s'ouvre au grand public de copier pour des usages non-commerciaux leurs
œuvres même si les droits d'auteur et de copie ont toujours visé les usages commerciaux. Au point de défendre que la copie privée devrait être illégale ou de remettre en cause le droit à la vie
privée pour surveiller nos usages. 

De l'autre côté les monopoles conférés par les droits de copie continuent à s'étendre et à forger le monde. 

Vu que les atteintes à la vie privée sont à la une des
journaux nous nous intéressons à l'expansion des droits accordés par le législateur.
\subsection{Les invasions barbares}


Toyota a touché le cœur de l'âme américaine dans les années 1970 en envahissant le marché américain. Les voitures américaines -~les voitures~! des voitures américaines~!~- n'étaient pas assez
performantes pour les 
Américains eux-mêmes. Tout le monde achetait Toyota. L'apocalypse approchait. Les États-Unis étaient finis, impuissants face à la compétitivité asiatique.

Lorsqu'il est devenu clair que les États-Unis n'étaient plus industriellement concurrentiels et ne pourraient continuer leur domination économique par ce biais, de nombreuses commissions furent formées
pour
répondre à cette question cruciale~: Comment maintenir la domination économique américaine tout en ne produisant rien de concurrentiellement valable~?

\subsection{La réponse pharmaceutique}
La réponse vint d'un endroit inattendu~: Pfizer.

Le président de Pfizer, Edmun Pratt, écrivit un édito fulminant dans le \livre{New York Times} du 9 juin 1982~: «~Voler les esprits~». 

Ce qui le faisait enrager était le vol de propriété
intellectuelle
par le tiers-monde. Il voulait dire par là qu'il lui était inconcevable que les pays du tiers-monde utilisent leurs connaissances locales et leur temps pour produire avec leurs matières premières dans
leurs usines des médicaments pour soigner leurs peuples, qui mourraient fréquemment de maladies bénignes chez nous mais fatales dans les conditions de vie locales. 

Les politiciens y virent une réponse
à leurs interrogations sur l'avenir des États-Unis, et Pratt alla diriger une nouvelle commission. Cette commission fut la magique ACTN~: Commission de conseil sur les négociations commerciales (Advisory Committee on Trade Negotiations).

Ce que l'ACTN recommanda, en suivant la direction de Pfizer, était si provocant que personne n'était tout à fait sûr de la nécessité de son application. Les États-Unis devraient mener de front les
négociations commerciales et la politique extérieure afin de stigmatiser les pays qui n'obéissaient pas aux impératifs du «~libre échange~» à l'américaine, par exemple en étant inscrit sur la liste
de surveillance «~Special 301~». Cette liste serait une liste de pays qui ne respecteraient pas assez les lois sur le copyright. La plupart des pays du monde, Canada inclus, seraient trop tolérants aux
infractions au copyright pour les Américains.

La solution à l'absence de production de valeur ajoutée fut donc la redéfinition des termes «~absence~», «~production~» et «~valeur~» via la diplomatie internationale, au forceps. 

Ça a marché.
Les représentants du commerce extérieur des États-Unis, en faisant diplomatiquement et économiquement pression sur les gouvernements étrangers pour mettre en place des législations favorisant les
intérêts industriels américains et en établissant tout une palanquée d'accords bilatéraux et multilatéraux de «~libre échange~» aux mêmes fins, firent mieux respecter les intérêts américains dans le
monde entier.

De cette manière, les États-Unis pouvaient créer de la valeur en louant des plans de construction et en récupérant des produits finis selon ces plans. C'était à présent un marché équitable selon les
accords de «~libre échange~» qui redéfinissaient artificiellement ce que valeur veut dire.

Les industries américaines des monopoles par droit de copie ou de brevet étaient derrière ce putsch. Ils allèrent tous ensemble à l'Organisation mondiale de la propriété intellectuelle pour y répéter
l'abordage réussi par l'industrie de la musique en 1961, pour y recueillir légitimité et hospitalité sous l'égide d'une nouvelle convention appelée «~Berne Plus~».

\subsection{Naissance de l'OMC}
À un moment, il devint nécessaire aux États-Unis de rejoindre la convention de Berne, puisque l'OMPI surveille l'application de la convention.

Cependant, l'OMPI comprit le plan et les mit plus ou moins dehors. L'OMPI n'a pas été créée pour donner à un pays en particulier ce type d'avantage sur les autres. Les autres membres furent outrés par
la tentative assumée de détourner les lois sur les brevets et le droit d'auteur.

Une autre plate-forme d'accord fut donc créée. Les consortiums américains approchèrent le GATT (acronyme anglais pour Accord général sur les tarifs douaniers et le commerce) pour y gagner en
influence. 

Une procédure fut lancée afin d'obliger plus ou moins directement les membres à adhérer à un nouvel accord qui enterrerait la convention de Berne et renforcerait l'industrie américaine via
la redéfinition des termes «~production~», «~valeur~» et «~absence~». 

Cet accord fut appelé TRIPS en anglais ou ADPIC en français~: Aspects des droits de propriété intellectuelle qui
touchent au commerce. Lors de la ratification de l'ADPIC, le GATT fut renommé «~OMC~»~: Organisation mondiale du commerce. Les 52 pays du GATT qui refusèrent l'OMC furent bientôt dans une position
intenable et seul un pays sur les 129 pays signataires du GATT refusa de joindre l'OMC.  

L'ADPIC a été considérablement critiqué parce qu'il est construit pour enrichir les riches au détriment des pauvres, pour que ceux-ci payent de leur santé ou de leur vie quand ils n'ont pas d'argent.
Il interdit aux pays pauvres de produire leurs propres médicaments pour leur population même si celle-ci en meurt. Quelques aménagements marginaux ont bien été concédés avec le temps mais ils sont rares.

Pour donner une idée de l'importance de l'industrie américaine du droit de copie dans l'affaire, regardons ce qui s'est passé quand la Russie a demandé à entrer pour des raisons incompréhensibles dans l'OMC. Les États-Unis exigèrent que la Russie ferme AllofMP3, un site qui vendait des MP3s et était classé comme radio en Russie en payant une licence pour.

Analysons un peu la situation. Les États-Unis et la Russie siègent à la même table. Ce sont d'anciens ennemis qui gardaient leurs armes nucléaires braquées sur l'autre 24h/24, 7j/7, par tous les
temps. Les États-Unis auraient pu demander n'importe quoi au vaincu et l'auraient obtenu. 

Qu'ont-ils demandé~?

La fermeture d'un bête magasin de vente de musique en ligne.
