\chapter{La déception du conservateur}\label{conserv}
\section{Le domaine public est en danger en France}
Les marchés, la démocratie, la science, la liberté d'expression et l'art dépendent bien plus des œuvres et productions constituant le domaine public librement accessible que des productions informationnelles couvertes par des droits de propriété intellectuelle. Le domaine public n'est pas un résidu qui se déposerait lorsque tout ce qui a de la valeur aurait été saisi par les lois sur la propriété intellectuelle. Le domaine public est la carrière dont nous extrayons les pierres avec lesquelles nous bâtissons notre culture. En fait, il constitue la majorité de notre culture.

En France, ce domaine public est en danger. Il est menacé par les institutions culturelles qui devraient au contraire le protéger. Entendons-nous bien, je ne suis pas en train de dire que les bibliothèques, musées ou archives n’assurent pas leur rôle de conservation patrimoniale des oeuvres physiques qu’elles conservent. Le problème est qu'elles sont une majorité écrasante détourner le droit d'auteur pour porter atteinte à la liberté de réutilisation qui devrait être le pendant logique du domaine public.

Le problème est connu~: allez sur les sites des musées, des bibliothèques et des archives. Vous y trouverez de très nombreuses oeuvres du domaine public numérisées et offertes à la consultation du public. Mais dans une écrasante majorité des cas, les images seront accompagnées d’une mention restrictive, qui restreindra les usages d’une manière ou d’une autre, quand un brutal «~Copyright : tous droits réservés~» ne sera pas purement et simplement appliqué.

Une telle revendication empêche toutes les formes de réutilisation, y compris les plus légitimes~: 

\begin{itemize}
\item Elle bloque les usages pédagogiques et de recherche.
\item Elle interdit aux simples internautes de les reprendre pour illustrer leurs blogues et leurs sites. 
\item Elle ne permet pas d’aller enrichir les articles de Wikipédia et d’autres sites collaboratifs. 
\item Elle bloque les réutilisations commerciales.
\end{itemize}

Dans les musées, au lieu de ce droit d'auteur brutal, on use parfois d’un autre stratagème, en reconnaissant un droit d’auteur aux photographes qui prennent des clichés de tableaux. Le musée se fait ensuite céder ce droit d’auteur par contrat, ce qui lui permet d’appliquer à la fois le sien et celui du photographe.

D’autres institutions ont récemment développé une tactique encore plus subtile. Elles considèrent que la numérisation produit des données publiques (les oeuvres deviennent des séries de 0 et de 1, qui seraient constitutives d’informations publiques au sens de la loi du 17 juillet 1978). C’est le cas par exemple parfois à la Réunion des musées nationaux, à la Bibliothèque nationale de France pour Gallica, aux Archives nationales pour Archim et dans bon nombre de services d’archives départementales. Elles peuvent alors conditionner certaines formes de réutilisation à la passation d’une licence et au paiement d’une redevance. Recouvert par cette couche de droit des données publiques, le domaine public disparaît. Pourtant cette lecture de la loi de 1978 est juridiquement contestable, même si aucun tribunal n’a encore statué sur la question.

Les manifestations de ce saccage juridique peuvent prendre d’autres détours encore, comme lorsqu’au Musée d’Orsay, on interdit avec obstination aux visiteurs de prendre des photographies (même sans flash), y compris lorsque les oeuvres appartiennent au domaine public.

\section{La marchandisation du domaine public par les institutions publiques}

Aujourd’hui, dans le but de marchandiser le domaine public, on ne met plus en ligne les images numérisées pour mieux les vendre sous forme de bases de données, en partenariat avec des entreprises privées qui assureront la numérisation et se rémunèreront sur le produit des ventes.

Cette formule est appliquée en ce moment à la Bibliothèque nationale de France dans le cadre de plusieurs appels à partenariats mettant à contribution les Investissements d’avenirs du Grand Emprunt national.

Un prestataire privé numérise les ouvrages en recevant un soutien financier via le Grand Emprunt. Seule une partie minime du corpus pourra être accessible librement et gratuitement sur Gallica, la bibliothèque numérique de la Bibliothèque nationale de France (5\% pour les livres). Le reste sera transformé en une base de données, non accessible en ligne, qui sera commercialisée via une filiale de la Bibliothèque nationale de France.

Cet embargo sur la mise en ligne sera maintenu durant une durée variable selon les corpus (7 ans pour les livres et la presse, 10 ans pour les fonds sonores) par le biais d’une exclusivité reconnue au partenaire commercial. À l’issue seulement de ce délai, les contenus numérisés pourront rejoindre Gallica en ligne.

Cette formule pourrait paraître constituer un compromis équilibré en permettant la numérisation de documents patrimoniaux et en répartissant les coûts importants entre le public et le privé. Mais ce n’est en réalité pas du tout le cas et ces montages violent des recommandations importantes faites au niveau international, si ce n’est la loi française~!

\section{Mépris des recommandations européennes}

Un Comité des Sages réuni par la Commission européenne avait publié en janvier 2011 une série de recommandations concernant les partenariats public-privé en matière de numérisation du patrimoine. Les sages européens ont recommandé que la durée des exclusivités accordées aux partenaires privés n’excède pas les 7 ans~:

\begin{quotation}
La période d’exclusivité ou d’usage préférentiel des œuvres numérisées dans le cadre d’un partenariat public-privé ne doit pas dépasser une durée de 7 ans. Une telle durée peut, en effet, être considérée comme pertinente pour, d’une part, générer  suffisamment d’incitation à la numérisation pour le partenaire privé et, d’autre part, garantir un contrôle suffisant des institutions culturelles sur les œuvres numérisées.
\end{quotation}

Il s’agit d’une durée maximale de 7 ans pour une exclusivité commerciale seulement, mais pas du tout une exclusivité sur la mise en ligne elle-même, car le texte du rapport indique formellement plus haut que les oeuvres du domaine public numérisées doivent être mises en ligne~:

\begin{quotation}
Afin de protéger les intérêts des institutions publiques qui concluraient un partenariat avec une entreprise privée, le Comité des  sages considère que certaines conditions doivent a minima être respectées~:
\begin{itemize}
\item Le contenu de l’accord entre une institution culturelle publique et son partenaire privé doit nécessairement être rendu public
\item \textbf{Les œuvres du domaine public ayant fait l’objet d’une numérisation dans le cadre de ce partenariat doivent être accessibles gratuitement dans tous les 
Etats membres de l’UE}
\item Le partenaire privé doit fournir à l’institution culturelle des fichiers numériques de qualité identique à ceux qu’il utilise pour son propre usage.
\end{itemize}
\end{quotation}

En 2009 la révélation d’un projet de numérisation à l’étude entre Google et la Bibliothèque nationale de France avait fait scandale. A l’époque, Google voulait numériser à ses frais les fonds en contrepartie de l’imposition d’une exclusivité commerciale de 25 ans, identique à celle qu’il avait imposée à la Bibliothèque municipale de Lyon.

Les accords conclus par Google ne comportent aucune exclusivité concernant la mise en ligne elle-même. Le but de Google était bien de mettre en ligne les oeuvres du domaine public qu’il numérisent, sur Google Books et à présent sur Google Play, où l’on trouve d’ailleurs déjà des livres de la Bibliothèque municipale de Lyon en téléchargement gratuit. Même si Google impose des restrictions, les bibliothèques partenaires sont aussi autorisées à mettre en ligne les ouvrages sur leur propre site.

La manière dont la Bibliothèque nationale de France va encapsuler des oeuvres du domaine public numérisées dans des bases de données coupées du web pour mieux les vendre est donc à tout prendre bien pire en terme d’atteinte au domaine public que ce qui était envisagée avec Google. Or la mobilisation du Grand Emprunt devait normalement permettre de trouver une solution plus satisfaisante que celle proposée par Google selon les préconisations du rapport Tessier. C’est exactement l’inverse qui s’est produit, car pour rembourser l'emprunt la Bibliothèque nationale de France s’est tournée vers un modèle économique qui passe par la marchandisation de la matière brute du domaine public lui-même, et donc par son anéantissement pur et simple.

\section{Une dégradation encore renforcée par la crise et les coupes budgétaires}

Il est évident que la période qui s’ouvre va constituer un risque majeur pour l’intégrité du domaine public sous forme numérique. Les fortes restrictions budgétaires annoncées par le Ministère de la Culture et le climat de crise économique ambiant vont nécessairement peser sur les capacités des institutions culturelles à numériser leurs collections par leurs propres moyens.

Il y a donc de fortes chances que des partenariats public-privé de plus en plus déséquilibrés en termes d’accès et de réutilisation des contenus soient conclus, afin de permettre aux institutions de dégager des ressources propres. Ces considérations économiques vont être très difficile à contrer.

Pourtant, cette optique de marchandisation du domaine public relève d’une bien courte perspective économique. Numériser et de rendre réutilisable le domaine public en ligne, y compris à des fins commerciales, peut constituer un effet de levier important sur de nombreux secteurs d’activité~: 

\begin{itemize}
\item Le domaine public numérisé peut être utilisé par les chercheurs et les enseignants.
\item Les consommateurs ont accès à une offre légale gratuite.
\item Le secteur marchand peut adapter ou rééditer ces œuvres dans le cadre de services et d’applications.
\end{itemize}

Cette richesse potentielle induite par la diffusion gratuite du domaine public ne pourra se déployer que si l’on met pas d’entrave à la réutilisation des oeuvres. Avant tout il faut continuer à les diffuser en ligne et non les réduire en produits commerciaux coupés du web.

À défaut, on aboutira à une abolition pure et simple du domaine public sous forme numérique, alors que la numérisation aurait dû permettre au contraire d’en réaliser toutes les promesses.
