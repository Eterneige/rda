\chapter{Le plaidoyer de l'amoureux de la liberté d'expression}\label{libe}

\section{Des 0 et des 1}
Les tentatives de renforcer l’interdiction actuelle du partage
non-commercial de la culture entre les citoyens menacent les droits fondamentaux comme le droit au
secret de la communication, celui à la liberté d’expression et même celui à un procès juste.

Le partage de fichier, c’est quand deux individus s’envoient des uns et des zéros. La seule manière
de limiter le partage de fichier c’est de surveiller ces suites binaires et donc toutes les
communications privées des uns et des autres. Il n’existe pas de possibilité de séparer les données
protégées des données privées. Il faut ouvrir tous les contenus pour les examiner. Le secret postal,
le droit de communiquer en privé avec son avocat ou à l’intimité lors des flirts par webcam
disparaissent tout comme la protection des sources pour les journalistes.

Je ne suis pas prêt à abandonner nos droits fondamentaux pour renforcer le droit d’auteur. Le
droit à la vie privée est plus important que le droit des grandes sociétés à continuer à gagner leur
vie comme avant. Ce dernier droit n’existe pas.

\section{Le droit d’auteur menace le droit au secret de la correspondance}

Le respect du droit d’auteur actuel est irréconciliable avec celui de la vie
privée.

Si je vous envoie un courriel, celui-ci peut contenir un morceau de musique. Si nous avons une
conversation vidéo, je peux partager une vidéo protégée. Pour faire appliquer le droit d’auteur, il
faut pouvoir le détecter. Or le seul moyen de détecter des contenus protégés est d’écouter tous les
0 et les 1 entrants et sortants des ordinateurs.

Il n’y a aucun moyen de permettre le droit à la correspondance privée pour certains contenus et pas
pour d’autres. Vous devez briser le sceau et analyser les contenus pour les trier en deux catégories~: autorisé et non autorisé. Dès ce moment le sceau est brisé. Soit il y a un sceau sur tout soit
il
n’y en a aucun.

Plus généralement tout canal de communication numérique qui peut être utilisé pour échanger des
informations privées peut aussi être utilisé pour transférer à des inconnus des copies numériques de
fichiers protégés. Faire absolument respecter le droit d’auteur implique donc un viol massif du
secret de la correspondance. 

Les lois qui s’appliquent hors-ligne devraient aussi s'appliquer en ligne. C’est tout à fait raisonnable. Internet n’est pas un cas particulier mais fait partie de la réalité. Le problème apparaît quand une industrie dépassée mais puissante réalise qu’une application juste et égalitaire de la loi signifie que son monopole de la
distribution est terminé.

Pour comprendre l’absurdité des requêtes de l’industrie du droit d’auteur, on doit se demander quels
droits nous considérons pour acquis dans le monde analogique. Ces droits doivent aussi s’appliquer
au monde numérique, puisqu’au moins en théorie la loi ne fait pas de différences entre les moyens
de communication.

Regardons quels droits j’ai quand je communique avec quelqu’un via des canaux analogiques, en
utilisant du papier, un stylo, une enveloppe et un timbre, c’est-à-dire quand j’écris une lettre~:

\begin{itemize}
\item 
J’ai le droit de communiquer anonymement ou non. Moi et moi seul choisis de m’identifier dans
l’adresse sur l’enveloppe et/ou dans l’enveloppe.\item
Personne n’a le droit de l’intercepter pour briser le cachet et examiner son contenu sauf si je suis
déjà suspecté d’un crime. Dans ce cas cela passe par une procédure judiciaire.\item
Personne n’exige que je l’aide à ouvrir mes lettres pour me surveiller.\item
Nul n’a le droit d’en altérer le contenu ou de refuser de l’acheminer.\item
Nul n’a le droit de rester devant la boîte et de noter à qui j’écris, la durée de mes messages ou
qui me répond.\item
Le facteur n’est pas responsable de ce que j’ai écrit. Il n’est qu’un intermédiaire.\end{itemize}

Toutes ces règles fondamentales sont systématiquement attaquées par l’industrie du droit d’auteur.

Cette industrie exige que les fournisseurs d’accès à Internet installent des appareils
d’enregistrement et de censure des communications au milieu de leurs propres appareils. Elle les
poursuit en justice pour cela. Elle se plaint sans cesse de l’immunité des intermédiaires. Elle
demande aux autorités d’identifier les gens qui communiquent, nous niant ainsi nos droits
fondamentaux, surtout ceux de la liberté d’expression et au secret de la correspondance. Elle a
l’outrance d’aller jusqu’à demander la censure des télécommunications.

De tels droits font partie des libertés pour la défense desquelles nos aïeux ont donné leur sang.
Il est plus qu’obscène de les abandonner au profit d’une industrie obsolète qui désire préserver son
monopole et veut pour ce avoir à titre privé encore plus de pouvoirs que la police quand elle
recherche des criminels.

Lorsque les photocopieurs sont arrivés dans les années 1960, les éditeurs ont tenté de les interdire
au prétexte qu’ils pourraient être utilisés pour copier des livres que les gens s’enverraient
ensuite par la poste. À l’époque, tout le monde a expliqué aux éditeurs la dure vérité~: Bien que le
monopole du droit d’auteur soit valide, personne n’a le droit de briser le secret de la
correspondance simplement pour vérifier qu’aucune violation du droit d’auteur n’est commise, donc il
n’y a rien à faire. La règle tient toujours hors-ligne. Pourquoi ne s’appliquerait-elle pas en
ligne~?

L’industrie du droit d’auteur se plaint parfois de la licence qui existerait sur Internet alors que
les mêmes lois devraient valoir pour les deux domaines. Sur ce point, je ne peux être plus d'accord.

Malheureusement, c’est bien le contraire qui est en train de se passer. Les corporations essayent de
prendre le contrôle de nos moyens de communications en prétextant des problèmes de droit d’auteur.
Le plus souvent elles aident les politiciens qui aspirent à la même chose mais prétextent eux des
impératifs de lutte contre le terrorisme ou des craintes McCarthystes pour la sécurité publique.
Regardons par exemple ce qui s’est passé dans le monde arabe ou en Angleterre en 2011.

Les blanc-seings aveuglément donnés aux autorités me font peur car je me
méfie de l’incessant désir de savoir ce dont je discute et avec qui qu'affichent ouvertement les corporations et les politiques.

Pour dramatiser la situation, il ne s’agit pas que de mouchardage. Les corporations et les
politiques veulent gagner le droit de nous réduire au silence.

L’industrie du droit d’auteur exige le droit de tuer les commutateurs essentiels de nos
communications. Pour peu que nos propos soient suffisamment dérangeants, que ce soit selon les
membres de l’industrie du divertissement ou de la classe politique dirigeante, la ligne est coupée.

Il y a 20 ans, cela aurait paru être une horreur absolue. Aujourd’hui, c’est devenu la réalité. Vous
ne le croyez pas~? Essayez de parler de \site{The Pirate Bay} sur MSN ou Facebook et admirez le silence.
L’industrie du droit d’auteur se bat pour devenir de plus en plus envahissante. De même certains
politiciens ont en poche leur propre calendrier sur la question.

Bien que l’industrie du droit d’auteur et les politiciens Big Brother ne partagent pas les mêmes
motivations, ils poussent dans la même direction et promeuvent les mêmes changements.

Pendant ce temps, les déplacements des citoyens dans les rues sont tracés minute par minute et leur
historique est enregistré.

Comment imaginer une révolution possible quand tout ce que vous dites est étouffé dans l’œuf avant
d’atteindre une quelconque oreille et quand le régime peut surveiller qui a rencontré qui, où et
quand, voire peut vous déconnecter par un simple appui sur un bouton distant~? Comment exercer son
droit de résistance à l’oppression dans un monde où absolument tout est sans cesse surveillé~?

\section{La riposte graduée, ou l’art de débrancher les gens par caprice}

«~Trois coups et t’es dehors~» est une expressione américaine qui a ses origines dans le
base-ball. Les
politiciens américains l'ont transformée en principe légal. Dans le contexte de la régulation d'Internet,
«~trois coups~» signifie que quelqu’un accusé de partage illégal trois fois de suite par des
ayants-droits est déconnecté. «~Réponse graduée~» est l’expression la plus utilisée en France.

En France, nous avons la loi Hadopi, qui demande aux Fournisseurs d’Accès
à Internet (FAI) de
débrancher ceux qui ont reçu un, deux ou trois avertissements. Ceci dit, en France, les
avertissements ne sont pas nécessaires, la déconnexion peut être immédiate. En Angleterre, le
Digital Economy Act dit essentiellement met la même chose. En Italie, qui a voulu
faire mieux que
tout le monde, un seul
avertissement suffit pour être banni.

Sur le principe, ces lois laissent aux majors la liberté de se poser en juge et partie, en désignant
les individus suspectés de porter atteinte à leurs droits et en forçant les FAI à appliquer la
sanction qu’est la déconnexion.

Laissons de côté la question de savoir si privatiser l’application de la loi est une bonne idée. Ce
que je défends ici c’est que vouloir débrancher arbitrairement les gens d’Internet est un
caprice irresponsable. Caprice car, comme je le défends, il faut réformer le droit d’auteur
plutôt que l’appliquer aveuglément. Irresponsable à cause des effets qu’une coupure d’Internet a sur
les membres d’un foyer.

Considérons deux secondes ce que ça veut concrètement dire~:

\begin{itemize}
\item 
L’arrêt des études

La plupart des systèmes scolaires, particulièrement l’université, prennent pour acquis que vous avez
un accès Internet. Si vous êtes un étudiant, vous aurez besoin d’un accès Internet pour toutes les
choses de la vie courante comme vous tenir au courant de votre emploi du temps, préparer des
exposés en groupes ou chercher la littérature sur un sujet précis. Les études montrent que la
majorité des étudiants piratent. Devriez-vous leur couper leur accès et arrêter là leurs études~? Ou
peut-être faire un exemple sur 5-10\%~? Quel est le meilleur sacrifice~?

\item 
La mort des petites entreprises

Si vous possédez une PME, vous dépendez entièrement d’Internet quelque soit votre branche. Pour
contacter les clients, mettre à jour vos actualités, commander vos fournitures et simplement
correspondre. Est-il raisonnable de ruiner le père ou la mère de famille parce que l’adolescent a
téléchargé le dernier tube à la mode~? Couper une connexion punit un foyer entier.
\item 
La coupure des relations sociales

Les jeunes sont socialisés en grande partie via Internet. Il n’est pas rare que certains aient des
connaissances très proches qu’ils n’ont que rarement rencontrés en vrai. Ce n’était pas le cas il y
a 30 ans mais les choses ont changé. Être tout d’un coup coupé du monde est normalement réservé aux
dangereux criminels.
\item 
La perte des droits civiques

L’accès à Internet est devenu essentiel pour prendre part aux débats public. Non seulement pour être
tenu au courant mais aussi pour monter son propre blog, commenter ceux des autres, réagir en direct
sur Twitter et organiser des événements ou les rejoindre.\end{itemize}

«~Si vos enfants sont méchants, nous leur enlevons leurs jouets~» est un bon résumé de l’approche
répressive des politiques qui défendent la riposte graduée.

Mais les citoyens ne sont pas des enfants, n’ont pas à subir l’arrogance de leurs représentants.
Internet n’est pas un jouet. C’est devenu un rouage essentiel de la société, et une infrastructure
dont tout le monde a besoin.

C’est dans cet esprit que le Conseil Constitutionnel
français 
a déclaré le 10 juin 2009 que la suspension ne pouvait
passer que par un juge car~:
\begin{quotation}

Aux termes de l’article 11 de la Déclaration des droits de l’homme et du citoyen de 1789~:
\begin{quotation}
«~La libre communication des pensées et des opinions est un des droits les plus précieux de l’homme~: tout citoyen peut donc parler, écrire, imprimer librement, sauf à répondre de l’abus de cette
liberté dans les cas déterminés par la loi.~»
\end{quotation}

En l’état actuel des moyens de communication et eu égard au développement généralisé des services de
communication au public en ligne ainsi qu’à l’importance prise par ces services pour la
participation à la vie démocratique et l’expression des idées et des opinions, ce droit implique la
liberté d’accéder à ces services.
\end{quotation}

Les politiques qui ne le comprennent pas ne devraient pas être surpris que la jeune génération ne
vote pas pour eux.

\section{De toute manière, la répression ne marche pas}
La politique répressive des lobbys de la culture n’a jamais donné que des résultats éphémères et
incertains. Car rien n’arrêtera le piratage.

Il y a quelques siècles, la peine pour une copie interdite a fini par être le supplice de la roue.
Après d’affreuses souffrances le condamné mourrait de soif dans les jours suivant la destruction de
ses membres.

Le monopole de copie à cette époque concernait les modèles de couture. C’était au 18\ieme siècle en
France avant la Révolution. Certains modèles était plus populaires que d’autres et pour remplir un
peu plus ses caisses, le Roi avait vendu des monopoles d’exploitation à quelques nobles privilégiés qui en retour pouvaient casser des bras et des jambes (et le faisaient).

Mais les paysans et roturiers pouvaient produire ces modèles d’eux-mêmes. Ils pouvaient les
pirater et le firent largement. Les nobles demandèrent donc justice au Roi. Le Roi commença par
introduire des amendes, puis des châtiments corporels mineurs, puis finit par condamner ces
infractions au monopole nobiliaire par la torture et ne condamna pas seulement une poignée de
pauvres hères.

L’économiste et historien Eli Heckscher écrit dans son classique \livre{Merkantilismen}~:
\begin{quotation}
 Bien sûr, l’essai de stopper un développement encouragé par des modes féminines éphémères ne pouvait
pas réussir. On considère qu’en France la police a tué 16 000 personnes pour copie interdite sans
compter ceux condamnés aux galères. À Valence, il arriva que 77 personnes soient pendues en une seule fois, avec 58
rouées et 631 envoyées aux galères, un acquitté. Mais l’usage du calicot imprimé dont la copie était
réprimée a continué à se répandre en France et ailleurs.
\end{quotation}

Voilà le plus fascinant~:
\begin{quotation}
 La peine capitale n’a pas réussi à ralentir le piratage des fabriques des nobles. Même ceux qui
connaissaient des artisans exécutés et torturés continuèrent à pirater sur le même rythme.
\end{quotation}

Cela remet sérieusement en doute la pertinence d’une politique répressive. Combien de temps encore
est-ce que les politiciens continueront à croire que la répression sert à quelque chose quand
l’histoire nous apprend que même la peine de mort n’empêche pas un phénomène de se propager à toute
vitesse et de perdurer~?

Pour résoudre le problème, nous devons trouver une autre solution. Celle-ci existe. Une fois que
vous acceptez de réduire la protection des œuvres sous droit d’auteur et d’autoriser le partage
non-commercial, une foultitude d’avantages apparaît. Les deux milliards d’humains connectés de la
planète auraient accès 24h/24 à une culture immense à un coût dérisoire. C’est un
énorme progrès par rapport à la bibliothèque d’Alexandrie. La technologie existe. Il reste à
l’accepter.

\section{Partage de fichiers et droits fondamentaux – la ligne de tension}
 La relation entre le partage de fichier et les droits fondamentaux est très simple. Le partage de
fichier est là pour rester. Peu importe ce que les uns ou les autres feront, cela ne changera pas les faits. À long terme, il deviendra impossible de faire payer
simplement pour des copies numériques. Cela fait partie de l’histoire de la technologie et il n’y a
rien d’autre à ajouter.

Alors quel est le problème~? L’industrie du droit d’auteur ne sera pas capable d’arrêter le partage
de fichiers. Les pirates trouveront d’autres moyens de se protéger grâce à l’anonymisation, le
chiffrement, etc… Aucun problème pour eux. Mais l’industrie du droit d’auteur
punit et punira des
individus pris au hasard de façon dure et disproportionnée pour l’exemple.Ceci n’est pas acceptable. 

Je viens d'expliquer que la seule façon d’essayer de réduire le partage de fichiers est d’introduire la surveillance de masse
de tous les utilisateurs d’Internet. Mais même cela n’est pas très efficace comme l’ont montrées
les expériences des décennies passées. L’industrie du droit d’auteur sait tout cela.

Ceux qui pensent que le partage de fichiers est dangereux pour la société et qu’il faut
l’éliminer doivent se demander s’ils sont préparés à accepter une société totalement surveillée pour
arriver à cela. Parce qu’une fois que les systèmes de surveillance ont été installés, ils peuvent
être utilisés pour n’importe quoi qui plaira aux personnes qui les maîtrisent.

Vous pourriez avoir l’impression que vous n’avez «~rien à cacher~» lorsqu’il s’agit de partage
de fichiers, si c’est quelque chose que vous ne pratiquez pas. Mais pouvez-vous être certains que
vous n’aurez pas toujours «~rien à cacher~» quand il s’agira d’exprimer des points de vue que le futur
gouvernement pourrait ne pas apprécier~? Savez-vous déjà que vous serez loyal au gouvernement à la
prochaine période de McCarthysme ou pire, quand l’État commencera a écouter et à verrouiller
certaines sympathies politiques~?

Si vous construisez un système de surveillance de masse, il y aura un système de surveillance de
masse à disposition pour tous les abus. Voilà l’essentiel du problème, sa ligne de tension.

