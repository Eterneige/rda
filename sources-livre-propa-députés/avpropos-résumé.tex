\chapter*{Avant-propos de l'éditeur-traducteur}
\addcontentsline{toc}{chapter}{Avant-propos de l'éditeur-traducteur}

Ce livre est un livre de remix. Il compile des argumentaires et revendications portés par différentes associations et collectifs français. Il n'est pas nécessaire de le lire dans l'ordre. Les plus pressés iront directement à l'index des mesures proposées situé en page \pageref{index}. J'enjoins au lecteur d'essayer de découvrir les différents chapitres sans préjuger immédiatement de leur source originelle.

La croyance fondamentale qui guide les mesures proposées est que la libre circulation de l'information qu'Internet a facilitée devrait permettre un monde meilleur. C'est pourquoi le droit d'auteur dans sa forme actuelle est un frein au progrès. L'État doit abandonner une législation obsolète pour entrer avec fracas dans le nouveau millénaire. Ouverture et liberté doivent être les maîtres mots de la nouvelle société de l'information pour que nous en tirions le meilleur parti. 

Ce discours a déjà été répété à l'envi par de multiples personnes. Parmi les parlementaires et l'exécutif français comme européen, le nombre de gens qu'il convainc ne cesse de croître. En même temps, les industries de l'ancien monde continuent de lutter âprement pour retarder leur chute et éviter de changer de modèle économique. Elles sont soutenues par quelques politiques et industriels qui n'ont pas grandi dans un monde où l'information veut être libre et ont peur des nouvelles libertés qui s'offrent à nous. 

Pour ceux qui ne sont pas convaincus par les arguments développés dans ce livre, nous avons compilé en ligne un recueil\footnote{disponible en ligne à \url{http://fichiers.sploing.fr/contexte.pdf} au format PDF ou comme page web à \url{http://rda.sploing.fr/partie-2-les-temoignages}} de témoignages de personnages qui vise à mieux leur faire comprendre les enjeux de la nouvelle ère et à leur donner un avant-goût des bienfaits que ces libertés nous apporteront. Pour ceux qui doutent même de la pertinence de remettre en cause le droit d'auteur à l'heure actuelle, nous avons inclus une courte allégorie cycliste au tout début du livre.

Ces deux livres, le recueil de propositions et celui de témoignages, se veulent une bouteille à la mer~: les lira qui voudra pour en tirer les enseignements qu'il voudra. Nous, internautes qui avons financé l'impression et l'envoi à nos députés du premier livre de mesures, nous enjoignons nos députés à lire les deux livres avec attention et à transformer nos propositions en lois. 
