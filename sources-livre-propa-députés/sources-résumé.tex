\chapter{Index des mesures}\label{index}

\listtheorems{mesure}

\chapter{Sources}


Les œuvres que j'ai utilisées sont~:

\begin{itemize}
\item les billets \livre{I Have A Dream : une loi pour le domaine public en France~!} et \livre{Pour un droit au mashup, mashupons la loi !} disponibles à \url{https://scinfolex.wordpress.com/2012/10/27/i-have-a-dream-une-loi-pour-le-domaine-public-en-france/} et \url{http://scinfolex.wordpress.com/2013/06/20/pour-un-droit-au-mashup-mashupons-la-loi/} dans le domaine public de Lionel Maurel alias Calimaq pour les chapitres \ref{dompub} et \ref{remix}.
\item les \livre{Eléments pour la réforme du droit d'auteur et des politiques culturelles liées} disponibles à \url{https://www.laquadrature.net/fr/elements-pour-la-reforme-du-droit-dauteur-et-des-politiques-culturelles-liees} de la Quadrature du Net pour les chapitres \ref{verrous}, \ref{dompub}, \ref{licencelibre}, \ref{registre}, \ref{depen} et \ref{financement}.
\item l'article \livre{Licence libre} de Wikipedia disponible à \url{http://fr.wikipedia.org/wiki/Licence_libre} sous licence \href{http://creativecommons.org/licenses/by-sa/2.0/fr/}{CC-BY-SA} et l'article \livre{Définition} de Actions Open Data disponible à \url{http://actionsopendata.org/l-open-data/definition/} sous licence \href{http://creativecommons.org/licenses/by-sa/2.0/fr/}{CC-BY-SA} pour le chapitre \ref{licencelibre}.
\item le rapport Lescure disponible à \url{http://culturecommunication.gouv.fr/Actualites/A-la-une/Culture-acte-2-75-propositions-sur-les-contenus-culturels-numeriques} pour les propositions des chapitres \ref{verrous}, \ref{licencelibre} et \ref{depen}.
\item le rapport Bernard Lang \livre{L'exploitation des œuvres orphelines dans les secteurs de l'écrit et de l'image fixe} disponible à \url{http://bat8.inria.fr/~lang/orphan/oeuvres-orphelines-BLang.pdf} pour le chapitre \ref{remix}.
\item les positions de l'AFUL sur la question des œuvres orphelines et sur celle des licences libres disponibles à \url{http://aful.org/droit-auteur/index/oeuvres-orphelines/} et \url{http://aful.org/association/positions} pour les chapitres \ref{remix} et \ref{licencelibre}.
\item le pacte du logiciel libre proposé par l'April disponible à \url{http://www.april.org/pacte-du-logiciel-libre} pour le chapitre \ref{licencelibre}.
\item ma traduction du billet de blog \livre{Mein Rad} de Marcel-André Casasola Merkle disponible à \url{http://traductions.sploing.fr/politique/2012/05/09/mon-velo/} sous licence \href{http://creativecommons.org/licenses/by-sa/2.0/fr/}{CC-BY-SA} pour le chapitre \ref{cycliste}.
\item ma traduction du livre du Parti Pirate Suédois \livre{The Case for Copyright Reform} disponible à \url{http://reformedroitauteur.sploing.fr/} sous licence \href{http://creativecommons.org/licenses/by-sa/2.0/fr/}{CC-BY-SA} pour l'introduction générale en page \pageref{premintro} et les chapitres \ref{pater}, \ref{verrous}, \ref{remix}, \ref{registre}, \ref{depen} et \ref{dur}. 
\item ma traduction du site de la campagne de la Digitale Gesellschaft pour le droit au remix disponible à \url{http://politiquedunetz.sploing.fr/2013/06/campagne-de-la-digitale-gesellschaft-pour-le-droit-au-remix/} pour le chapitre \ref{remix}.
\end{itemize}

\nocite{lessig2004culture}
\nocite{florent2004bon}
\nocite{benkler2009richesse}
\nocite{benkler2009richesse}
\nocite{manach2010vie}
\nocite{lessig2005avenir}
\nocite{raymond1998cathedrale}
\nocite{stallman2010richard}
\nocite{sagot2002propriete}
\nocite{perline2004bataille}
\nocite{aigrain2008internet}
\nocite{hadopi}
\nocite{opendata}
\nocite{smiers2011monde}

