\chapter{Vingt ans de monopole commercial}\label{dur}

L’essentiel de l’industrie du divertissement actuelle est bâtie sur l’exclusivité commerciale des
travaux protégés. Nous voulons sauvegarder cette activité. Mais les durées d’exclusivité actuelles
sont absurdes. Aucun investisseur ne voudrait attendre un retour sur investissement aussi long.

\begin{mesure}[Réduction de la durée de protection]
 Nous souhaitons raccourcir les durées de protection à quelque chose de raisonnable à la fois du
point de vue de la société et des investisseurs et nous proposons vingt années à partir de la
publication.

Nous souhaitons la même période de protection pour tous les types de création.
\end{mesure}

\section{À productions différentes, durées différentes~?}

Ne serait-il pas judicieux d’avoir des durées de protection différentes pour les différents types de
création~? Vingt années de protection pour un programme informatique a certainement différentes
implications que vingt années pour un morceau de musique ou un film. Ne serait-il pas mieux
d’adapter les durées de protection selon ce qui est raisonnable pour chaque type de création~?

Le chiffre sur lequel il faut se prononcer est arbitraire. Cela pourrait être quinze ans ou vingt-et-un ans ou dix-huit ans sans changer grand chose. C'est donc partiellement une affaire de sensibilité personnelle et chacun va vouloir une durée de protection longue pour le type de création qui est le plus proche de sa sensibilité personnelle. Devoir définir des valeurs semi-arbitraires pour
chaque catégorie de production réduit donc les chances de trouver une
solution que l’on peut défendre de façon objective.

\section{Une durée rationnelle pour un investisseur}

Si l'on regarde la question du point de vue d’un investisseur, les choses deviennent
différentes. L’industrie de la musique a beau être très différente du secteur du logiciel, ils ont
quelque chose en commun. L’argent c’est de l’argent, quelque soit le secteur dans lequel on choisit d’investir.

Lorsqu’un investisseur prend la décision d’investir dans un projet, quelle que soit l’industrie – cela
peut être la musique, le cinéma, le logiciel grand public, ou tout autre chose – cet investisseur
établira sa stratégie avec une limite de temps pour obtenir un retour sur investissement. Si le
projet se développe selon les prévisions, il est supposé couvrir ses coûts et dégager des bénéfices
dans les x années. Si tel n’est pas le cas, c’est un échec.

X est toujours petit dans ce genre de prévisions. Que quelqu’un établisse une stratégie de
développement concernant un projet culturel dont le délai de retour sur investissement est supérieur
à trois ans est hautement improbable. Les personnes qui construisent des ponts, des réacteurs
nucléaires et autres infrastructures effectuent évidemment des investissements à plus long terme,
mais en dehors de ces industries les stratégies de développement de plus de trois ans ne sont
vraiment pas courantes.

C’est encore plus vrai dans le domaine de la culture. Qui peut prédire ce qui sera à la mode dans
deux ou trois ans, dans un paysage aussi changeant que celui de la culture~? On attend de la plupart
des projets culturels qu’ils génèrent des bénéfices dans l’année.

En considérant les durées de protections du point de vue d’un investisseur, on peut justifier le
fait d’avoir les mêmes durées pour toutes les créations. Le but de l’exploitation exclusive du droit
d’auteur est d’attirer les investisseurs vers le marché de la culture. Les investisseurs pensent
la même chose sans tenir compte de ce dans quoi ils sont en train d’investir.

Un projet doit dégager des bénéfices dans l’année ou les suivantes, autrement
c’est un échec. La faible probabilité que le projet que vous avez financé se révèle indémodable et
continue de générer des profits pendant des décennies est une chance pour l’investisseur, mais ça
n’a pas sa place dans un projet de développement sérieux.

\section{Pourquoi pas moins~?}

L’important c’est de se débarrasser des durées de protection actuelles d’une vie ou plus. Ces
longues périodes sont clairement néfastes pour la société puisqu’elles gardent la plupart de notre
héritage culturel commun bloqué même longtemps après que la majorité des productions aient perdu
toute valeur commerciale pour les ayants-droits. C’est une perte sèche économiquement parlant et un
scandale culturellement parlant.

Si les durées de protections étaient réduites à 20 ans, cela résoudrait la plupart des problèmes «
du trou noir du 20\ieme siècle~», et permettrait aux bibliothécaires et archivistes de commencer
l’urgente tâche de préservation des créations du 20\ieme siècle qui se dégradent dans les archives, en
les numérisant. Cinq ou dix ans seraient plus appropriés pour favoriser l’archivage, mais 20 ans
devraient convenir.

Dans le même temps, 20 ans est encore suffisant pour nourrir le rêve plaisant (mais hautement
improbable) de créer un succès majeur indémodable qui génère des revenus durant des décennies. Si
votre prochain projet trouve le bon filon et vous propulse soudainement sous les feux des
projecteurs pour longtemps tel Paul Mc Cartney ou ABBA, 20 ans devraient
être plus que suffisants pour que vous deveniez très riche et que vous n’ayez plus jamais jamais à
vous soucier d’argent.
