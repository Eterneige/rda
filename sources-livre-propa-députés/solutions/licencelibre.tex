\chapter{Reconnaître et promouvoir les licences libres}\label{licencelibre}

\section{Inscrire la définition d'une licence libre dans la loi}
La définition des licences libres relèvent actuellement purement du droit contractuel. La loi ne définit pas ce qu'est une licence libre. Or s'il est demandé aux services de l'État de publier leurs œuvres et données sous des licences libres, il faut qu'ils puissent se référer à des textes légaux définissant quelles licences sont considérées comme libres par l'État.

\begin{mesure}[Définition administrative des licences libres]
Inscrire dans la loi sur le droit d'auteur la définition suivante~:

\begin{quotation}
Une œuvre est réputée libre lorsque sa licence confère à toute personne morale ou physique, en tout temps et en tout lieu, les quatre possibilités suivantes~:
\begin{itemize}
\item La possibilité d'utiliser l'œuvre, pour tous les usages ;
\item La possibilité d'étudier l'œuvre ;
\item La possibilité de redistribuer des copies de l'œuvre ;
\item La possibilité de modifier l'œuvre et de publier ses modifications.
\end{itemize}
\end{quotation}
\end{mesure}

Bien sûr cette précision légale ne change rien pour les usages des particuliers non plus qu'elle n'impose aux communautés qui définissent leurs propres critères de liberté une définition particulière de la liberté des œuvres et données.

\section{Libérer toutes les données et œuvres qui peuvent l'être}
Nous proposons aussi de rendre obligatoire la libération des données et œuvres produites par les services de l'État ou subventionnés par l'État et ses services. Par exemple, un logiciel développé sur commande de l'administration devrait être libre, tout comme des cartes, des travaux de recherche ou un catalogue des métadonnées des œuvres enregistrées. Le rôle de l'État est de favoriser la diffusion de la connaissance et l'initiative individuelle, non de marchander cette connaissance déjà financée par les impôts des Français ou d'accorder des monopoles aux entreprises les plus en grâce avec ses agents. 

\begin{mesure}[Libération des œuvres subventionnées]
Toutes les œuvres ou données immatérielles produites sur commande de l'État ou de ses services ou au moins financées à hauteur de 50\% par ceux-ci doivent être publiées sous licence libre gratuitement ou pour un coût d'accès marginal. Cette libération s'effectue au plus tard dans les deux ans suivant la fin de la production de l'œuvre. Les données confidentielles ou critiques pour la sécurité publique sont les seules à ne pas devoir être publiées.
\end{mesure}

L'État a tout intérêt à travailler en co-création avec ses citoyens plutôt que contre eux. Le service de l'État Étalab a déjà publié une licence libre aux termes définis ci-dessus, appelée simplement \textit{Licence ouverte}. Cette licence pourrait être la licence standard des publications des bases de données crées par l'administration, sauf précisions spécifiques. 

Les œuvres qui relèvent du code de la propriété intellectuelle pourront être placées sous des licences libres comme la \textit{Creative Commons Paternité} ou \textit{Art libre} par exemple. La \textit{Creative Commons Paternité}, plus communément désignée CC-BY, n'impose comme condition à la réutilisation que la reconnaissance de la paternité, tandis que la licence \textit{Art libre} exige en plus que les œuvres dérivées soient publiées sous la même licence.

\subsection{La sécurité des citoyens comme limite}
Si nous insistons sur la nécessité de publier le plus grand nombre de données possibles, c'est parce qu'il n'est pas possible à l'État de prévoir quelles données seront utiles ou non. Bien souvent, c'est justement en reliant plusieurs bases de données individuellement anodines que l'on extrait des informations utiles. L'État ne peut pragmatiquement pas imaginer tous les liens qui peuvent être faits avec les données publiées. De plus, imposer que \emph{toutes} les données doivent être publiées interdit de publier des bases de données volontairement incomplètes et oblige à publier les données sources. 

La seule limite à cette logique d'ouverture de l'État doit être la protection de la vie privée ou de la sécurité des citoyens. Dans le cas de base de données brutes il faut prendre en compte la possibilité d'identifier les citoyens même si les données sont anonymisées. Par exemple, si la publication de données médicales comporte toujours le code postal et la date de naissance exacte des patients et permet à des assureurs de retrouver facilement quels clients sont affectés par des maladies graves, alors cette publication doit être être interdite telle quelle. Le niveau de granularité de la base de données doit être augmenté de sorte que cibler les individus ne soit plus possible.

\subsection{Libérer les travaux de recherche}

La libération des œuvres et données produites par les services de l'État concernerait en particulier les chercheurs et instaurerait une obligation nationale de publication en accès libre des travaux scientifiques financés par l'État. Cependant elle n'obligerait pas les chercheurs à ne publier que dans des revues en accès libre. La seule contrainte serait qu'après la première publication les travaux puissent être reversés dans des sites de dépôt qui soient en accès libre.  Cette contrainte permettra de protéger les travaux des chercheurs contre les maisons d'édition qui cherchent à se les approprier et à empêcher leur diffusion. Des sites de dépôts en accès libre gérés par des universités ou des centres de recherche existent déjà en France. Il est donc possible de s'appuyer sur l'existant pour poursuivre le mouvement.

 \section{Rendre accessibles les données libérées}
Publier des données sous licence libre n'est pas suffisant si celles-ci ne sont pas facilement accessibles parce que leur accès est onéreux, parce que le format de diffusion n'est pas utilisable informatiquement ou est non-standard et nécessite de payer des licences logicielles élevées pour être lu.

\begin{mesure}[Conditions d'accessibilité des données libérées]
Les données produites et publiées par l'administration sous licence libre doivent respecter les contraintes suivantes~:

\begin{itemize}
\item Entières~: Les bases de données sont intégralement publiées.
\item Brutes~: Leur format est directement utilisable par un ordinateur.
\item Documentées~: Elles sont accompagnées de leurs métadonnées dans un format documenté.
\item Interopérables~: La documentation du format de fichier est aisément accessible et complète. 
\item Actuelles~: Elles sont les plus récentes possibles.
\item Permanentes~: Leurs adresses d'accès sont durables.
\item Gratuites ou peu coûteuses~: Le coût d'accès est nul ou marginal.
\end{itemize}
\end{mesure}

Le moyen technique le plus simple de remplir ces contraintes est de mettre en place via Internet des portails de dépôt qui les recensent et les mettent à disposition via des protocoles standards comme ceux utilisés pour afficher les pages web. Le gouvernement français a déjà lancé le processus via le site Étalab mais les ressources humaines, juridiques ou informatiques du portail doivent être renforcés pour améliorer la qualité de service. 

Vu que les mesures s'appliquant aux services de l'État s'appliqueront aussi aux matériaux produits par l'Éducation nationale, elles auront comme effet direct de permettre aux déficients auditifs ou visuels d'accéder plus facilement à un plus grand nombre de ressources éducatives et culturelles, en favorisant l'utilisation de formats interopérables.

