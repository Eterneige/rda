\chapter{Reconnaître et promouvoir les licences libres}\label{licencelibre}

\section{La possibilité d'inscrire la définition d'une licence libre dans la loi}
La définition des licences libres relèvent actuellement purement du droit contractuel puisque la loi ne définit pas ce qu'est une licence libre. Il n'est pas nécessaire pour l'État de définir lui-même ce que des associations comme l'Open Knowledge Foundation ont déjà défini collaborativement. Il est en revanche nécessaire que les services de l'État se réfèrent à une définition précise des licences libres, qui est la suivante~:

\begin{quotation}
Une œuvre est réputée libre lorsque sa licence confère à toute personne morale ou physique, en tout temps et en tout lieu, les quatre possibilités suivantes~:
\begin{itemize}
\item La possibilité d'utiliser l'œuvre, pour tous les usages ;
\item La possibilité d'étudier l'œuvre ;
\item La possibilité de redistribuer des copies de l'œuvre ;
\item La possibilité de modifier l'œuvre et de publier ses modifications.
\end{itemize}
\end{quotation}

Des exemples de licences libres qui peuvent être appliquées à des œuvres ou données sont les licences Creative Commons CC-BY ou CC-BY-SA, la licence Art Libre, les licences Cecill, la General Public Licence, la licence BSD ou encore les licences gouvernementales Open Data (Licence Ouverte en France, Open Government Licence au Royaume-Uni). 

Pour les oeuvres et données produites par les administrations publiques, une préférence devrait être accordée aux licences comportant une clause de partage à l'identique (Share Alike) si ces administrations désirent limiter les risques de réappropriation exclusive. 

\section{Libérer toutes les données et œuvres qui peuvent l'être}
Nous proposons de rendre obligatoire la libération des données et des œuvres produites ou subventionnées par les services de l’État ou des collectivités locales. Par exemple, un logiciel développé sur commande de l’administration devrait être libre, tout comme des cartes, des travaux de recherche ou un catalogue des métadonnées des œuvres enregistrées. Le rôle des personnes publiques est de favoriser la diffusion de la connaissance et l’initiative individuelle, non de marchander cette connaissance déjà financée par les impôts des citoyens ou d’accorder des monopoles à des entreprises. C'est particulièrement le cas dans le domaine de la recherche scientifique et les travaux des chercheurs produits sur des fonds publics devraient faire l'objet d'une publication en libre accès sous licence libre.

\begin{mesure}[Libération des œuvres subventionnées]
Toutes les œuvres ou données immatérielles produites sur commande des personnes morales de droit public ou co-financées par celles-ci doivent être publiées sous licence libre gratuitement ou pour un coût d'accès marginal. Ce passage sous licence libre devra aussi s'appliquer pour toutes les oeuvres divulguées par les administrations. Les données confidentielles ou critiques pour la sécurité publique sont les seules à ne pas devoir être publiées.
\end{mesure}

L'État a tout intérêt à travailler en co-création avec ses citoyens plutôt que contre eux. Le service de l'État Étalab a déjà publié une licence libre aux termes définis ci-dessus, appelée simplement \textit{Licence ouverte}. Cette licence pourrait être la licence standard des publications des bases de données crées par l'administration, sauf précisions spécifiques. 

Les œuvres qui relèvent du code de la propriété intellectuelle pourront être placées sous des licences libres comme la \textit{Creative Commons Paternité} ou \textit{Art libre} par exemple. La \textit{Creative Commons Paternité}, plus communément désignée CC-BY, n'impose comme condition à la réutilisation que la reconnaissance de la paternité, tandis que la licence \textit{Art libre} exige en plus que les œuvres dérivées soient publiées sous la même licence.

\subsection{La sécurité des citoyens comme limite}
Si nous insistons sur la nécessité de publier le plus grand nombre de données possibles, c'est parce qu'il n'est pas possible à l'État de prévoir quelles données seront utiles ou non. Bien souvent, c'est justement en reliant plusieurs bases de données individuellement anodines que l'on extrait des informations utiles. L'État ne peut pragmatiquement pas imaginer tous les liens qui peuvent être faits avec les données publiées. De plus, imposer que \emph{toutes} les données doivent être publiées interdit de publier des bases de données volontairement incomplètes et oblige à publier les données sources. 

La révision de la directive européenne sur les informations publiques a étendu le droit à la réutilisation dont bénéficie les citoyens européens, mais elle ne va toujours pas jusqu'à imposer aux administrations une obligation de publier en ligne les données qu'elles produisent, ce qui limite fortement la dynamique de l'ouverture des données.


La seule limite à cette logique d'ouverture de l'État doit être la protection de la vie privée ou de la sécurité des citoyens. Dans le cas de base de données brutes il faut prendre en compte la possibilité d'identifier les citoyens même si les données sont anonymisées. Par exemple, si la publication de données médicales comporte toujours le code postal et la date de naissance exacte des patients et permet à des assureurs de retrouver facilement quels clients sont affectés par des maladies graves, alors cette publication doit être être interdite en l'état. Le niveau de granularité de la base de données doit être augmenté de sorte que cibler les individus ne soit plus possible.

\subsection{Libérer les travaux de recherche}

La libération des œuvres et données produites par les services de l’État concernerait en particulier les chercheurs et instaurerait une obligation nationale de publication en accès libre des travaux scientifiques financés par le biais de l'argent public. Une telle obligation existe déjà au niveau européen dans le cadre du programme OpenAIRE par exemple et plusieurs Etats européens étudient en ce moment la possibilité de mettre en place des politiques de libre accès au niveau national.

A minima, l'obligation faite aux chercheurs produisant des travaux sur financement public devrait porter sur le versement de leurs articles dans des dépôts librement accessibles en ligne. Les chercheurs pourraient continuer à publier dans des revues commerciales, à la condition que les cessions de droits consenties aux éditeurs ne les empêchent pas de déposer par ailleurs leurs travaux en libre accès. Au-delà, il importe que les États soutiennent le développement de revues en accès libre, en leur assurant des moyens suffisants pour garantir leur pérennité sans avoir à reporter l'intégralité de leurs coûts sur les auteurs d'articles. 

Des sites de dépôts en accès libre gérés par des universités ou des centres de recherche existent déjà en France, ainsi que des initiatives de revues en accès libre. Il est donc possible de s’appuyer sur l’existant pour poursuivre le mouvement.


 \section{Rendre accessibles les données libérées}
Libérer les données n'est pas suffisant si celles-ci ne sont pas facilement accessibles parce que leur accès est onéreux, parce que le format de diffusion n'est pas utilisable informatiquement ou est non-standard et nécessite de payer des licences logicielles élevées pour être lu.

\begin{mesure}[Conditions d'accessibilité des données libérées]
Les données produites et publiées par l'administration sous licence libre doivent respecter les contraintes suivantes~:

\begin{itemize}
\item Entières~: Les bases de données sont intégralement publiées.
\item Brutes~: Leur format est directement utilisable par un ordinateur.
\item Documentées~: Elles sont accompagnées de leurs métadonnées dans un format documenté.
\item Interopérables~: La documentation du format de fichier est aisément accessible et complète. 
\item Actuelles~: Elles sont les plus récentes possibles.
\item Permanentes~: Leurs adresses d'accès sont durables.
\item Gratuites ou peu coûteuses~: Le coût d'accès est nul ou marginal.
\end{itemize}
\end{mesure}

Le moyen technique le plus simple de remplir ces contraintes est de mettre en place via Internet des portails de dépôt qui les recensent et les mettent à disposition via des protocoles standards comme ceux utilisés pour afficher les pages web. Plusieurs gouvernements en Europe ont lancé de tels portails de diffusion des informations, comme le site data.gouv.fr en France. Mais les ressources humaines, juridiques ou informatiques dont disposent ces portails doivent être renforcées pour améliorer la qualité de service.

