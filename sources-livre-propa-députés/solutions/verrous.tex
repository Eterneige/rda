\chapter{Bannir les verrous numériques}\label{verrous}

Les MTP ou Mesures Techniques de
Protection, plus
connues sous le sigle anglais DRM pour Digital
Rights Management, visent à restreindre les usages possibles des consommateurs d’œuvres «achetées»
légalement et sur lesquelles ils devraient donc pouvoir exercer tous leurs droits. 

Nous proposons deux mesures les concernant. La première n'est réalisable qu'à long terme et la deuxième peut être immédiatement débattue et votée par le Parlement.

\begin{mesure}[Interdire les mesures de protection technique]
 Il devrait être systématiquement légal de passer outre les MTP et nous devrions bannir les MTP qui
empêchent des usages légaux. Les grandes multinationales ne devraient pas avoir le droit d’écrire
leurs propres lois d’utilisation des fichiers.
\end{mesure}

La législation doit bénéficier à la société tout entière, y compris les consommateurs. En même temps avoir le droit
de faire quelque chose selon la loi n’a que peu de valeur en soi si vous n’avez pas les moyens
pratiques de le faire.

Il n’y a aucun intérêt à ce que nos parlements introduisent une législation équilibrée et
raisonnable sur le droit d’auteur si en même temps nous permettons aux multinationales d’écrire
leurs propres lois et d’imposer leur respect par des moyens techniques.

Dans son livre \livre{Culture Libre}, le professeur de droit Lawrence Lessig donne l’exemple d’un livre
numérique publié par Adobe. Le livre était \livre{Alice au Pays des Merveilles}. Il a été publié la première
fois en 1865 et les droits patrimoniaux qui le protégeaient ont expiré depuis longtemps. Puisqu’il n’est plus protégé,
chacun devrait pouvoir faire ce qu’il veut du texte de Lewis Carroll.

Mais un éditeur auquel il a acheté le texte a décidé de régler les verrous DRM de telle sorte qu'il ne pouvait pas en extraire une copie ni l’imprimer ni le louer et encore moins le donner à un ami. C'est une atteinte manifeste à l'intégrité du domaine public.  

Les aveugles et malvoyants qui ont besoin de convertir les livres numériques dans des formats audio
qui leurs soient accessibles sont souvent entravés par les verrous DRM. Un traité de l'OMPI a récemment consacré au plus haut niveau une exception en faveur des handicapés visuels pour l'accès aux livres protégés et la production d'œuvres adaptées. Cette exception ne pourra devenir effective que si les DRM n'en empêchent pas la mise en œuvre concrète.

Un autre exemple est le zonage régional sur les DVDs qui empêche de regarder des films légalement achetés s'ils sont achetés dans une autre zone que celle où a été acheté le lecteur.

\begin{mesure}[Autorisation de contournement des verrous]
Si des verrous existent, ils doivent pouvoir être légalement cassés s'ils empêchent la jouissance pleine et entière de l'œuvre et de ses usages légaux. 
\end{mesure}

Les verrous DRM ont été juridiquement sanctuarisés par les traités de l’OMPI de 1996 et la directive européenne 2001/29. Toutes les marges de manœuvre doivent être exploitées au niveau européen pour assouplir les dispositions relatives aux DRM, notamment en créant un véritable droit à l'interopérabilité et en consacrant la possibilité de contourner les DRM pour bénéficier de toutes les exceptions au droit d'auteur.  

Il existe déjà un régime de régulation des mesures techniques de protection dans la loi française même s’il est peu et difficilement appliqué pour des questions de lourdeur administrative. Une telle exception existe aussi aux États-Unis via une exemption au Digital Millenium Copyright Act qui permet de contourner les verrous sur des DVD pour produire des œuvres transformatives.


