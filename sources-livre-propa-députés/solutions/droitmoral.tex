\chapter{Conserver le droit à la paternité et au repentir}\label{pater}

\begin{mesure}[Préservation du droit à la paternité]
 «~Rendre à César ce qui est à César~» est une maxime qui met tout le monde d’accord.
\end{mesure}

Dans les faits, les convenances sont souvent
plus strictes sur le sujet que n’importe quelle législation relative au droit d’auteur.

Les scientifiques ou les blogueurs ont tendance à citer leurs sources d’une façon qui fait bien plus que respecter le
minimum légal. Il y a plusieurs raisons à cela. Cela rend votre article plus crédible si vous donnez
les liens vers vos sources afin que vos lecteurs puissent en vérifier l’origine s'ils le
souhaitent. Les personnes que vous citez sont contentes, elles seront donc plus enclines à citer
vos propres articles si l’occasion se présente et votre influence augmentera. 

Le droit d’être reconnu en tant qu’auteur sur Internet n’est pas menacé. Nous proposons donc de
laisser inchangé ce point de la législation du droit d’auteur. Nous proposons de laisser inchangés les deux points suivants aussi~:

\begin{mesure}[Préservation du droit à la divulgation]
Le mode et le moment de la première diffusion ne peuvent être décidés que par les auteurs des œuvres. 
\end{mesure}

Il est important pour des raisons marketing que l'auteur soit maître du calendrier de diffusion de son œuvre. C'est ce que vise à garantir le droit à la divulgation. 

\begin{mesure}[Préservation du droit au repentir]
Outre une censure judiciairement motivée, seul l'auteur peut empêcher la diffusion ou réutilisation commerciales de son œuvre, moyennant compensation des éventuels contrats ainsi rompus. Ce repentir ne concerne que son œuvre originale et non les œuvres dérivées.
\end{mesure}

Il faut aussi que les auteurs puissent rompre les contrats qui les lient aux éditeurs et maisons de production, moyennant une compensation financière pour ces éditeurs et maisons d'édition, d'où le maintien du droit de repentir. Ce retrait ne concerne que la diffusion et la réutilisation \textbf{commerciale} de l'œuvre car ce droit de repentir ne doit pas servir à empêcher le public d'accéder aux œuvres ou de les réutiliser sans but lucratif. Ce serait réitérer la lutte contre le piratage des œuvres pour des motifs cette fois-ci non financiers. 

En revanche, pour permettre l'émergence d'un droit au remix à but non lucratif développé dans le chapitre \ref{remix}, il faut permettre les atteintes à l'intégrité de l'œuvre pour ne permettre à l'auteur de ne porter plainte que lorsque les transformations non-commerciales de ses œuvres endommagent sa réputation. C'est ce que propose la Convention de Berne~:

\begin{quotation}
Indépendamment des droits patrimoniaux d’auteur, et même après la cession des dits droits, l’auteur conserve le droit de revendiquer la paternité de l’œuvre et de s’opposer à toute déformation, mutilation ou autre modification de cette œuvre ou à toute autre atteinte à la même œuvre, \textbf{préjudiciables à son honneur ou à sa réputation}.
\end{quotation}

C'est à travers la jurisprudence que le droit moral est devenu en France aussi étendu qu'aujourd'hui, pas à travers les conventions internationales. Le législateur pourrait rectifier le droit en ajoutant un article L-210 au Code de la propriété intellectuelle~:

\begin{mesure}[Restriction du droit à l'intégrité de l'œuvre]
L’auteur jouit du droit au respect de l’intégrité de son oeuvre. Il peut s’opposer à toute déformation, mutilation ou autre modification de cette œuvre, dans la mesure où elles sont préjudiciables à son honneur ou à sa réputation.
\end{mesure}

