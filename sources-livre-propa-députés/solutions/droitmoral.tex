\chapter{Conserver ce qui est utile du droit moral}\label{pater}

\begin{mesure}[Préservation du droit à la paternité]
 «~Rendre à César ce qui est à César~» est une maxime qui met tout le monde d’accord.
\end{mesure}

Dans les faits, les convenances sont souvent
plus strictes sur le sujet que n’importe quelle législation relative au droit d’auteur.

Les scientifiques ou les blogueurs ont tendance à citer leurs sources d’une façon qui fait bien plus que respecter le
minimum légal. Il y a plusieurs raisons à cela. Cela rend votre article plus crédible si vous donnez
les liens vers vos sources afin que vos lecteurs puissent en vérifier l’origine s'ils le
souhaitent. Les personnes que vous citez sont contentes, elles seront donc plus enclines à citer
vos propres articles si l’occasion se présente et votre influence augmentera. 

Le droit d’être reconnu en tant qu’auteur sur Internet n’est pas menacé. Nous proposons donc de
laisser inchangé ce point de la législation du droit d’auteur. Nous proposons également de laisser inchangé le point suivant~:

\begin{mesure}[Préservation du droit à la divulgation]
Le mode et le moment de la première diffusion ne peuvent être décidés que par les auteurs des œuvres. Une personne qui ne souhaite pas que sa création soit portée à la connaissance du public ne doit pas être contrainte à le faire.  
\end{mesure}

Il est également important pour des raisons commerciales que l’auteur et les intermédiaires auxquels il peut éventuellement faire appel soient maîtres du calendrier de diffusion de son œuvre. C’est ce que vise à garantir le droit à la divulgation.

En revanche, pour permettre l'émergence d'un droit au remix à but non lucratif développé dans le chapitre \ref{remix}, il faut permettre la modification et l'adaptation des œuvres pour ne permettre à l'auteur de ne porter plainte que lorsque les transformations non commerciales de ses œuvres nuisent à sa réputation. C'est ce que propose déjà la Convention de Berne~:

\begin{quotation}
Indépendamment des droits patrimoniaux d’auteur, et même après la cession des dits droits, l’auteur conserve le droit de revendiquer la paternité de l’œuvre et de s’opposer à toute déformation, mutilation ou autre modification de cette œuvre ou à toute autre atteinte à la même œuvre, \textbf{préjudiciables à son honneur ou à sa réputation}.
\end{quotation}

C’est à travers la jurisprudence que le droit moral est devenu en France aussi étendu qu’aujourd’hui,en consacrant un droit quasi absolu au respect de l'intégrité des œuvres au bénéfice de l'auteur, mais il n'en est pas ainsi dans les conventions internationales.

La directive européenne 2001/29 sur le droit d'auteur dans la solution pourrait être modifiée dans le sens de la Convention de Berne et le législateur français pourrait également intervenir en ajoutant un article L-210 au Code de la propriété intellectuelle :

\begin{mesure}[Restriction du droit à l'intégrité de l'œuvre]
L’auteur jouit du droit au respect de l’intégrité de son œuvre. Il peut s’opposer à toute déformation, mutilation ou autre modification de cette œuvre, dans la mesure où elles sont préjudiciables à son honneur ou à sa réputation.
\end{mesure}

