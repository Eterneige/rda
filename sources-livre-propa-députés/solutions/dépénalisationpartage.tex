\chapter{Légalisation du partage non-marchand}\label{depen}

Jusqu’il y a 20 ans, le droit d’auteur concernait à peine le commun des mortels. Les réglementations visaient les acteurs commerciaux, comme les labels, les chaînes de télévision ou les maisons d’édition.

Les citoyens qui voulaient copier un poème et l’envoyer à un de leur proche ou enregistrer une chanson
sur une cassette et la donner à un ami n’avaient pas à s’inquiéter des poursuites judiciaires.

Mais le droit d'auteur n'a pas évolué depuis et impose de graves restrictions dans
la vie quotidienne des individus. Alors que la technologie a rendu le partage de plus en plus
simple, la législation protégeant le droit d'auteur a évolué dans le sens inverse, vers une criminalisation croissante de ce
partage.

\begin{mesure}[Dépénalisation du partage sans but lucratif]
 Nous voulons que le droit d’auteur redevienne ce pourquoi il a été conçu et rendre clair qu’il ne
doit réguler que les échanges commerciaux. Publier un travail protégé sans but lucratif
ne devrait jamais être interdit. La persistance du piratage d'œuvres protégées sans but lucratif est une bonne raison pour cette
légalisation.
\end{mesure}

Contrairement à ce qu'affirment les représentants des industries culturelles, une telle réforme est possible, dans le cadre des traités internationaux tels que la Convention de Berne ou ceux de l’Organisation Mondiale de la Propriété Intellectuelle.

Le partage des oeuvres ne constituent pas un préjudice qui devrait faire l'objet d'une compensation au profit des titulaires de droits. L'usage d'une oeuvre ne la déprécie pas, mais au contraire augmente toujours sa valeur. De nouvelles pistes pour le financement de la création peuvent et doivent être explorées, notamment parce que le nombre des créateurs augmentent considérablement à mesure que le numérique met en capacité de créer un plus grand nombre de citoyens. La légalisation du partage ne doit pas s'accompagner de systèmes de compensation à l'image de la copie privée. 

Pour éviter tous risques de dérives, il convient au maximum d'éviter d'importer des notions liées au droit d'auteur dans la consécration des échanges non-marchands. Notamment il ne paraît pas opportun de recourir à de nouvelles exceptions ou à des systèmes de gestion collective obligatoire, qui impliqueront ensuite nécessairement une logique compensatoire de rémunération. 

La solution la plus efficace pour légaliser les échanges non-marchands consiste sans doute à étendre la notion d'épuisement du droit d'auteur, déjà consacrée au niveau européen, en la rendant applicable dans l'environnement numérique, mais seulement aux échanges effectués sans but lucratif. Une telle réforme peut être opérée en révisant la directive 2001/29 sur le droit d'auteur dans la société de l'information.  

\section{Il faut s'adapter au sens de l'histoire}

Une telle réforme est~:

\begin{itemize}
 \item \emph{inéluctable}
 
On peut penser que ce serait une bonne chose si tous les échanges illégaux de fichiers
disparaissaient. Mais ça
ne change rien à la réalité. 

La limitation du partage de fichiers par les lois et la répression ne
fonctionnent pas. Le partage de fichier est là pour durer, avec notre accord ou sans.
\item \emph{indispensable}

Les tentatives d’imposer
l’interdiction du partage de fichiers mettent en danger les droits fondamentaux.

Cela serait une solution inacceptable même si la répression fonctionnait ou si le secteur de la culture était réellement en train de mourir. Ni l'un ni l'autre ne sont vrais.
\item \emph{inoffensive}

Les artistes et le secteur de la
culture se portent bien malgré le partage de fichiers (ou peut être grâce à
lui), il n’y a donc pas de réel problème à résoudre.

\item \emph{facile à mettre en place}

La raison est très simple. «~Suivre l’argent~» suffit aux autorités pour leur permettre de garder une trace des activités
commerciales.

Si un entrepreneur souhaite gagner de l’argent, la première des choses qu’il doit faire, c’est faire connaître au plus grand nombre possible ce qu’il a à proposer. Mais s'il propose quelque
chose d’illégal, cela arrivera aux oreilles de la police avant qu’il ait eu le temps d’attirer une
clientèle importante.

Aucune restriction supplémentaire des droits fondamentaux n’est nécessaire. Les systèmes de contrôle
déjà en place pour d’autres raisons suffisent pour garder une trace des activités commerciales.

\end{itemize}

\section{La différence entre le commercial et le non-commercial}
La distinction entre la sphère des échanges non-marchands et la sphère marchande n'est pas facile à tracer, notamment sur Internet où interviennent de multiples intermédiaires lors des échanges (fournisseurs d'accès à Internet, plateformes de stockage ou de partage, moteurs de recherche).  
 
Il existe déjà un arsenal juridique développé par les tribunaux, pour distinguer les entreprises à but lucratif de celles qui ne poursuivent pas de telles fins. Il existe aussi plusieurs licences basées sur le droit d’auteur, comme les licences Creative Commons Attribution Non Commerciale, qui s’appuient sur cette distinction.

Néanmoins, afin de garantir un maximum de sécurité juridique, il convient de définir de manière stricte la notion d'échange non-marchand en la restreignant aux formes de partage décentralisées entre individus, n'impliquant pas le recours à des plateformes centralisées de stockage des fichiers. Par ailleurs, le bénéfice de la légalisation des échanges non-marchands ne pourrait être invoqué que si aucun usage commercial des oeuvres n'est effectué, y compris des usages indirects de type recettes publicitaires.  

En revanche, le fait de référencer et de signaler des liens vers des sites mettant à disposition des oeuvres doit rester libre. Cela permettra la mise en place d'annuaires de liens permettant l'accès aux oeuvres pour les internautes. Dans l'esprit de notre mesure, ces annuaires ne pourront être gérés que par des associations à but non lucratif qui ne dépendent pas d'une activité commerciale pour se financer.

De telles propositions favoriseraient des formes d'échanges décentralisés, utilisant l'architecture du réseau Internet en conformité avec sa nature et elles empêcheraient la reconstitution d’une industrie rentable du piratage comme Megaupload en avait donné l'exemple.

Tout comme actuellement, les entreprises ou associations qui proposeraient au téléchargement des fichiers protégés par le droit d'auteur et en généreraient un bénéfice substantiel resteraient obligées de reverser une partie de leurs bénéfices aux ayants-droits. 

\section{Une exception facilitant les usages pédagogiques et de recherche}

Prescripteurs de culture, les enseignants jouent un rôle fondamental en matière de sensibilisation à la création
culturelle et artistique. Avec leurs élèves, ils utilisent de plus en plus souvent des ordinateurs et Internet dans le cadre de leurs cours pour diffuser des œuvres protégées par le droit d'auteur. Cependant l'exception pédagogique actuelle repose sur des accords sectoriels complexes et prête à confusion. L’enchevêtrement de dispositions spécifiques conduit les enseignants à se situer aux marges du droit de la propriété littéraire et artistique.

Les chercheurs qui doivent fréquemment utiliser des oeuvres protégées dans le cadre de leurs travaux ont les mêmes soucis. Des usages innovants de recherche comme le datamining (fouille de bases de données) peuvent nécessiter d'utiliser à grande échelle des oeuvres protégées.  

La complexité des règles, source de lourdeur bureaucratique et d’insécurité juridique, est d’autant moins compréhensible que les enjeux financiers en cause sont limités.

La dépénalisation du partage non-marchand aurait comme conséquence immédiate de permettre aux enseignants et professeurs de l'Éducation nationale et du monde de la recherche de pouvoir librement utiliser et diffuser des travaux protégés par le droit d'auteur à leurs élèves ou collègues. En revanche les entreprises d'enseignement privé sans contrat avec l'État ne seraient protégées que par l'exception pédagogique actuelle. 

L'obligation que nous préconisons de placer sous licence libre et de publier les travaux des agents publics aurait aussi pour effet de favoriser le développement des ressources pédagogiques libres.

Une exception spécifique introduite au niveau européen et couvrant l'ensemble des usages d'oeuvres protégées effectués sans but commercial au sein des établissements d'enseignement et de recherche constituerait toutefois une garantie bien meilleure. S'agissant d'usages aussi légitimes que l'éducation et de la recherche, il importe que cette exception ne fasse pas l'objet d'une compensation financière. C'est déjà le cas au Canada depuis une réforme opérée en 2012 et cela devrait aussi l'être en Europe et en France.  


