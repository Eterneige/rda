\chapter{Dépénalisation du partage sans but lucratif}\label{depen}

Jusqu’il y a 20 ans, le droit d’auteur concernait à peine le commun des mortels. Les régulations
visaient les acteurs commerciaux, comme les labels ou maisons d’édition.

Les citoyens qui voulaient copier un poème et l’envoyer à leur amoureux ou enregistrer une chanson
sur une cassette et la donner à un ami n’avaient pas à s’inquiéter des poursuites judiciaires.

Mais le droit d'auteur n'a pas évolué depuis et impose de graves restrictions sur
la vie quotidienne des individus. Alors que la technologie a rendu le partage de plus en plus
simple, la législation protégeant le droit d'auteur a évolué dans le sens inverse, vers une criminalisation croissante de ce
partage.

\begin{mesure}[Dépénalisation du partage sans but lucratif]
 Nous voulons que le droit d’auteur redevienne ce pourquoi il a été conçu et rendre clair qu’il ne
doit réguler que les échanges commerciaux. Publier un travail protégé sans but lucratif
ne devrait jamais être interdit. La persistance du piratage d'œuvres protégées sans but lucratif est une bonne raison pour cette
légalisation.
\end{mesure}

Cette mesure peut être ajoutée comme une restriction dans la législation
relative au droit d’auteur, en conformité avec les traités internationaux tels que la convention de
Berne ou celui de
l’Organisation Mondiale de la Propriété
Intellectuelle.

\section{Il faut s'adapter au sens de l'histoire}

Une telle réforme est~:

\begin{itemize}
 \item \emph{inéluctable}
 
On peut penser que ce serait une bonne chose si tous les échanges illégaux de fichiers
disparaissaient. Mais ça
ne change rien à la réalité. 

La limitation du partage de fichiers par les lois et la répression ne
fonctionnent pas. Le partage de fichier est là pour durer, avec notre accord ou sans.
\item \emph{indispensable}

Les tentatives d’imposer
l’interdiction du partage de fichiers mettent en danger les droits fondamentaux.

Cela serait une solution inacceptable même si ça fonctionnait ou si le secteur de la culture était réellement en train de mourir. Ni l'un ni l'autre ne sont vrais.
\item \emph{inoffensive}

Les artistes et le secteur de la
culture se portent bien malgré le partage de fichiers (ou peut être grâce à
lui), il n’y a donc pas de réel problème à résoudre.

\item \emph{facile à mettre en place}

La raison est très simple. «~Suivre l’argent~» suffit aux autorités pour leur permettre de garder une trace des activités
commerciales.

Si un entrepreneur souhaite gagner de l’argent, la première des choses qu’il doit faire, c’est faire connaître au plus grand nombre possible ce qu’il a à proposer. Mais s'il propose quelque
chose d’illégal, cela arrivera aux oreilles de la police avant qu’il ait eu le temps d’attirer une
clientèle importante.

Aucune restriction supplémentaire des droits fondamentaux n’est nécessaire. Les systèmes de contrôle
déjà en place pour d’autres raisons suffisent pour garder une trace des activités commerciales.

\end{itemize}

\section{La différence entre le lucratif et le non-lucratif}
Il est vrai qu’il y a une zone d’ombre entre les activités à but lucratif et celles qui ne le sont pas, mais
c’est un problème que les tribunaux ont déjà résolu à de nombreuses reprises dans des domaines
différents. Nous possédons déjà un arsenal juridique qui fait la distinction entre intention commerciale et non
commerciale, incluant la législation sur le droit d’auteur telle qu’elle existe aujourd’hui.

Il y a aussi plusieurs licences basées sur le droit d’auteur, y compris les licences Creative Commons
Attribution Non Commerciale qui s'appuyent sur cette distinction.

De façon générale, la limite est grossièrement à
l’endroit où l'on s'y attend. Si en tant que personne privée vous possédez un blogue sans
aucune publicité, c’est à but non lucratif. Si vous percevez quelques euros par mois de Google Ads qui servent à financer votre hébergement, votre blogue est toujours à but non lucratif. Mais si c’est un blogue qui génère des revenus substantiels de la publicité et que ces revenus vous servent à vivre, il franchit sûrement la ligne et devient commercial. 

Tout comme actuellement, les entreprises ou associations qui proposeraient au téléchargement des fichiers protégés par le droit d'auteur et en généreraient un bénéfice substantiel resteraient obligées de reverser une partie de leurs bénéfices aux ayants-droits. 

Il est possible que des petits montants pourront être générés illégalement par des particuliers en échappant au contrôle de l'administration fiscale. Ceux-ci passeront peut-être à travers les mailles du filet tout comme le piratage continue à exister malgré la répression. Cependant cela n'est pas essentiel, car notre proposition vise à empêcher la constitution d'une industrie profitable du piratage comme Megaupload.

Il est aussi possible que des associations à but non lucratif se créent pour favoriser l'échange de fichiers protégés par le droit d'auteur et génèrent ainsi des montants conséquents. Tant que ces associations ne redistribuent pas ces montants à leurs membres pour que ceux-ci en vivent mais payent simplement leurs frais de fonctionnement avec, nous n'y voyons pas d'inconvénient. 

\section{Une exception pédagogique de facto pour l'Éducation nationale}

Prescripteurs de culture, les enseignants jouent un rôle fondamental en matière de sensibilisation à la création
culturelle et artistique. Avec leurs élèves, ils utilisent de plus en plus souvent des ordinateurs et Internet dans le cadre de leurs cours pour diffuser des œuvres protégées par le droit d'auteur. Cependant l'exception pédagogique actuelle repose sur des accords sectoriels complexes et prête à confusion. L’enchevêtrement de dispositions spécifiques conduit les enseignants à se situer aux marges du droit de la propriété littéraire et artistique.

La complexité des règles, source de lourdeur bureaucratique et d’insécurité juridique, est d’autant moins compréhensible que les enjeux financiers en cause sont limités.

La dépénalisation du partage non-marchand a comme conséquence immédiate de permettre aux enseignants et professeurs de l'Éducation nationale et du monde de la recherche de pouvoir librement utiliser et diffuser des travaux protégés par le droit d'auteur à leurs élèves ou collègues. En revanche les entreprises d'enseignement privé sans contrat avec l'État ne seraient protégés que par l'exception pédagogique actuelle.

