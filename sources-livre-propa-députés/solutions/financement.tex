\chapter{Remodeler le financement de la culture}\label{financement}

\section{Trois sources d'investissement principales}
Les investissements privés, les subventions publiques et le mécénat individuel ou collectif sont les principales sources de financement de la culture.

Les investissements privés comme ceux opérés par les maisons d'édition ou les studios d'enregistrement doivent continuer. C'est pourquoi nous défendons au chapitre \ref{depen} qu'il ne faut pas dépénaliser la diffusion à but lucratif. Il faut permettre aux modèles économiques existants de perdurer.

Le rôle des subventions publiques dans la création aujourd'hui est essentiel. L'État, les régions et les collectivités locales investissent dans la culture car ils considèrent qu'il est du ressort des institutions publiques de soutenir l'offre culturelle à destination des citoyens. Nous soutenons pleinement cette idée. Nous considérons de plus qu'il faut que ces institutions exigent en retour de leurs partenaires que les contenus ainsi produits soient plus facilement et rapidement diffusables et réutilisables par les citoyens qui les ont financés. C'est le sens des propositions des chapitres \ref{licencelibre} et \ref{dompub}. 

Le mécénat, qu'il soit individuel ou collectif, doit surtout être réglementé par le droit des contrats pour éviter les fraudes fiscales ou les arnaques au don. Ainsi, les plates-formes de financement participatif doivent être encouragées et non freinées par des incertitudes juridiques sur le statut fiscal des dons ou investissements. Cela peut être fait en leur accordant des statuts particuliers auprès des autorités de régulation de contrôle prudentiel ou de régulation des marchés. Par ailleurs la loi doit prendre en compte ces nouveaux types de financements et ne pas honorer les ventes ou investissements reçus via ces plates-formes doit pouvoir être facilement sanctionné. De telles mesures ne sont cependant pas centrales pour le financement de la création. Le financement participatif se développe rapidement même en l'absence d'un cadre juridique plus précis.

\section{Les sociétés de gestion des droits}

La logique des sociétés de gestion relève de l'investissement privé. Un artiste décide individuellement de déléguer la gestion de ses droits à une entité qui lui reverse une partie des droits perçus au prorata de la diffusion et de la réutilisation de ses œuvres. L'utilité de telles entités est certaine car elle simplifie la collecte des droits. 

Avec les réformes que nous proposons, leur rôle est amené à s'élargir. Il serait logique qu'elles veillent au respect des licences libres amenées à se développer selon le chapitre \ref{licencelibre}, qu'elles perçoivent au nom de leurs sociétaires les revenus issus de la commercialisation des œuvres dérivées des leurs comme proposé au chapitre \ref{remix} et qu'elles participent au maintien du registre défendu au chapitre \ref{registre}. De plus, dans le cas où l'État déciderait de financer la culture via le prélèvement d'une redevance qui s'ajouterait à un abonnement à Internet, les sociétés de gestion pourraient aider à répartir une partie des revenus issus de cette redevance.

Cependant, les sociétés de gestion sont actuellement entachées par plusieurs soucis majeurs, dont:

\begin{itemize}
\item Les données de répartition ne sont pas publiées par les sociétés de gestion.
\item Elles s'arrogent trop facilement des droits de perception sur des œuvres même quand elles n'en détiennent pas.
\item Elles se basent parfois sur la possibilité de diffusion des œuvres au public pour établir une redevance et non sur la diffusion concrète.
\item Elles ne sont pas démocratiques en raison de votes censitaires ou d'organisations collégiales.
\item Elles monopolisent la gestion des droits des artistes qui y sont inscrits même quand ceux-ci voudraient s'y opposer pour une partie de leurs œuvres ou voudraient accorder des dérogations.
\end{itemize}

Ces travers dérivent à notre avis d'obligations légales de transparence trop faibles ainsi que d'un manque de concurrence entre les sociétés de gestions qui conduit à des abus de type monopolistique. Chaque société de gestion monopolise l'exploitation des droits patrimoniaux dans un domaine spécifique de la vie culturelle comme le cinéma ou la musique. 

\begin{mesure}[Sociétés de gestion]
Nous proposons de~:

\begin{description}
\item soit
	\begin{itemize}
	\item ouvrir au marché la redistribution des droits patrimoniaux des auteurs
	\item imposer aux sociétés de gestion la transparence des critères de répartition et des statistiques de répartition selon ces critères.
	\item sanctionner plus sévèrement les perceptions abusives
	\end{itemize}

\item soit 
	\begin{itemize}
	\item imposer la règle «~un sociétaire pour un vote~»
	\item laisser aux sociétaires le choix de décider œuvre par œuvre des règles d'inclusion dans le catalogue des sociétés de gestion
	\item imposer aux sociétés de gestion la transparence des critères de répartition et des statistiques de répartition selon ces critères.
	\item sanctionner plus sévèrement les perceptions abusives
	\end{itemize}
\end{description}
\end{mesure}

Les sociétés de gestion devraient dans tous les cas continuer à être agréées par l'État car leurs agents doivent être assermentés pour pouvoir constater la matérialité des infractions au droit d'auteur. L'existence d'un registre comme proposé au chapitre \ref{registre} doit permettre de limiter les perceptions abusives.

Dans le cas de la libéralisation du marché de la perception des droits d'auteurs, un éclatement complet du marché est impossible car une société de gestion ne peut correctement faire respecter les droits de ses sociétaires que lorsqu'elle obtient une taille suffisante. En revanche la scission des sociétés de gestion actuelles en plusieurs sociétés moins prônes aux abus et plus à l'écoute de leurs sociétaires et des besoins du public serait le résultat souhaité de la libéralisation.

