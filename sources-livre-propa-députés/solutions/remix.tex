\chapter{Créer un droit à la transformation des œuvres}\label{remix}

La législation très restrictive d’aujourd’hui est un obstacle majeur pour les créateurs qui veulent embrasser toutes les possibilités offertes par le numérique.

\begin{mesure}[Valoriser les œuvres transformatives]
Nous voulons que soient reconnus trois droits à la création:

\begin{itemize}
\item Le droit de réutiliser toute œuvre dans une autre et de publier le résultat. 
\item Le droit d’utiliser toute œuvre existante pour la modifier et de publier ces modifications. 
\item Le droit d’utiliser lucrativement ces œuvres dérivées en échange d’un paiement équilibré aux auteurs originaux. 
\end{itemize}
\end{mesure}

Le premier droit concerne par exemple la réutilisation d'une musique dans une vidéo amateur, le deuxième l'utilisation des images d'un film pour en faire une fausse bande-annonce et le troisième la vente de mashups sur iTunes.

Dans la création originelle classique, l’ancien disparaissait dans le nouveau. La culture du remix contemporaine se différencie en ce que l’ancien reste visible derrière le nouveau. Le remix est une nouvelle copie créative que l’on reconnaît comme telle. Dans cette mesure, puisque la copie créative fait partie du quotidien de nos interactions à travers toutes les couches sociales, la reconnaissance d’un droit au remix est une condition essentielle pour la liberté de création et d'expression de notre société.

Il faut protéger la culture du remix car elle nous fait passer d’une culture de la consommation passive à une culture de la création massive.

À l'heure actuelle les majors revendiquent
la propriété sur des sons individuels et de très courts extraits. Si vous êtes un musicien hip-hop,
attendez vous à payer des centaines de milliers d’euros par avance pour avoir le droit d’utiliser
des samples si vous souhaitez toujours rendre votre musique accessible au public.

C’est clairement une restriction du droit de créer de nouvelles cultures. C'est aussi un frein à une juste rémunération des auteurs. Si un droit au remix était reconnu, les auteurs des œuvres remixées pourraient être rémunérés en touchant une partie des bénéfices que génèrent les remix en échange de leur renonciation à leur droit moral à l'intégrité de leurs œuvres. En revanche, il serait totalement illégitime de rémunérer l'usage transformatif des oeuvres dans un cadre non-commercial, car celui-ci relève fondamentalement de la liberté d'expression. 

Une telle réforme peut être conduite de plusieurs manières et notamment par le biais d'un élargissement de l'exception de citation. C’est ce que recommande le rapport Lescure en France. La Commission européenne avait auparavant proposé d'introduire une nouvelle exception dans son livre vert sur le droit d'auteur dans l'économie de la connaissance publié en 2008. Le Canada a de son côté introduit en 2012 une exception en faveur du remix.  

La directive européenne de 2001 sur le droit d’auteur est suffisamment ouverte, s'agissant du droit de citation, pour s'appliquer aux usages transformatifs. En France les contours de l'exception sont encore beaucoup trop étroits pour s'appliquer aux mashups et au remix. C'est pourquoi l’article 122-5 du Code de la propriété intellectuelle pourrait être modifié de la manière suivante :

\begin{mesure}[Étendre le droit de citation]
Les analyses et citations concernant une oeuvre protégée au sens des articles L. 112-1 et L. 112-2 du présent Code, justifiées par le caractère critique, polémique, pédagogique, scientifique, d’information, \textbf{ou de création sans but lucratif} de l’oeuvre à laquelle elles sont incorporées et effectuées dans la mesure justifiée par le but poursuivi.
\end{mesure}

Nous précisons dans cette mesure qu'elle s'applique à toutes les sortes d'œuvres pour empêcher que la jurisprudence ne la restreigne aux textes comme c'est actuellement le cas en France.


