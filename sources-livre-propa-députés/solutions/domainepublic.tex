\chapter{Préserver et promouvoir le domaine public}\label{dompub}

\section{Le domaine public est un bien commun qui a besoin d'être défendu}
Depuis des siècles, les bibliothèques, les archives et les musées ont été, partout en Europe, les
gardiens de notre riche et diversifié patrimoine culturel. Qu’il s’agisse de sculpture, de peinture,
de musique ou de littérature, ces institutions ont su préserver nos trésors de connaissance, de
beauté et d’imagination et en donner l’accès au plus grand nombre.

La numérisation apporte un souffle nouveau aux œuvres du passé et les transforme non
seulement en une source d’intérêt pour les utilisateurs individuels mais aussi en matériau précieux
pour construire l’économie numérique de demain.

Nous sommes convaincus que la mission de rendre accessible en ligne notre patrimoine culturel
et de le préserver pour les générations futures est avant tout du ressort des
institutions publiques. Il est inconcevable d’abandonner cette responsabilité à un ou plusieurs
acteurs privés, au risque de le(s) voir imposer une forme de contrôle. Mais cela ne signifie pas que
les entreprises privés ne doivent pas s’impliquer en matière de numérisation : bien au contraire,
nous considérons qu’elles ont un rôle et souhaitons qu’elles renforcent leurs investissements dans
le cadre de partenariats équilibrés et profitables.

Or ces partenariats sont rarement équilibrés faute de politique nationale forte. Il est donc essentiel de le consacrer et de le protéger par la loi. On ne peut plus laisser une question aussi essentielle relever du ressort des seuls établissements culturels et des collectivités dont ils dépendent, qui sont souvent mal armés pour aborder la question et engagés dans des logiques de dégagement de ressources propres qui peuvent les pousser à marchandiser le domaine public. Le domaine public doit être le même pour tous les citoyens en France, car derrière cette notion, c’est la liberté fondamentale d’accès à la Culture et le droit de créer à partir des œuvres du passé qui sont en jeu. Le défendre impliquera donc la création de nouvelles peines pour infraction aux règles du domaine public.

\section{Quatre mesures pour que rien ne puisse être soustrait au domaine public}
\begin{mesure}[Inscription du domaine public dans la loi]
Inscrire le domaine public dans la loi. Au terme du délai de validité du droit d'auteur, l'œuvre est réputée appartenir au domaine public.
\end{mesure}

Cette précision permet de faire référence par la suite à des règles particulières attachées aux œuvres dans le domaine public pour le protéger.

\begin{mesure}[Protection du domaine public numérisé]
Les reproductions fidèles d’œuvres appartenant au domaine public doivent aussi être dans le domaine public.
\end{mesure}

Ce point permet d'interdire les pratiques des très nombreux musées, bibliothèques et services d’archives en France qui estiment qu’ils bénéficient d’un droit d’auteur sur les reproductions numériques d’œuvres élevées dans le domaine public. Ce principe a déjà été consacré dans la jurisprudence aux États-Unis à l’occasion de la décision Bridgeman Art library v. Corel Corp. La mesure s'appliquera principalement aux œuvres numérisées ou intégrées à des bases de données.

\begin{mesure}[Le domaine public n'appartient pas à l'État]
Exclure que les œuvres du domaine public puissent être considérées comme des informations appartenant à l'État.
\end{mesure}

Plusieurs institutions culturelles considèrent qu’en numérisant des œuvres du domaine public, elles produisent des données (des suites de 0 et de 1) relevant du champ d’application de la loi du 17 juillet 1978 sur les informations publiques.

Cette interprétation a un effet redoutable, car cette loi de 1978 , si elle n’autorise pas en principe les administrations à s’opposer à la réutilisation des informations, leur permet de la soumettre au paiement d’une redevance, notamment pour les usages commerciaux (exemple). La loi de 1978 permet d'installer un système de domaine public payant. De plus, les institutions culturelles bénéficient d’un régime dérogatoire complexe, dit exception culturelle, qui leur donne une plus grande marge de manœuvre pour poser des restrictions à la réutilisation.

\begin{mesure}[Élévation volontaire dans le domaine public]
Permettre aux auteurs de renoncer à tous leurs droits sur leurs œuvres pour enrichir le domaine public.
\end{mesure}

On devrait permettre aux auteurs qui le souhaitent de verser par anticipation leurs œuvres dans le domaine public. La renonciation inclurait les droits patrimoniaux comme moraux. 

Enfin les œuvres créées par les agents publics dans le cadre de leurs missions de service pourraient être versées automatiquement dans le domaine public, comme c’est le cas actuellement aux États-Unis pour les œuvres produites par les agents fédéraux. 







