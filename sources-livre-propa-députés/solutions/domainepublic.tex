\chapter{Préserver et promouvoir le domaine public}\label{dompub}

\section{Le domaine public est un bien commun qui a besoin d'être défendu}
Depuis des siècles, les bibliothèques, les archives et les musées ont été, partout en Europe, les
gardiens de notre riche et diversifié patrimoine culturel. Qu’il s’agisse de sculpture, de peinture,
de musique ou de littérature, ces institutions ont su préserver nos trésors de connaissance, de
beauté et d’imagination et en donner l’accès au plus grand nombre.

La numérisation apporte un souffle nouveau aux œuvres du passé et les transforme non
seulement en une source d’intérêt pour les utilisateurs individuels mais aussi en matériau précieux
pour construire l’économie numérique de demain.

Nous sommes convaincus que la mission de rendre accessible en ligne notre patrimoine culturel
et de le préserver pour les générations futures est avant tout du ressort des
institutions publiques. Il est inconcevable d’abandonner cette responsabilité à un ou plusieurs
acteurs privés, au risque de le(s) voir imposer une forme de contrôle. Mais cela ne signifie pas que
les entreprises privés ne doivent pas s’impliquer en matière de numérisation : bien au contraire,
nous considérons qu’elles ont un rôle et souhaitons qu’elles renforcent leurs investissements dans
le cadre de partenariats équilibrés et rentables.

Or ces partenariats sont rarement équilibrés faute de politique nationale forte. Il est donc essentiel de consacrer le domaine public et de le protéger par la loi. On ne peut plus laisser une question aussi essentielle relever du ressort des seuls établissements culturels et des collectivités dont ils dépendent, qui sont souvent mal armés pour aborder la question et engagés dans des logiques de dégagement de ressources propres qui peuvent les pousser à marchandiser le domaine public. Le domaine public doit être le même pour tous les citoyens en France, car derrière cette notion, c’est la liberté fondamentale d’accès à la Culture et le droit de créer à partir des œuvres du passé qui sont en jeu. Le défendre impliquera donc la création de nouvelles peines pour atteinte à l'intégrité du domaine public.

\section{Quatre mesures pour que rien ne puisse être soustrait au domaine public}
\begin{mesure}[Inscription d'une définition positive du domaine public dans la loi]
Inscrire le domaine public dans la loi.Les créations appartiennent en principe au domaine public. Ce n'est que par exception, lorsqu'elles sont mises en forme de manière originale et tant que durent des droits patrimoniaux, qu'elles sont protégées par le droit d'auteur. Au terme du délai de validité du droit d’auteur, l’œuvre est réputée appartenir au domaine public et elle ne doit plus pouvoir en être soustraite.  
\end{mesure}

\begin{mesure}[Protection du domaine public numérisé]
Les reproductions fidèles d’œuvres du domaine public doivent aussi appartenir au domaine public.
\end{mesure}

Ce point permet d’interdire les pratiques des très nombreux musées, bibliothèques et services d’archives qui estiment qu’ils bénéficient d’un droit d’auteur sur les reproductions numériques d’œuvres élevées dans le domaine public. Ce principe a déjà été consacré dans la jurisprudence aux États-Unis à l’occasion de la décision Bridgeman Art library v. Corel Corp pour les reproductions d'œuvres en deux dimensions. La mesure s’appliquera principalement aux œuvres numérisées ou intégrées à des bases de données.

\begin{mesure}[Le domaine public n'appartient pas à l'État]
Exclure que les œuvres du domaine public numérisées puissent être considérées comme des informations appartenant à l'État.
\end{mesure}

Plusieurs institutions culturelles considèrent qu’en numérisant des œuvres du domaine public, elles produisent des données (des suites de 0 et de 1) relevant du champ d’application de la directive européenne sur la réutilisation des informations publiques (PSI) et des différentes lois l'ayant transposée dans les pays de l'Union européenne.

Cette interprétation a un effet redoutable, car la directive européenne , si elle n’autorise pas en principe les administrations à s’opposer à la réutilisation des informations, leur permet de la soumettre au paiement d’une redevance, notamment pour les usages commerciaux. La réglementation permet en définitive d’installer un système de domaine public payant. De plus, les institutions culturelles bénéficient d’un régime dérogatoire complexe, dit exception culturelle, qui leur donne une plus grande marge de manœuvre pour poser des restrictions à la réutilisation.

Ce dernier point a fait l'objet d'une modification dans la nouvelle mouture de la directive PSI adoptée en juin 2013 par le Parlement européen, mais ne nouveau texte ne prend toujours pas en compte la spécificité du domaine public numérisé. C'est pourquoi il importe d'indiquer expliciter que les œuvres numérisées appartenant au domaine public ne doivent pas être considérées comme des informations publiques.

\begin{mesure}[Élévation volontaire dans le domaine public]
Permettre aux auteurs de renoncer à tous leurs droits sur leurs œuvres pour enrichir le domaine public.
\end{mesure}

On devrait permettre aux auteurs qui le souhaitent de verser par anticipation leurs œuvres dans le domaine public. La renonciation inclurait les droits patrimoniaux comme moraux. Cela peut poser problème dans différentes législations européennes et particulièrement en France concernant le droit moral.Mais lorsque la renonciation au droit moral se fait par le biais d'une expression non-ambiguë de la volonté de l'auteur, valable à l'égal de tous, et non d'une cession au profit d'un éditeur par exemple, elle ne devrait pas être empêchée.

Enfin les œuvres créées par les agents publics dans le cadre de leurs fonctions pourraient être versées automatiquement dans le domaine public, comme c’est le cas actuellement aux États-Unis pour les œuvres produites par les agents fédéraux. 







