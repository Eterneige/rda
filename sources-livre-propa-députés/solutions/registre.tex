\chapter{Enregistrer les œuvres tous les cinq ans}\label{registre}

Les œuvres orphelines sont un vrai problème. Bien souvent il est difficile de localiser le
propriétaire d’une œuvre mais celui-ci se manifeste quand l’œuvre dérivée est publiée. De plus la
majorité des œuvres orphelines ont peu ou aucune valeur commerciale mais il est quand même
impossible de les diffuser sans risquer des poursuites.

\begin{mesure}[Création d'un registre des œuvres protégées]
 La protection du droit d’auteur doit être accordée automatiquement dès la publication comme
aujourd’hui, mais si les propriétaires veulent continuer à jouir de l'entièreté de leurs droits après les cinq
premières années de publication, ils doivent se manifester régulièrement de sorte qu’ils soient facilement
trouvables. 

Dans le cas où l'ayant-droit ne se manifeste pas et dans l'attente de son réengistrement, ses travaux sont soit placés sous une licence libre soit versés dans le domaine public. Il récupére ses droits à partir de la date de réenregistrement.

Pour simplifier sa mise en œuvre, cette mesure peut ne s'appliquer qu'aux œuvres à venir et non aux œuvres déjà publiées.
\end{mesure}

\section{Qu'est-ce qu'une œuvre orpheline~?}

Une œuvre orpheline est une œuvre encore protégée par le droit d’auteur mais pour laquelle le
détenteur des droits n’est pas connu ou ne peut être retrouvé. Cela peut être un livre, une chanson,
un film, une photo ou tout autre création qui tombe sous le coup de la législation relative au droit
d’auteur.

Les œuvres orphelines représentent un important problème pour quiconque souhaite les utiliser. Si
vous le faites sans en avoir obtenu la permission, vous courez le risque que le détenteur des droits
s’en souvienne soudainement, vous intente un procès et vous réclame beaucoup d’argent. Comme nous le
savons tous, les tribunaux peuvent être assez enclins à attribuer des réparations même pour des
violations mineures de droits d’auteur et à condamner à des sommes astronomiques. Dans la plupart
des cas, le risque n’est tout simplement pas acceptable.

Puisqu’il n’y a pas de détenteur de droits à qui s’adresser pour demander une licence, vous ne
pouvez rien faire. Peu importe combien vous trouvez important de partager cette œuvre avec le reste
du monde, il n’y a aucun moyen de le faire sans enfreindre la loi et sans vous exposer à un grand
risque financier. Les œuvres orphelines sont de fait bloquées par le droit d’auteur.

\section{Le trou noir du vingtième siècle}

Ce n’est pas un problème marginal. Une grande partie de notre héritage culturel commun du 20\ieme
siècle tombe dans cette catégorie. Environ 75\% des livres que Google souhaite numériser dans le
cadre de leur «~Google Books initiative~» sont épuisés mais toujours sous droit d’auteur.

Même s’il est théoriquement possible de retrouver le détenteur des droits pour beaucoup de ces
livres en entreprenant une investigation pour chaque cas individuel, cela devient en pratique
infaisable lorsque vous voulez numériser en masse.

Google Books n’est pas le seul projet qui numérise des œuvres et les rend disponibles même si
c’est celui qui a attiré le plus d’attention dernièrement. Il y a un projet européen appelé
Europeana avec un objectif similaire, ainsi que
l’initiative ouverte du Projet Gutenberg.
Tous ces
projets sont freinés par le problème des œuvres orphelines.

Si nous n’agissons pas, une grande part de notre héritage culturel commun du 20\ieme siècle risque de
se retrouver perdue avant qu’il ne soit légal de la sauver pour la postérité.

\section{Un registre bénéficierait à tous}

Dans le même temps, l’existence de bases de données publiques, dans lesquelles les détenteurs des
droits en questions peuvent facilement être retrouvés par tous ceux intéressés par l’obtention d’une
licence commerciale d’une œuvre, sera bien évidemment bénéfique aux détenteurs de droits. Si vous
voulez vendre quelque chose, rendre votre identité connue de potentiels acheteurs est bien sûr dans
votre propre intérêt.

Inversement, cette base de donnée permettra de savoir quelles œuvres appartiennent au domaine public ou de connaître les licences sous lesquelles les auteurs veulent placer leurs œuvres. Elle permettra donc aux auteurs de savoir rapidement quelles sont les œuvres qu'ils peuvent réutiliser facilement. Cela facilitera les processus de création.

Dans la dynamique de libération des données recommandée plus bas, cette base de donnée devrait être sous une licence libre et facilement cherchable via un portail sur Internet ou indexable par des moteurs de recherche externes. 

De telles bases de données existent déjà partiellement pour les œuvres du domaine public. Europeana par exemple se veut être un portail d'accès libre au domaine public européen.

\section{Une réponse conforme à la Convention de Berne}
La Convention de Berne interdit de poser des conditions préalables à la jouissance du droit d'auteur mais elle n'interdit pas de mettre sous tutelle la gestion des œuvres au cas où les auteurs sont introuvables en attendant leur réapparition. C'est pourquoi la présente proposition préserve les droits à rémunération des titulaires introuvables en cas de réapparition.

\section{Des accords actuellement défavorables aux auteurs}

Certains accords comme ReLIRE actuellement en vigueur ou en négociation sont défavorables aux auteurs, car les éditeurs souhaitent que l'exploitation des œuvres orphelines soit nécessairement gérée commercialement et ils sont chargés de contacter tous les auteurs pour les avertir de la prochaine perte de leurs droits s'ils ne se signalent pas. 

Or d'une part, si ce sont les éditeurs qui sont chargés de contacter les auteurs, ceux-ci pourraient avoir intérêt à ne pas le faire ou à ralentir le processus pour pouvoir rééditer leurs catalogues sans verser de droits aux auteurs. D'autre part imposer systématiquement des conditions commerciales réduit l'usage de l'œuvre et le bénéfice en notoriété de l'auteur. On ne saurait autrement préjuger de la volonté des auteurs surtout en imposant des modalités de paiement a priori qui limitent les usages des œuvres orphelines alors que l'objectif même de la législation est précisément inverse.

C'est pourquoi permettre aux auteurs de récupérer leurs droits dès qu'ils se signalent voire de poursuivre les éditeurs qui ne les ont pas contacté dans les temps rééquilibre la loi en faveur des auteurs tout en favorisant la réappropriation de leurs œuvres par le public. 





