\chapter{Enregistrer les œuvres tous les cinq ans}\label{registre}

Les œuvres orphelines constituent un vrai problème. Bien souvent il est difficile de localiser le
propriétaire d’une œuvre ce qui empêche de recueillir l'autorisation requise pour son utilisation. Or il est légitime que la charge de la gestion des droits soit également répartie entre auteurs et utilisateurs. De plus la
majorité des œuvres orphelines ont peu ou aucune valeur commerciale mais il est quand même
impossible de les diffuser ou de les archiver sans risquer des poursuites. Une grande part de notre héritage culturel commun du 20\ieme siècle risque de
se retrouver perdue avant qu’il ne soit légal de la sauver pour la postérité.

\begin{mesure}[Création d'un registre des œuvres protégées]
La protection du droit d’auteur doit être accordée automatiquement dès la publication comme
aujourd’hui, mais si les auteurs veulent continuer à jouir de l'entièreté de leurs droits après les cinq
premières années de publication, ils doivent se manifester régulièrement de sorte qu’ils soient facilement
trouvables. Les œuvres qui ne sont plus enregistrées sont élevées dans le domaine public.

Les auteurs peuvent se réenregistrer n'importe quand après les délais échus pour récupérer l'entièreté de leurs droits. Il perdent cependant leurs droits sur toutes les œuvres dérivées des leurs pendant la période où ils n'étaient plus enregistrés. 

Pour simplifier sa mise en œuvre, cette mesure peut ne s'appliquer qu'aux œuvres à venir et non aux œuvres déjà publiées.
\end{mesure}

\section{Qu'est-ce qu'une œuvre orpheline~?}

Une œuvre orpheline est une œuvre encore protégée par le droit d’auteur mais pour laquelle le
détenteur des droits n’est pas connu ou ne peut être retrouvé. Cela peut être un livre, une chanson,
un film, une photo ou tout autre création qui tombe sous le coup de la législation relative au droit
d’auteur.

Les œuvres orphelines représentent un important problème pour quiconque souhaite les utiliser. Si
vous le faites sans en avoir obtenu la permission, vous courez le risque que le détenteur des droits
s’en souvienne soudainement, vous intente un procès et vous réclame beaucoup d’argent. Comme nous le
savons tous, les tribunaux peuvent être assez enclins à attribuer des réparations même pour des
violations mineures de droits d’auteur et à condamner à des sommes astronomiques. Dans la plupart
des cas, le risque n’est tout simplement pas acceptable.

Puisqu’il n’y a pas de détenteur de droits à qui s’adresser pour demander une licence, vous ne
pouvez rien faire. Peu importe combien vous trouvez important de partager cette œuvre avec le reste
du monde, il n’y a aucun moyen de le faire sans enfreindre la loi et sans vous exposer à un grand
risque financier. Les œuvres orphelines sont de fait bloquées par le droit d’auteur.

\section{Le trou noir du vingtième siècle}

Ce n’est pas un problème marginal. Une grande partie de notre héritage culturel commun du 20siècle tombe dans cette catégorie. Environ 75\% des livres que Google souhaite numériser dans le cadre de leur « Google Books initiative » sont épuisés mais toujours protégé par des droits d’auteur. Dans cet ensemble, un nombre important d'ouvrages constituent des œuvres orphelines et le phénomène s'aggrave à mesure que l'on remonte dans le temps.

Même s’il est théoriquement possible de retrouver le détenteur des droits pour beaucoup de ces
livres en entreprenant une investigation pour chaque cas individuel, cela devient en pratique
infaisable lorsque vous voulez numériser en masse.

Google Books n’est pas le seul projet qui numérise des œuvres et les rend disponibles, même si c’est celui qui a attiré le plus d’attention. De nombreuses initiatives associatives ou institutionnelles existent comme Europeana, Internet Archive, le projet Gutenberg ou Wikisource. Tous ces projets sont freinés par le problème des œuvres orphelines.

Une directive européenne a été adoptée en 2012 au niveau de l'Union européenne, qui a introduit une nouvelle exception permettant aux bibliothèques, musées et archives d'utiliser certains types d'œuvres orphelines à la condition d'effectuer des recherches diligentes pour retrouver les titulaires de droits et prévoir une rémunération au cas où ils réapparaitrait. Ce texte est difficile à mettre en œuvre en pratique. 

C'est pourquoi le meilleur moyen d'éviter l'apparition des œuvres orphelines et de mieux répartir socialement les coûts liés à la gestion des droits consiste à prévoir un mécanisme d'enregistrement des œuvres.  

\section{Un registre bénéficierait à tous}

L’existence de bases de données publiques, dans lesquelles les détenteurs des
droits en questions peuvent facilement être retrouvés par tous ceux intéressés par l’obtention d’une
licence commerciale d’une œuvre, sera bien évidemment bénéfique aux détenteurs de droits. Si vous
voulez vendre quelque chose, rendre votre identité connue de potentiels acheteurs est bien sûr dans
votre propre intérêt.

Inversement, cette base de donnée permettra de savoir quelles œuvres appartiennent au domaine public ou de connaître les licences sous lesquelles les auteurs veulent placer leurs œuvres. Elle permettra donc aux auteurs de savoir rapidement quelles sont les œuvres qu'ils peuvent réutiliser. Cela facilitera les processus de création.

Dans la dynamique de libération des données recommandée plus haut, cette base de donnée devrait être placée sous licence ouverte et facilement interrogeable via un portail sur Internet ou indexable par des moteurs de recherche externes.

De telles bases de données existent déjà partiellement pour les œuvres du domaine public, comme Europeana au niveau européen. Mais il s'agit de promouvoir, comme le fait le rapport Lescure en France, la mise en place de bases de données les plus complètes possibles, au niveau des États, et de connecter ces bases entre elles au niveau européen.  

\section{Une réponse conforme à la Convention de Berne}
La Convention de Berne interdit de poser des conditions préalables à la jouissance du droit d’auteur, mais elle n’interdit de mettre en place des formalités ultérieurement, comme en attestent les licences collectives étendues dans les pays scandinaves. Un tel système est conforme à l'intérêt général et ne constitue pas non plus une exception au droit d'auteur.

\section{Des dispositifs actuellement défavorables aux auteurs}

Certains dispositifs comme celui mis en place en place pour la numérisation des livres indisponibles du 20ème siècle sont défavorables aux auteurs, car ils permettent en réalité aux éditeurs de conserver des droits sur des œuvres épuisées qui devraient normalement revenir pleinement aux auteurs. Par ailleurs, ce système ne conçoit que l'exploitation commerciale des œuvres, y compris les œuvres orphelines, pour lesquelles d'autres équilibres peuvent être trouvés.

Imposer systématiquement des conditions commerciales réduit l'usage de l'œuvre et le bénéfice en notoriété de l'auteur. On ne saurait autrement préjuger de la volonté des auteurs surtout en imposant des modalités de paiement a priori qui limitent les usages des œuvres orphelines alors que l'objectif même de la législation est précisément inverse.

C’est pourquoi permettre aux auteurs de récupérer leurs droits dès qu’ils se signalent rééquilibrerait la loi en faveur des auteurs tout en favorisant la réappropriation de leurs œuvres par le public.





