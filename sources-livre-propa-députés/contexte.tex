% rubber: set program lualatex

\RequirePackage[l2tabu, orthodox]{nag} % Avertissements contre les mauvaises pratiques et les packages obsolètes
\documentclass[11pt,liststotoc,bibtotoc]{scrbook}

\usepackage{polyglossia}
	\setdefaultlanguage{french}
	\newcommand{\ieme}{\textsuperscript{e}}
	
\usepackage{luatextra} % Wrapper pour les utilitaires luatex
\usepackage{fontspec} % Permet l'utilisation des polices OpenType avec luatex
	\setmainfont[Ligatures=TeX]{Linux Libertine O}
	\setsansfont[Ligatures=TeX]{Linux Biolinum O}

% Deux-trois francisations supplémentaires
\addto\captionsfrench{\renewcommand{\contentsname}{Sommaire}}

% Typo plus propre et règles de typo françaises
\usepackage{xspace}
\usepackage{microtype}
\usepackage[hyphenation,parindent,lastparline]{impnattypo}
\usepackage[all]{nowidow}

%pour énumérer différement
\usepackage{enumitem}

% ajoute (entre autre) la bibliographie dans la table des matieres 
\usepackage[nottoc]{tocbibind}
\setcounter{secnumdepth}{1}
\setcounter{tocdepth}{2}
\addto\captionsfrench{\renewcommand{\bibname}{Lectures complémentaires en français}} 

%graphique
\usepackage{graphicx}
\usepackage{caption}
\usepackage{float}
\graphicspath{{images/}}
% urls internes et externes en noir cliquables 
\usepackage[urlcolor=blue,linkcolor=blue,colorlinks,breaklinks=true]{hyperref}
\usepackage[anythingbreaks]{breakurl}

\newcommand{\livre}[1]{\emph{#1}}
\newcommand{\site}[1]{\emph{#1}}

%titre
\title{Témoignages en faveur de la réforme du droit d'auteur}
% \subtitle{}
\author{Édité par Xavier Gillard}
%\date{\today}
\publishers{Soutenu par SavoirsCom1 et…}
% \uppertitleback{}
\lowertitleback{\centering

Livre sous licence Creative Commons CC-BY-SA :

Le texte complet de la licence est disponible à
\url{http://creativecommons.org/licenses/by-sa/2.0/fr/}

N'hésitez pas à me contacter à xavier@sploing.fr}

\begin{document}
\renewcommand{\labelitemi}{$\bullet$}
%\renewcommand{\bibsection}{\chapter{\refname}}

\maketitle

\chapter*{Avant-propos de l'éditeur-traducteur}
\addcontentsline{toc}{chapter}{Avant-propos de l'éditeur-traducteur}

Ce livre est la deuxième partie d'un premier livre de propositions politiques pour réformer le droit d'auteur. Comme lui, il est un livre de remix. Comme dans le premier livre et pour les mêmes raisons, j'enjoins le lecteur à essayer de découvrir les différents chapitres sans préjuger immédiatement de leur auteur original et dans le désordre s'il le souhaite.

Les témoignages de ce livre se répondent et expliquent les mesures proposées dans le premier livre. Tous les personnages croient qu'Internet doit permettre un monde meilleur, mais chacun aborde ce monde à sa manière. 

En page \pageref{cycliste}, le cycliste s'intéresse avant tout aux usages concrets qu'il peut faire de son vélo et constate un monde totalitaire. L'amoureux des droits de l'homme dénonce ce monde. Le propriétaire considère en page \pageref{proprio} que la propriété fait partie des droits naturels et la défend obstinément. Pour lui, la propriété intellectuelle est l'instrument de contrôle totalitaire idéal. Le conservateur désespère de la lenteur du progrès ou plutôt de la rapidité du regrès. La culture y est aux mains des plus riches pour lui en page  \pageref{conserv}. Le développeur se réjouit des opportunités à venir. Le chercheur feuillette avec joie l'abondance de travaux de ses collègues à sa disposition. 

À l'inverse, dans le monde dont rêve le poète en page \pageref{poete}, que le développeur développe en page \pageref{devadmin} et que le chercheur pratique en page \pageref{cherch} en feuillettant avec joie l'abondance de travaux produits par ses collègues, l'État remplit sa tâche première. Il sert ses citoyens.

La vue d'ensemble qui se dégage de la lecture des différents argumentaires doit permettre de comprendre quels sont les problèmes essentiels que pose le droit d'auteur à l'ère d'Internet. Elle doit convaincre de la nécessité pour l'État d'adopter le type de mesures promues dans le premier livre.

Ce deuxième livre est une deuxième bouteille à la mer~: le lira qui voudra pour en tirer les enseignements qu'il voudra. Nous, internautes qui avons financé l'impression et l'envoi à nos députés du premier livre de mesures, nous enjoignons nos députés à le lire avec attention et à transformer nos propositions en lois. 


\tableofcontents

\chapter{L'allégorie du cycliste}\label{cycliste}

Je me suis acheté un vélo. Un beau modèle. Je l’ai attendu longtemps.

Aux États-Unis, ça fait déjà longtemps qu’il est sur le marché. Pas en Allemagne. Et l’importer aurait été illégal. Le mois dernier, on pouvait le louer auprès d’une grande chaîne de télé pendant une
semaine et faire une virée. Ça m’a plu.

Mais à la fin de la journée, il était de nouveau verrouillé. Je devais attendre.

Par la suite le vélo est devenu disponible dans les arrières-cours de mon quartier pour pas un rond. Ça m’a paru un peu louche.

Mais je m’en fiche. Maintenant, j’ai mon vélo. Il est beau.

Il est marqué jusque sur les autocollants du cadre que je ne dois pas voler ou reconstituer de vélo. Logique. Pourquoi d’ailleurs~? Je l’ai bel et bien acheté.

Avant le premier démarrage j’ai dû appeler le fabricant et lui expliquer quels étaient les trois quartiers de la ville dans lesquels je voulais utiliser mon vélo. Lorsque je circule dans un quartier
non autorisé les freins s’enclenchent tout seul. Je n’ai rien à faire. Ça fait partie du service. Je peux alors appeler le fabricant et reconfigurer le vélo. De la sorte je circule dans toute la
ville.

Si je voulais louer mon vélo ça ne me serait pas permis. La selle envoie des petites décharges dans le corps et signifie son désaccord. C’est à la répartition des masses à l’arrière qu’elle reconnaît
qui s’assoit sur le vélo. Si ce n’est pas moi la sonnette carillonne. Du coup je fais attention à mon régime. Sinon mon vélo ne me reconnaît plus.

Il y a peu j’ai voulu le repeindre. Je trouvais que le kaki faisait vieux jeu. En grande surface on m’a ri au nez. Ça serait tout à fait illégal. Est-ce que j’avais demandé au fabricant~? Il aurait
sûrement dû prévoir quelque chose pour la couleur.

La ville vient de construire de nouvelles pistes cyclables et je trouvais qu’elle avait raison. Mais j’ai entendu une rumeur~: Mon vélo ne peut plus rouler dessus. Les pneus sont trop minces. Ils ne
passent plus sur le nouveau revêtement.

Mais une nouvelle génération arrive. Avec des chenilles. Ils seront beaucoup plus sûrs.

Maintenant il y a des postes de police sur les pistes. Pour contrôler qui est sur quel vélo. Quand on perd le contact visuel avec tous les postes le vélo éjecte son passager. Par temps de
brouillard on voit souvent des hommes joncher la route comme des fruits trop mûrs.

Si on me vole mon vélo, ça peut devenir encore plus cher. Parce que je l’ai diffusé. Le constructeur ne peut plus en vendre un directement au voleur. J’en suis responsable.

Tout ça m’est devenu trop périlleux. Maintenant je veux donner mon vélo à d’autres. Mais on chuchote que ce ne serait pas permis. Mon vélo ne serait qu’à moi. Je l’ai donc simplement
supprimé.\footnote{\url{http://www.137b.org/?p=2445} pour la version allemande sous CC-BY. Traduction sous CC-BY-SA}

\chapter{Le manifeste du poète}\label{poete}

Il n’y a pas de concept plus rabâché dans les discours médiatiques que celui de génération. J’ai déjà essayé une fois de compter les «~générations~» autoproclamées dans les journaux ces 10 dernières années. Je pense en avoir trouvé au moins douze. Mais elles avaient toutes une chose en commun~: elles n’existent que sur le papier. Dans la réalité personne n’a vécu
cette impulsion à la fois unique en son genre, tangible et inoubliable, cette expérience commune à travers laquelle nous resterions différentiables de toutes les générations précédentes. Nous
cherchions loin à l’horizon, mais la mutation fondamentale était passée inaperçue. Elle se cachait dans les câbles grâce auxquels la télévision a embrassé le pays, dans l’éclipse du réseau fixe par
celui mobile et avant tout dans l’accès à Internet généralisé. Ce n’est que maintenant que nous comprenons tout ce qui a changé dans les 15 dernières années.

Nous, les enfants d'Internet, qui avons grandi avec Internet et sur Internet, nous sommes une génération qui satisfait aux critères du concept. Il n’y a pas eu de déclic mais plutôt une
métamorphose de la vie. Ce qui nous unit à présent n’est pas un contexte culturel commun et déterminé mais le sentiment que nous pouvons définir librement ce contexte et ses cadres.

Pendant que j’écris, je sais bien que j’abuse du mot «~Nous~». Parce que notre «~nous~» est changeant, flou. Avant on aurait dit~: temporaire. Quand je dis «~nous~», je pense «~beaucoup d’entre nous~»
ou «~quelques uns d’entre nous~». Quand je dis «~nous sommes~», je pense «~il arrive que nous soyons~». C’est pourquoi je dis «~nous~» pour pouvoir parler de nous.

\section{Nous avons grandi avec Internet et sur Internet}
C’est pourquoi nous sommes différents. C’est le point crucial et à vrai dire pour nous l’étonnante différence~: Nous ne «~surfons~» pas sur Internet, et
Internet n’est pas pour nous un «~lieu~» ou un «~espace virtuel~». Internet n’est pas pour nous une extension externe de notre réalité, mais en fait partie~: une couche invisible mais toujours
présente qui s’entrelace à notre environnement physique, une sorte de seconde peau.

Nous n’utilisons pas Internet, nous vivons dedans et avec. Si nous devions vous expliquer à vous, les analogiques, notre «~Bildungsroman~»\footnote{«~Roman initiatique~». Style littéraire allemand du
18\ieme siècle qui thématise l'apprentissage de la vie d'un jeune personnage.}, nous vous dirions que toutes les
expériences essentielles que nous avons faites avaient le réseau en commun. En ligne, nous nous sommes créés des amis et des ennemis, nous avons préparé nos anti-sèches, nous avons planifié des
soirées et des sessions de travail, nous sommes tombés amoureux et nous nous sommes séparés.

Internet n’est pas pour nous une technologie que nous devions apprendre et que nous avons intégrée d’une manière ou d’une autre. Le réseau est avant tout un processus continu qui évolue en permanence
sous nos yeux, avec nous et à travers nous. Les technologies se créent et disparaissent dans notre environnement, les sites web naissent, se déploient et meurent, mais le réseau subsiste parce que
nous
sommes le réseau, nous qui communiquons bien plus efficacement que jamais dans l’histoire de l’humanité.

Nous avons grandi sur Internet, c’est pourquoi nous pensons différemment. Pouvoir trouver une information est pour nous aussi évident que pour vous pouvoir trouver une gare ou une poste dans une ville
inconnue. Quand nous voulons quelque chose, comme les premiers symptômes de la varicelle, les raisons du naufrage de l’«~Estonie~» ou savoir pourquoi notre facture d’eau semble aussi suspicieusement
haute, nous prenons les devants avec la sûreté d’un automobiliste guidé par un GPS.

Nous connaissons beaucoup d’endroits où trouver les informations désirées, nous savons comment on y arrive et nous pouvons juger de leur fiabilité. Nous avons appris à accepter que nous trouverons
rarement une réponse mais bien plutôt plusieurs. Nous extrayons de cette pluralité l’option la plus vraisemblable pour ignorer les autres. Nous sélectionnons, filtrons, nous souvenons et sommes
prêts à échanger ce que nous savons déjà contre quelque chose de neuf, de meilleur, quand nous butons contre un obstacle.

Pour nous Internet est une sorte de disque dur externe. Nous ne retenons pas de définition précise~: les dates, les montants, les formules, les paragraphes et les définitions exactes. Un résumé avec
le cœur de l’affaire nous suffit, et nous le travaillons pour le relier avec d’autres informations. Si nous avons besoin de détails, nous les cherchons dans les secondes qui suivent.

Nous n’avons pas besoin d’être des experts dans tout ce que nous connaissons parce que nous trouvons les hommes qui en ont fait leur spécialité et que nous pouvons croire. Les autres hommes ne
partagent pas leur expertise avec nous pour de l’argent mais plutôt parce qu’ils sont comme nous convaincus que l’information connaît un flux continuel et veut être libre, que nous profitons tous de
l’échange. Et ce tous les jours~: pendant nos études, au travail, lors de la résolution de problèmes quotidiens ou lorsque ça nous intéresse. Nous savons comment la concurrence fonctionne et l’aimons.
Mais notre compétition, notre vœu d’être différent, se base sur la capacité de manipuler et interpréter les informations, pas sur leur monopolisation.

\section{La participation à la culture n’est pas notre occupation des jours de fête}
La culture globale est le socle de notre identité, plus importante que notre compréhension particulière comme
tradition, les histoires de nos aînés, le statut social, l’origine ou même notre langue. Dans l’océan des événements culturels nous pêchons ce que bon nous semble, nous interagissons avec, notons et
sauvegardons nos évaluations sur des sites web dédiés qui proposent d’autres albums, films ou jeux qui pourraient nous plaire.

Nous regardons avec d’autres collègues certains de ces films, de ces séries ou vidéos, ou alors avec des amis du monde entier. Pour certains contenus notre appréciation ne sera jamais partagée qu’avec
un petit nombre de personnes que parfois nous ne verrons peut-être jamais dans la vie réelle. C’est pourquoi nous avons le sentiment que notre culture devient à la fois globale et individuelle. C’est
la raison pour laquelle nous avons besoin d’y accéder librement.

Cela ne veut pas dire que nous exigeons un accès gratuit à tous les biens culturels, même si quand nous créons quelque chose, bien souvent nous le mettons simplement en circulation. Nous comprenons
que la créativité demande toujours des efforts et de l’investissement, et ce malgré la démocratisation des techniques de montage audio ou vidéo. Nous sommes prêts à payer, mais les renchérissements
gigantesques des intermédiaires nous paraissent bêtement et simplement inadaptés. Pourquoi devrions nous payons pour la copie d’une information qui peut pourtant être copiée parfaitement très
rapidement, sans changer seulement d’un iota la valeur de l’original~? Si nous ne recevons que l’information brute, nous demandons un prix adapté. Nous sommes prêts à payer plus, mais alors nous
attendons aussi plus~: un emballage intéressant, un gadget, une meilleure qualité, la possibilité de pouvoir le regarder tout de suite, ici et maintenant, sans attendre la fin du téléchargement. Nous
pouvons même montrer de la gratitude et donner à l’artiste (puisque l’argent ne correspond plus qu’à des suites de chiffres sur un écran, payer est presque devenu un acte symbolique duquel les deux
partis devraient profiter), mais les objectifs de vente de quelque sorte que ce soit ne nous intéressent pas du tout. Ce n’est pas notre faute si votre modèle économique ne fait plus aucun sens dans
sa forme traditionnelle et si vous vous décidez à défendre votre modèle daté au lieu d’accepter les nouvelles exigences et d’essayer de nous fournir plus que ce que nous pourrions avoir autrement.

Encore une chose~: Nous ne voulons pas payer pour nos souvenirs. Les films qui datent de notre enfance, la musique qui nous a bercée pendant 10 ans~: Dans une mémoire mise en réseau, ce ne sont plus
que des souvenirs. Les rappeler et les échanger, les redévelopper, c’est pour nous aussi normal que pour vous les souvenirs de «~Casablanca~». Nous trouvons sur Internet les films que nous avons vus
enfants. Pouvez-vous vous imaginer que quelqu’un vous poursuive pour ça en justice~? Nous non plus.

\section{Ce qui nous importe le plus, c’est la liberté}
Nous sommes habitués à payer automatiquement nos factures tant que l’état de notre compte bancaire le permet. Nous savons que nous devons seulement remplir en ligne un formulaire et signer un contrat
livré par la poste quand nous ouvrons un compte ou voulons changer d’opérateur téléphonique. C’est pourquoi, en tant qu’utilisateur de l’État, nous sommes de plus en plus énervés par son interface
utilisateur archaïque. Nous ne comprenons pas pourquoi nous devrions remplir plusieurs formulaires papiers où le plus gros peut comporter plus de cent questions. Nous ne comprenons pas pourquoi nous
devons justifier d’un domicile fixe (il est absurde de devoir en avoir un) avant de pouvoir entreprendre d’autres démarches, comme si les administrations ne pouvaient pas régler ces choses sans que
nous intervenions.

Nous avons perdu la conviction née dans la crainte de nos parents que les trucs administratifs sont d’une importance capitale et que les affaires réglées par l’État sont sacrées. Ce respect ancré dans
la distance entre le citoyen solitaire et la hauteur majestueuse dans laquelle réside la classe dominante, à peine visible là-haut dans les nuages, nous ne l’avons plus. Notre compréhension de la
structure sociale est différente de la leur~: La société est un réseau, pas une pyramide. Nous sommes habitués à pouvoir adresser la parole à presque n’importe qui, qu’il soit journaliste, maire,
professeur d’université ou star de la pop, et nous n’avons pas besoin de qualifications qui iraient de pair avec notre statut social. Le succès d’une interaction tient uniquement à l’appréciation par
les autres de l’importance du contenu de notre message et de la pertinence d’y répondre. Et puisque nous avons le sentiment, grâce à la collaboration et à des disputes incessantes où nous défendons
contre la critique nos arguments, que nos opinions sont les meilleures, pourquoi ne pourrait-on pas attendre de dialogue sérieux avec le gouvernement~?

Nous ne sentons pas de respect religieux pour les «~institutions démocratiques~» dans leur forme actuelle, nous ne croyons pas à l’irrévocabilité de leurs rôles comme tous ceux qui considèrent que les
institutions démocratiques sont des objets de vénération qui se construisent d'eux-mêmes et à leur propre fin. Nous n’avons pas besoin de monuments. Nous avons besoin d’un système qui réponde
à nos
attentes, d’un système transparent et en état de marche. Et nous avons appris que le changement est possible, que tout système difficile à manier peut être remplacé par un plus efficace, qui soit
mieux adapté à nos exigences et laisse plus de marge de manœuvre.

Ce qui nous importe le plus, c’est la liberté. La liberté de s’exprimer, d’accéder à l’information et à la culture. Nous croyons qu’Internet est devenu ce qu’il est grâce à cette liberté et nous
pensons que c’est notre devoir de défendre cette liberté. Nous devons cela aux générations futures comme nous leur devons de protéger l’environnement.

Il est possible qu’aucun nom adapté n’existe pour désigner ce que nous voulons, ou que nous ne soyons pas encore tout à fait conscient qu'il s'agit là d'une vraie et réelle démocratie. Une
démocratie qui n’a peut-être jamais
été rêvée par vos journalistes.

\chapter{Le plaidoyer de l'amoureux de la liberté d'expression}\label{libe}

\section{Des 0 et des 1}
Les tentatives de renforcer l’interdiction actuelle du partage
non-commercial de la culture entre les citoyens menacent les droits fondamentaux comme le droit au
secret de la communication, celui à la liberté d’expression et même celui à un procès juste.

Le partage de fichier, c’est quand deux individus s’envoient des uns et des zéros. La seule manière
de limiter le partage de fichier c’est de surveiller ces suites binaires et donc toutes les
communications privées des uns et des autres. Il n’existe pas de possibilité de séparer les données
protégées des données privées. Il faut ouvrir tous les contenus pour les examiner. Le secret postal,
le droit de communiquer en privé avec son avocat ou à l’intimité lors des flirts par webcam
disparaissent tout comme la protection des sources pour les journalistes.

Je ne suis pas prêt à abandonner nos droits fondamentaux pour renforcer le droit d’auteur. Le
droit à la vie privée est plus important que le droit des grandes sociétés à continuer à gagner leur
vie comme avant. Ce dernier droit n’existe pas.

\section{Le droit d’auteur menace le droit au secret de la correspondance}

Le respect du droit d’auteur actuel est irréconciliable avec celui de la vie
privée.

Si je vous envoie un courriel, celui-ci peut contenir un morceau de musique. Si nous avons une
conversation vidéo, je peux partager une vidéo protégée. Pour faire appliquer le droit d’auteur, il
faut pouvoir le détecter. Or le seul moyen de détecter des contenus protégés est d’écouter tous les
0 et les 1 entrants et sortants des ordinateurs.

Il n’y a aucun moyen de permettre le droit à la correspondance privée pour certains contenus et pas
pour d’autres. Vous devez briser le sceau et analyser les contenus pour les trier en deux catégories~: autorisé et non autorisé. Dès ce moment le sceau est brisé. Soit il y a un sceau sur tout soit
il
n’y en a aucun.

Plus généralement tout canal de communication numérique qui peut être utilisé pour échanger des
informations privées peut aussi être utilisé pour transférer à des inconnus des copies numériques de
fichiers protégés. Faire absolument respecter le droit d’auteur implique donc un viol massif du
secret de la correspondance. 

Les lois qui s’appliquent hors-ligne devraient aussi s'appliquer en ligne. C’est tout à fait raisonnable. Internet n’est pas un cas particulier mais fait partie de la réalité. Le problème apparaît quand une industrie dépassée mais puissante réalise qu’une application juste et égalitaire de la loi signifie que son monopole de la
distribution est terminé.

Pour comprendre l’absurdité des requêtes de l’industrie du droit d’auteur, on doit se demander quels
droits nous considérons pour acquis dans le monde analogique. Ces droits doivent aussi s’appliquer
au monde numérique, puisqu’au moins en théorie la loi ne fait pas de différences entre les moyens
de communication.

Regardons quels droits j’ai quand je communique avec quelqu’un via des canaux analogiques, en
utilisant du papier, un stylo, une enveloppe et un timbre, c’est-à-dire quand j’écris une lettre~:

\begin{itemize}
\item 
J’ai le droit de communiquer anonymement ou non. Moi et moi seul choisis de m’identifier dans
l’adresse sur l’enveloppe et/ou dans l’enveloppe.\item
Personne n’a le droit de l’intercepter pour briser le cachet et examiner son contenu sauf si je suis
déjà suspecté d’un crime. Dans ce cas cela passe par une procédure judiciaire.\item
Personne n’exige que je l’aide à ouvrir mes lettres pour me surveiller.\item
Nul n’a le droit d’en altérer le contenu ou de refuser de l’acheminer.\item
Nul n’a le droit de rester devant la boîte et de noter à qui j’écris, la durée de mes messages ou
qui me répond.\item
Le facteur n’est pas responsable de ce que j’ai écrit. Il n’est qu’un intermédiaire.\end{itemize}

Toutes ces règles fondamentales sont systématiquement attaquées par l’industrie du droit d’auteur.

Cette industrie exige que les fournisseurs d’accès à Internet installent des appareils
d’enregistrement et de censure des communications au milieu de leurs propres appareils. Elle les
poursuit en justice pour cela. Elle se plaint sans cesse de l’immunité des intermédiaires. Elle
demande aux autorités d’identifier les gens qui communiquent, nous niant ainsi nos droits
fondamentaux, surtout ceux de la liberté d’expression et au secret de la correspondance. Elle a
l’outrance d’aller jusqu’à demander la censure des télécommunications.

De tels droits font partie des libertés pour la défense desquelles nos aïeux ont donné leur sang.
Il est plus qu’obscène de les abandonner au profit d’une industrie obsolète qui désire préserver son
monopole et veut pour ce avoir à titre privé encore plus de pouvoirs que la police quand elle
recherche des criminels.

Lorsque les photocopieurs sont arrivés dans les années 1960, les éditeurs ont tenté de les interdire
au prétexte qu’ils pourraient être utilisés pour copier des livres que les gens s’enverraient
ensuite par la poste. À l’époque, tout le monde a expliqué aux éditeurs la dure vérité~: Bien que le
monopole du droit d’auteur soit valide, personne n’a le droit de briser le secret de la
correspondance simplement pour vérifier qu’aucune violation du droit d’auteur n’est commise, donc il
n’y a rien à faire. La règle tient toujours hors-ligne. Pourquoi ne s’appliquerait-elle pas en
ligne~?

L’industrie du droit d’auteur se plaint parfois de la licence qui existerait sur Internet alors que
les mêmes lois devraient valoir pour les deux domaines. Sur ce point, je ne peux être plus d'accord.

Malheureusement, c’est bien le contraire qui est en train de se passer. Les corporations essayent de
prendre le contrôle de nos moyens de communications en prétextant des problèmes de droit d’auteur.
Le plus souvent elles aident les politiciens qui aspirent à la même chose mais prétextent eux des
impératifs de lutte contre le terrorisme ou des craintes McCarthystes pour la sécurité publique.
Regardons par exemple ce qui s’est passé dans le monde arabe ou en Angleterre en 2011.

Les blanc-seings aveuglément donnés aux autorités me font peur car je me
méfie de l’incessant désir de savoir ce dont je discute et avec qui qu'affichent ouvertement les corporations et les politiques.

Pour dramatiser la situation, il ne s’agit pas que de mouchardage. Les corporations et les
politiques veulent gagner le droit de nous réduire au silence.

L’industrie du droit d’auteur exige le droit de tuer les commutateurs essentiels de nos
communications. Pour peu que nos propos soient suffisamment dérangeants, que ce soit selon les
membres de l’industrie du divertissement ou de la classe politique dirigeante, la ligne est coupée.

Il y a 20 ans, cela aurait paru être une horreur absolue. Aujourd’hui, c’est devenu la réalité. Vous
ne le croyez pas~? Essayez de parler de \site{The Pirate Bay} sur MSN ou Facebook et admirez le silence.
L’industrie du droit d’auteur se bat pour devenir de plus en plus envahissante. De même certains
politiciens ont en poche leur propre calendrier sur la question.

Bien que l’industrie du droit d’auteur et les politiciens Big Brother ne partagent pas les mêmes
motivations, ils poussent dans la même direction et promeuvent les mêmes changements.

Pendant ce temps, les déplacements des citoyens dans les rues sont tracés minute par minute et leur
historique est enregistré.

Comment imaginer une révolution possible quand tout ce que vous dites est étouffé dans l’œuf avant
d’atteindre une quelconque oreille et quand le régime peut surveiller qui a rencontré qui, où et
quand, voire peut vous déconnecter par un simple appui sur un bouton distant~? Comment exercer son
droit de résistance à l’oppression dans un monde où absolument tout est sans cesse surveillé~?

\section{La riposte graduée, ou l’art de débrancher les gens par caprice}

«~Trois coups et t’es dehors~» est une expressione américaine qui a ses origines dans le
base-ball. Les
politiciens américains l'ont transformée en principe légal. Dans le contexte de la régulation d'Internet,
«~trois coups~» signifie que quelqu’un accusé de partage illégal trois fois de suite par des
ayants-droits est déconnecté. «~Réponse graduée~» est l’expression la plus utilisée en France.

En France, nous avons la loi Hadopi, qui demande aux Fournisseurs d’Accès
à Internet (FAI) de
débrancher ceux qui ont reçu un, deux ou trois avertissements. Ceci dit, en France, les
avertissements ne sont pas nécessaires, la déconnexion peut être immédiate. En Angleterre, le
Digital Economy Act dit essentiellement met la même chose. En Italie, qui a voulu
faire mieux que
tout le monde, un seul
avertissement suffit pour être banni.

Sur le principe, ces lois laissent aux majors la liberté de se poser en juge et partie, en désignant
les individus suspectés de porter atteinte à leurs droits et en forçant les FAI à appliquer la
sanction qu’est la déconnexion.

Laissons de côté la question de savoir si privatiser l’application de la loi est une bonne idée. Ce
que je défends ici c’est que vouloir débrancher arbitrairement les gens d’Internet est un
caprice irresponsable. Caprice car, comme je le défends, il faut réformer le droit d’auteur
plutôt que l’appliquer aveuglément. Irresponsable à cause des effets qu’une coupure d’Internet a sur
les membres d’un foyer.

Considérons deux secondes ce que ça veut concrètement dire~:

\begin{itemize}
\item 
L’arrêt des études

La plupart des systèmes scolaires, particulièrement l’université, prennent pour acquis que vous avez
un accès Internet. Si vous êtes un étudiant, vous aurez besoin d’un accès Internet pour toutes les
choses de la vie courante comme vous tenir au courant de votre emploi du temps, préparer des
exposés en groupes ou chercher la littérature sur un sujet précis. Les études montrent que la
majorité des étudiants piratent. Devriez-vous leur couper leur accès et arrêter là leurs études~? Ou
peut-être faire un exemple sur 5-10\%~? Quel est le meilleur sacrifice~?

\item 
La mort des petites entreprises

Si vous possédez une PME, vous dépendez entièrement d’Internet quelque soit votre branche. Pour
contacter les clients, mettre à jour vos actualités, commander vos fournitures et simplement
correspondre. Est-il raisonnable de ruiner le père ou la mère de famille parce que l’adolescent a
téléchargé le dernier tube à la mode~? Couper une connexion punit un foyer entier.
\item 
La coupure des relations sociales

Les jeunes sont socialisés en grande partie via Internet. Il n’est pas rare que certains aient des
connaissances très proches qu’ils n’ont que rarement rencontrés en vrai. Ce n’était pas le cas il y
a 30 ans mais les choses ont changé. Être tout d’un coup coupé du monde est normalement réservé aux
dangereux criminels.
\item 
La perte des droits civiques

L’accès à Internet est devenu essentiel pour prendre part aux débats public. Non seulement pour être
tenu au courant mais aussi pour monter son propre blog, commenter ceux des autres, réagir en direct
sur Twitter et organiser des événements ou les rejoindre.\end{itemize}

«~Si vos enfants sont méchants, nous leur enlevons leurs jouets~» est un bon résumé de l’approche
répressive des politiques qui défendent la riposte graduée.

Mais les citoyens ne sont pas des enfants, n’ont pas à subir l’arrogance de leurs représentants.
Internet n’est pas un jouet. C’est devenu un rouage essentiel de la société, et une infrastructure
dont tout le monde a besoin.

C’est dans cet esprit que le Conseil Constitutionnel
français 
a déclaré le 10 juin 2009 que la suspension ne pouvait
passer que par un juge car~:
\begin{quotation}

Aux termes de l’article 11 de la Déclaration des droits de l’homme et du citoyen de 1789~:
\begin{quotation}
«~La libre communication des pensées et des opinions est un des droits les plus précieux de l’homme~: tout citoyen peut donc parler, écrire, imprimer librement, sauf à répondre de l’abus de cette
liberté dans les cas déterminés par la loi.~»
\end{quotation}

En l’état actuel des moyens de communication et eu égard au développement généralisé des services de
communication au public en ligne ainsi qu’à l’importance prise par ces services pour la
participation à la vie démocratique et l’expression des idées et des opinions, ce droit implique la
liberté d’accéder à ces services.
\end{quotation}

Les politiques qui ne le comprennent pas ne devraient pas être surpris que la jeune génération ne
vote pas pour eux.

\section{De toute manière, la répression ne marche pas}
La politique répressive des lobbys de la culture n’a jamais donné que des résultats éphémères et
incertains. Car rien n’arrêtera le piratage.

Il y a quelques siècles, la peine pour une copie interdite a fini par être le supplice de la roue.
Après d’affreuses souffrances le condamné mourrait de soif dans les jours suivant la destruction de
ses membres.

Le monopole de copie à cette époque concernait les modèles de couture. C’était au 18\ieme siècle en
France avant la Révolution. Certains modèles était plus populaires que d’autres et pour remplir un
peu plus ses caisses, le Roi avait vendu des monopoles d’exploitation à quelques nobles privilégiés qui en retour pouvaient casser des bras et des jambes (et le faisaient).

Mais les paysans et roturiers pouvaient produire ces modèles d’eux-mêmes. Ils pouvaient les
pirater et le firent largement. Les nobles demandèrent donc justice au Roi. Le Roi commença par
introduire des amendes, puis des châtiments corporels mineurs, puis finit par condamner ces
infractions au monopole nobiliaire par la torture et ne condamna pas seulement une poignée de
pauvres hères.

L’économiste et historien Eli Heckscher écrit dans son classique \livre{Merkantilismen}~:
\begin{quotation}
 Bien sûr, l’essai de stopper un développement encouragé par des modes féminines éphémères ne pouvait
pas réussir. On considère qu’en France la police a tué 16 000 personnes pour copie interdite sans
compter ceux condamnés aux galères. À Valence, il arriva que 77 personnes soient pendues en une seule fois, avec 58
rouées et 631 envoyées aux galères, un acquitté. Mais l’usage du calicot imprimé dont la copie était
réprimée a continué à se répandre en France et ailleurs.
\end{quotation}

Voilà le plus fascinant~:
\begin{quotation}
 La peine capitale n’a pas réussi à ralentir le piratage des fabriques des nobles. Même ceux qui
connaissaient des artisans exécutés et torturés continuèrent à pirater sur le même rythme.
\end{quotation}

Cela remet sérieusement en doute la pertinence d’une politique répressive. Combien de temps encore
est-ce que les politiciens continueront à croire que la répression sert à quelque chose quand
l’histoire nous apprend que même la peine de mort n’empêche pas un phénomène de se propager à toute
vitesse et de perdurer~?

Pour résoudre le problème, nous devons trouver une autre solution. Celle-ci existe. Une fois que
vous acceptez de réduire la protection des œuvres sous droit d’auteur et d’autoriser le partage
non-commercial, une foultitude d’avantages apparaît. Les deux milliards d’humains connectés de la
planète auraient accès 24h/24 à une culture immense à un coût dérisoire. C’est un
énorme progrès par rapport à la bibliothèque d’Alexandrie. La technologie existe. Il reste à
l’accepter.

\section{Partage de fichiers et droits fondamentaux – la ligne de tension}
 La relation entre le partage de fichier et les droits fondamentaux est très simple. Le partage de
fichier est là pour rester. Peu importe ce que les uns ou les autres feront, cela ne changera pas les faits. À long terme, il deviendra impossible de faire payer
simplement pour des copies numériques. Cela fait partie de l’histoire de la technologie et il n’y a
rien d’autre à ajouter.

Alors quel est le problème~? L’industrie du droit d’auteur ne sera pas capable d’arrêter le partage
de fichiers. Les pirates trouveront d’autres moyens de se protéger grâce à l’anonymisation, le
chiffrement, etc… Aucun problème pour eux. Mais l’industrie du droit d’auteur
punit et punira des
individus pris au hasard de façon dure et disproportionnée pour l’exemple.Ceci n’est pas acceptable. 

Je viens d'expliquer que la seule façon d’essayer de réduire le partage de fichiers est d’introduire la surveillance de masse
de tous les utilisateurs d’Internet. Mais même cela n’est pas très efficace comme l’ont montrées
les expériences des décennies passées. L’industrie du droit d’auteur sait tout cela.

Ceux qui pensent que le partage de fichiers est dangereux pour la société et qu’il faut
l’éliminer doivent se demander s’ils sont préparés à accepter une société totalement surveillée pour
arriver à cela. Parce qu’une fois que les systèmes de surveillance ont été installés, ils peuvent
être utilisés pour n’importe quoi qui plaira aux personnes qui les maîtrisent.

Vous pourriez avoir l’impression que vous n’avez «~rien à cacher~» lorsqu’il s’agit de partage
de fichiers, si c’est quelque chose que vous ne pratiquez pas. Mais pouvez-vous être certains que
vous n’aurez pas toujours «~rien à cacher~» quand il s’agira d’exprimer des points de vue que le futur
gouvernement pourrait ne pas apprécier~? Savez-vous déjà que vous serez loyal au gouvernement à la
prochaine période de McCarthysme ou pire, quand l’État commencera a écouter et à verrouiller
certaines sympathies politiques~?

Si vous construisez un système de surveillance de masse, il y aura un système de surveillance de
masse à disposition pour tous les abus. Voilà l’essentiel du problème, sa ligne de tension.


\chapter{La colère du propriétaire}\label{proprio}

\section{Biens tangibles et intangibles}

Les biens que l'on peut posséder sont a minima les biens tangibles, ceux que l'on peut toucher. Ce sont les meubles et immeubles. Les immeubles comprennent les biens tangibles qui ne peuvent être déplacés, comme les maisons et les terres. Les meubles concernent tout ce qui peut l'être, comme les meubles de la maison, les voitures, les montres…

Tous les objets tangibles, qu'il s'agisse de biens acquis ou créés, de meubles ou d'immeubles ou de nos propres corps, peuvent être soumis à un contrôle légitime par des individus précis appelé droit de propriété.

Lorsque l'on passe des objets tangibles et corporels à des objets intangibles la situation est moins claire. Est-ce que nous avons des droits sur nos créations intellectuelles~? Est-ce que le système légal devrait protéger ces droits~? Nous verrons que les droits patrimoniaux amènent surtout une réduction de nos droits de propriété sur des objets tangibles. Ils restreignent donc nos libertés. À l'ère d'Internet, cette restriction ne peut pas être justifiée. Pour cette raison ils doivent être abrogés ou diminués. 

\section{Droit de propriété et rareté}

Jetons un coup d'œil sur l'idée du droit de propriété pour les biens tangibles. Qu'est-ce qui est intéressant dans les biens tangibles et rend leur appropriation par certains et non d'autres intéressante~? 

C'est leur rareté qui est responsable de la situation. C'est un fait que des conflits d'exclusivité d'usage existent ou peuvent exister autour des biens tangibles. C'est cette possibilité de conflit autour d'eux qui rend les biens tangibles rares, et incite à vouloir créer des règles pour gérer leur usage. Ainsi, la fonction sociale fondamentale et éthique des droits de propriété est de réguler les conflits que les biens engendrent. 

Si nous vivions dans un jardin d'Éden où la terre et les biens étaient infiniment abondants, il n'y aurait aucun besoin d'avoir des règles de propriété. Le concept même de propriété serait vicié car il n'y aurait aucun conflit. Si vous pouvez recréer immédiatement une tondeuse à gazon lorsque je vous prends la vôtre, je ne vous la «volerais» pas. 

Or la nature contient des choses qui sont économiquement rares. Mon utilisation de ces choses exclue que les utilisiez en même temps ou après, et vice-versa. La fonction des droits de propriété est donc d'allouer à chacun un droit exclusif d'utilisation de certaines ressources. Pour que cela soit possible, il faut que ces droits soient à la fois visibles et justes. Clairement, afin que les individus puissent éviter d'utiliser les biens des autres, les frontières des biens en question doivent être objectives, id est certifiables par tout un chacun, intersubjectivement. Elles doivent être visibles. En d'autres termes, «les bonnes barrières font les bons voisins». 

Les droits de propriété doivent aussi être raisonnablement justes parce qu'ils ne pourront servir à éviter les conflits s'ils ne sont pas reconnus comme loyaux par ceux qu'ils affectent. Si les droits de propriété sont alloués de manière déloyale ou par la force, ils sont nuls et non avenus, c'est le règne de la force et la disparition du droit. 

\section{Le versant patrimonial du droit d'auteur}

Le droit d'auteur comprend des droits moraux inaliénables et des droits patrimoniaux. Les droits moraux obligent essentiellement à citer l'auteur d'un travail que l'on réutilise sans laisser croire qu'il soutient votre travail. Ces droits moraux sont des convenances inscrites dans la loi qui ne menacent en rien les droits fondamentaux ou la créativité.

Les droits d'auteur patrimoniaux sont des droits donnés aux auteurs de «travaux originaux» comme les livres, les articles, les films et les programmes d'ordinateur. Ils donnent à leurs possesseurs le droit exclusif de reproduire leurs œuvres, d'en créer des œuvres dérivées, de les mettre en scène ou de les diffuser au public. Ces droits ne protègent que la forme ou l'expression des idées, pas les idées sous-jacentes elles-même.

Ces droits patrimoniaux apparaissent automatiquement dès que l'œuvre est fixée dans un médium tangible d'expression et dure pour la vie entière de l'auteur plus soixante-dix ans.

Le droit d'auteur est un droit sur un objet idéel. La propriété d'une idée ou d'un objet idéel donne de fait à son propriétaire des droits sur les instanciations physiques de l'œuvre ou invention. Dans le cas d'un livre protégé, le détenteur du droit de copie A a un droit sur l'objet idéel. Chaque livre n'en est rien de plus qu'un exemplaire. Le système donne des droits sur l'arrangement même des mots. Par conséquent A a un droit sur toute réalisation concrète de cet arrangement comme une impression papier ou un affichage sur un écran. 

Ainsi, si A écrit un roman, il a un droit de copie sur l'œuvre même. S'il vend une copie physique du roman à B, sous forme d'un livre, alors B ne possède que la copie physique du roman, il ne possède pas le roman lui-même, et n'a pas le droit de faire une copie du roman, même avec sa propre encre et son propre papier.

Les droits patrimoniaux d'un auteur sur ses œuvres intangibles imposent donc des restrictions des droits de propriété des autres sur leurs biens tangibles. Nous montrons que ces restrictions sont injustifiées.

\section{Pourquoi un droit de propriété sur les idées~?}

\subsection{Les idées ne sont pas rares}

De même que la tondeuse magique du jardin d'Éden, les idées ne sont pas rares. Si j'invente une technique pour filer le coton, que vous copiez cette technique ne m'empêchera pas de l'utiliser. J'aurai toujours et ma technique et mon coton. Il n'y a là aucune rareté économique. Aucun conflit ne peut suivre de la rareté de l'objet. Il n'y a donc aucun besoin d'exclusivité. Cela vaut pour les livres aussi par exemple. Si vous copiez un livre que j'ai écrit, j'ai toujours l'original avec moi et votre copie de mon organisation des mots n'exclue pas que je continue à la posséder. C'est pourquoi les écrits ou les inventions ne sont pas rares dans le même sens que des hectares de terre ou des voitures. Pour reprendre les mots célèbres de Thomas Jefferson, lui-même inventeur et membre du premier bureau d'enregistrement des États-Unis. Jefferson était partisan d'une approche utilitariste du système de brevet. C'est pourquoi il est resté tout sa vie sceptique à son propos.

\begin{quotation}
Celui à qui je donne une idée s'instruit sans nuire à mon instruction, comme celui qui allume une lampe m'éclaire sans en être moins éclairé. 
\end{quotation}

Puisque l'utilisation de nos idées par les autres ne nous en prive pas, il n'y a aucun conflit d'usage possible et aucun intérêt à mettre en place des droits de propriété. Pourrait-on cependant justifier autrement que par la rareté des ressources un droit de propriété sur les idées~?

\subsection{Créer ne suffit pas pour s'approprier}
La règle générale pour les biens tangibles est que si le bien n'appartient à personne le premier à s'en emparer en l'occupant ou l'utilisant le possède. Cela permet d'établir une règle claire qui ne soit pas biaisée ou arbitraire. Il y a plusieurs manières de s'emparer d'un objet qui n'appartient à personne. Je peux par exemple cueillir une pomme dans une forêt qui n'appartient à personne pour me l'approprier, ou planter une barrière dans un territoire vierge. 

Cependant, quel serait l'équivalent de cette règle pour les biens idéels~? Le candidat le plus évident est que toute idée appartient à son créateur.

Or parler de création n'est pertinent tant que la création révèle une primo-occupation ou est en soi un acte de primo-occupation. Cependant, la création n'est pas par elle-même une condition suffisante pour primo-posséder un bien. Il n'est pas possible de créer des objets tangibles sans utiliser d'abord des matières premières, et ces matières premières doivent être rares. Soit je les possède, soit je ne les possède pas. Si je ne les possèdais pas et si elles appartenaient à quelqu'un lorsque j'ai créé le nouvel objet, alors je ne possédais pas ce nouvel objet non plus, et je devrai peut-être même dédommager son propriétaire pour avoir utilisé ses biens. Par exemple, si je forge une épée avec un métal qui ne m'appartient pas, le métal ne m'apppartient pas plus après avoir forgé l'épée qu'avant. Seule la primo-appropriation de ce métal est un critère valable pour primo-posséder un objet, non sa création. 

\subsection{Protéger l'originalité c'est empêcher la créativité}
On peut croire qu'à la différence des métaux les idées n'existent pas avant que leurs créateurs leur donnent vie. Elles seraient originales. Mais qui serait assez fou pour se croire génial au point d'avoir des idées complètement originales~? Les auteurs sont toujours perchés sur les épaules de maîtres auxquels ils payent leur tribut.

Sir Arthur Conan Doyle écrivait :
\begin{quotation}
Si chaque auteur qui est redevable de sa production à Poe et reçoit des honoraires pour une histoire devait en donner un dixième pour l’érection d’un monument à la mémoire du maître, la pyramide serait aussi grande celle de Kéops.
\end{quotation}

Justifier les droits d'auteur patrimoniaux par l'originalité des créations, c'est donc les étouffer dans l'œuf. Personne n'ayant jamais une idée absolument originale, personne ne devrait complètement posséder une idée. 

Il serait quand même possible de posséder partiellement une idée~: on possèderait son «~agencement~». Si aucune idée n'est complètement originale, les réorganisations d'anciennes idées pourraient justifier une règle d'appropriation. 

Seulement cette justification est viciée car il est difficile de savoir dans quelle mesure de nouveaux agencements sont vraiment nouveaux. Définir l'originalité n'est pas chose aisée, même pour un juge. Une règle de propriété basée sur cette règle ne respecte donc pas l'impératif de clarté et d'univocité de la loi. Son existence entraînera des conflits qui sinon n'auraient jamais existé. Ce n'est pas ce que l'on attend d'une loi. 

Par conséquent il est masochiste pour un auteur de défendre des droits patrimoniaux fondés sur une exigence d'originalité. Est-ce vraiment le but d'un auteur de passer son temps à calculer ce qu'il doit à chaque personne à laquelle il a emprunté une idée~? Est-ce que les auteurs sont tellement enchantés que cela de payer des redevances à tous ceux qui les menacent de procès pour infraction au droit d'auteur par peur des frais judiciaires, quand bien même les requêtes seraient infondées~? Nous-mêmes, voulons-nous favoriser un monde où seuls les plus riches peuvent créer~?

Pour les auteurs, un système juridique axé sur l'originalité ressemble plus à un cauchemar bureaucratique destiné à empêcher toute création qu'à autre chose. Au contraire ne pas protéger les idées revient à créer un jardin d'Éden pour la créativité. Cependant certains semblent penser le contraire. Ils défendent que le droit d'auteur est utile pour le progrès des arts.

\section{Peut-on justifier cette infraction à mes droits de propriété par son utilité~?}

Le droit d'auteur redistribue les régimes de propriété en prenant à ceux qui possèdent des biens tangibles pour donner aux créateurs et inventeurs. Prima facie, le droit d'auteur est donc une infraction aux droits de propriété. C'est cette redistribution invasive qui doit être justifiée. Le plus souvent ce sont des arguments utilitaristes qui sont utilisés. 

Les utilitaristes soutiennent que le but des droits d'auteur patrimoniaux est l'encouragement à la création et l'innovation. C'est pourquoi les moyens apparemment immoraux que sont les restrictions des droits de propriété des individus sur leurs biens physiques sont justifiés. Mais il y a quelques problèmes fondamentaux qui découlent d'une justification purement utilitariste de la loi ou de droits fondamentaux. 

Tout d'abord, supposons que le bien-être ou l'utilité soient maximisées en adoptant certaines règles légales~: la «taille du gâteau» est agrandie. Est-ce que ces règles sont pour autant justifiées~?

Par exemple, on pourrait dire que le bien-être net de la population totale s'améliore lorsque nous donnons la moitié de la fortune du pour-cent le plus riche aux dix pour-cents les plus pauvres, et que ce vol de la propriété de A pour le donner à B améliore plus le bien-être de B que cela ne diminue celui de A. Cela ne veut pas dire que le vol des biens de A soit justifié. Le but de la loi n'est pas nécessairement de maximiser le bien-être des gens, mais d'être de donner à chacun ce qui lui revient de droit. Même si le droit d'auteur arrivait à augmenter notre bien-être total, cela ne voudrait pas dire que la violation de certains droits fondamentaux comme le droit à la propriété serait pleinement justifiée. Plus caricaturalement, ce n'est pas non plus parce que tuer quelques personnes peut permettre d'augmenter le bien-être général que c'est justifié.

Outre ces problèmes éthiques, une approche purement utilitariste est incohérente. Elle implique nécessairement de faire des comparaisons d'utilité interpersonnelles illégitimes, comme lorsque le «coût» des lois sur le droit d'auteur est soustrait des «bénéfices» de ces lois pour déterminer s'il y a un bénéfice net. Mais parler de valeur ne veut pas dire tout restreindre à des valeurs marchandes, des prix. De fait, cela n'a rien à voir. La libre circulation du savoir a une valeur sociale inestimable qui dépasse largement le cadre marchand. Est-ce que fixer un prix objectif au savoir par la loi est vraiment intellectuellement honnête~?

Finalement, même si nous laissons de côté les problèmes de comparaison interpersonnelle et de justice de la redistribution pour continuer, en utilisant des techniques de mesure utilitaristes standards, il n'est pas du tout clair que le droit d'auteur amène réellement quelque changement que ce soit au bien-être général. Aucune étude économétrique n'est arrivée à montrer des gains de bien-être net de manière convaincante. Il paraît même évident que le droit d'auteur limite la diversité des œuvres disponibles en empêchant la culture du remix. Celle-ci s'épanouit sur Internet car elle méprise les lois sur le droit d'auteur ou de copie.

Ce qui est certain, c'est que les poursuites pour droit d'auteur engendrent des coûts juridiques et administratifs élevés. Hadopi, les chasses internationales aux pirates, l'installation d'infrastructures de censure ou de surveillance et les procès inutiles ont un coût bien réel mais n'ont jamais arrêté le piratage. Or est-ce que ceux qui appellent à l'utilisation de la force pour réglementer l'usage de leurs biens par les autres ne devraient pas pouvoir montrer que cet usage de la force est vraiment utile~?

Il faut rappeler que lorsque nous appelons à l'existence de certaines lois ou de certains droits et que nous analysons leur légitimité, nous analysons la légitimité et l'éthique de l'usage de la force. Se demander si une loi doit exister ou non est se demander s'il est juste d'utiliser la force contre ses contrevenants. Il n'est pas étonnant qu'un simple calcul de maximisation des gains ne suffit pas à répondre à la question. 

En résumé, l'analyse utilitariste est largement inefficace, car il est méthodologiquement douteux de ne s'intéresser qu'à la taille réelle du gâteau, sans compter que déterminer cette taille est un vrai pari et qu'il est empiriquement improbable qu'elle augmente réellement.

\vspace{3em}
Les justifications des droits d'auteur patrimoniaux sont invalides ou pour le moins douteuses. Le droit d'auteur ne peut donc justifier aucune restriction de nos droits de propriétés sur des objets que nous avons achetés. Il ne peut pas non plus justifier le déploiement totalitaire d'outils de surveillance de nos moyens de communication, de nos ordinateurs et in fine de notre vie privée.

\chapter{La déception du conservateur}\label{conserv}
\section{Le domaine public est en danger en France}
Les marchés, la démocratie, la science, la liberté d'expression et l'art dépendent bien plus des œuvres et productions constituant le domaine public librement accessible que des productions informationnelles couvertes par des droits de propriété intellectuelle. Le domaine public n'est pas un résidu qui se déposerait lorsque tout ce qui a de la valeur aurait été saisi par les lois sur la propriété intellectuelle. Le domaine public est la carrière dont nous extrayons les pierres avec lesquelles nous bâtissons notre culture. En fait, il constitue la majorité de notre culture.

En France, ce domaine public est en danger. Il est menacé par les institutions culturelles qui devraient au contraire le protéger. Entendons-nous bien, je ne suis pas en train de dire que les bibliothèques, musées ou archives n’assurent pas leur rôle de conservation patrimoniale des oeuvres physiques qu’elles conservent. Le problème est qu'elles sont une majorité écrasante détourner le droit d'auteur pour porter atteinte à la liberté de réutilisation qui devrait être le pendant logique du domaine public.

Le problème est connu~: allez sur les sites des musées, des bibliothèques et des archives. Vous y trouverez de très nombreuses oeuvres du domaine public numérisées et offertes à la consultation du public. Mais dans une écrasante majorité des cas, les images seront accompagnées d’une mention restrictive, qui restreindra les usages d’une manière ou d’une autre, quand un brutal «~Copyright : tous droits réservés~» ne sera pas purement et simplement appliqué.

Une telle revendication empêche toutes les formes de réutilisation, y compris les plus légitimes~: 

\begin{itemize}
\item Elle bloque les usages pédagogiques et de recherche.
\item Elle interdit aux simples internautes de les reprendre pour illustrer leurs blogues et leurs sites. 
\item Elle ne permet pas d’aller enrichir les articles de Wikipédia et d’autres sites collaboratifs. 
\item Elle bloque les réutilisations commerciales.
\end{itemize}

Dans les musées, au lieu de ce droit d'auteur brutal, on use parfois d’un autre stratagème, en reconnaissant un droit d’auteur aux photographes qui prennent des clichés de tableaux. Le musée se fait ensuite céder ce droit d’auteur par contrat, ce qui lui permet d’appliquer à la fois le sien et celui du photographe.

D’autres institutions ont récemment développé une tactique encore plus subtile. Elles considèrent que la numérisation produit des données publiques (les oeuvres deviennent des séries de 0 et de 1, qui seraient constitutives d’informations publiques au sens de la loi du 17 juillet 1978). C’est le cas par exemple parfois à la Réunion des musées nationaux, à la Bibliothèque nationale de France pour Gallica, aux Archives nationales pour Archim et dans bon nombre de services d’archives départementales. Elles peuvent alors conditionner certaines formes de réutilisation à la passation d’une licence et au paiement d’une redevance. Recouvert par cette couche de droit des données publiques, le domaine public disparaît. Pourtant cette lecture de la loi de 1978 est juridiquement contestable, même si aucun tribunal n’a encore statué sur la question.

Les manifestations de ce saccage juridique peuvent prendre d’autres détours encore, comme lorsqu’au Musée d’Orsay, on interdit avec obstination aux visiteurs de prendre des photographies (même sans flash), y compris lorsque les oeuvres appartiennent au domaine public.

\section{La marchandisation du domaine public par les institutions publiques}

Aujourd’hui, dans le but de marchandiser le domaine public, on ne met plus en ligne les images numérisées pour mieux les vendre sous forme de bases de données, en partenariat avec des entreprises privées qui assureront la numérisation et se rémunèreront sur le produit des ventes.

Cette formule est appliquée en ce moment à la Bibliothèque nationale de France dans le cadre de plusieurs appels à partenariats mettant à contribution les Investissements d’avenirs du Grand Emprunt national.

Un prestataire privé numérise les ouvrages en recevant un soutien financier via le Grand Emprunt. Seule une partie minime du corpus pourra être accessible librement et gratuitement sur Gallica, la bibliothèque numérique de la Bibliothèque nationale de France (5\% pour les livres). Le reste sera transformé en une base de données, non accessible en ligne, qui sera commercialisée via une filiale de la Bibliothèque nationale de France.

Cet embargo sur la mise en ligne sera maintenu durant une durée variable selon les corpus (7 ans pour les livres et la presse, 10 ans pour les fonds sonores) par le biais d’une exclusivité reconnue au partenaire commercial. À l’issue seulement de ce délai, les contenus numérisés pourront rejoindre Gallica en ligne.

Cette formule pourrait paraître constituer un compromis équilibré en permettant la numérisation de documents patrimoniaux et en répartissant les coûts importants entre le public et le privé. Mais ce n’est en réalité pas du tout le cas et ces montages violent des recommandations importantes faites au niveau international, si ce n’est la loi française~!

\section{Mépris des recommandations européennes}

Un Comité des Sages réuni par la Commission européenne avait publié en janvier 2011 une série de recommandations concernant les partenariats public-privé en matière de numérisation du patrimoine. Les sages européens ont recommandé que la durée des exclusivités accordées aux partenaires privés n’excède pas les 7 ans~:

\begin{quotation}
La période d’exclusivité ou d’usage préférentiel des œuvres numérisées dans le cadre d’un partenariat public-privé ne doit pas dépasser une durée de 7 ans. Une telle durée peut, en effet, être considérée comme pertinente pour, d’une part, générer  suffisamment d’incitation à la numérisation pour le partenaire privé et, d’autre part, garantir un contrôle suffisant des institutions culturelles sur les œuvres numérisées.
\end{quotation}

Il s’agit d’une durée maximale de 7 ans pour une exclusivité commerciale seulement, mais pas du tout une exclusivité sur la mise en ligne elle-même, car le texte du rapport indique formellement plus haut que les oeuvres du domaine public numérisées doivent être mises en ligne~:

\begin{quotation}
Afin de protéger les intérêts des institutions publiques qui concluraient un partenariat avec une entreprise privée, le Comité des  sages considère que certaines conditions doivent a minima être respectées~:
\begin{itemize}
\item Le contenu de l’accord entre une institution culturelle publique et son partenaire privé doit nécessairement être rendu public
\item \textbf{Les œuvres du domaine public ayant fait l’objet d’une numérisation dans le cadre de ce partenariat doivent être accessibles gratuitement dans tous les 
Etats membres de l’UE}
\item Le partenaire privé doit fournir à l’institution culturelle des fichiers numériques de qualité identique à ceux qu’il utilise pour son propre usage.
\end{itemize}
\end{quotation}

En 2009 la révélation d’un projet de numérisation à l’étude entre Google et la Bibliothèque nationale de France avait fait scandale. A l’époque, Google voulait numériser à ses frais les fonds en contrepartie de l’imposition d’une exclusivité commerciale de 25 ans, identique à celle qu’il avait imposée à la Bibliothèque municipale de Lyon.

Les accords conclus par Google ne comportent aucune exclusivité concernant la mise en ligne elle-même. Le but de Google était bien de mettre en ligne les oeuvres du domaine public qu’il numérisent, sur Google Books et à présent sur Google Play, où l’on trouve d’ailleurs déjà des livres de la Bibliothèque municipale de Lyon en téléchargement gratuit. Même si Google impose des restrictions, les bibliothèques partenaires sont aussi autorisées à mettre en ligne les ouvrages sur leur propre site.

La manière dont la Bibliothèque nationale de France va encapsuler des oeuvres du domaine public numérisées dans des bases de données coupées du web pour mieux les vendre est donc à tout prendre bien pire en terme d’atteinte au domaine public que ce qui était envisagée avec Google. Or la mobilisation du Grand Emprunt devait normalement permettre de trouver une solution plus satisfaisante que celle proposée par Google selon les préconisations du rapport Tessier. C’est exactement l’inverse qui s’est produit, car pour rembourser l'emprunt la Bibliothèque nationale de France s’est tournée vers un modèle économique qui passe par la marchandisation de la matière brute du domaine public lui-même, et donc par son anéantissement pur et simple.

\section{Une dégradation encore renforcée par la crise et les coupes budgétaires}

Il est évident que la période qui s’ouvre va constituer un risque majeur pour l’intégrité du domaine public sous forme numérique. Les fortes restrictions budgétaires annoncées par le Ministère de la Culture et le climat de crise économique ambiant vont nécessairement peser sur les capacités des institutions culturelles à numériser leurs collections par leurs propres moyens.

Il y a donc de fortes chances que des partenariats public-privé de plus en plus déséquilibrés en termes d’accès et de réutilisation des contenus soient conclus, afin de permettre aux institutions de dégager des ressources propres. Ces considérations économiques vont être très difficile à contrer.

Pourtant, cette optique de marchandisation du domaine public relève d’une bien courte perspective économique. Numériser et de rendre réutilisable le domaine public en ligne, y compris à des fins commerciales, peut constituer un effet de levier important sur de nombreux secteurs d’activité~: 

\begin{itemize}
\item Le domaine public numérisé peut être utilisé par les chercheurs et les enseignants.
\item Les consommateurs ont accès à une offre légale gratuite.
\item Le secteur marchand peut adapter ou rééditer ces œuvres dans le cadre de services et d’applications.
\end{itemize}

Cette richesse potentielle induite par la diffusion gratuite du domaine public ne pourra se déployer que si l’on met pas d’entrave à la réutilisation des oeuvres. Avant tout il faut continuer à les diffuser en ligne et non les réduire en produits commerciaux coupés du web.

À défaut, on aboutira à une abolition pure et simple du domaine public sous forme numérique, alors que la numérisation aurait dû permettre au contraire d’en réaliser toutes les promesses.

\chapter{L'intérêt du développeur et la satisfaction de l'administration}\label{devadmin}

\begin{quotation}
\emph{Les données libres sont la matière première de la nouvelle révolution industrielle}

Francis Maude, Ministre pour le Bureau du Cabinet en Angleterre
\end{quotation}
\section{La liberté et les données publiques}
Les données publiques ne sont vraiment utiles que lorsqu'elles sont disponibles et référencées dans un format brut et sous une licence libre.

Des données sont brutes lorsque chaque élément de la base de donnée peut être immédiatement isolé comme une case dans un tableur pour être traité informatiquement. Publier des PDFs qui contiennent des tableaux est mieux que rien mais n'est pas très utile car le PDFs est un format d'impression lisible surtout par des humains et non modifiable. Il n'est pas évident voire impossible pour un ordinateur de réorganiser de tels données automatiquement.

Elles sont libres quand elles sont toujours à jour et qu'il est possible de les télécharger et de les réutiliser gratuitement pour tout usage, y compris commercial. Elles ne doivent donc pas être protégées par des brevets et comme le droit d'auteur est par défaut restrictif, elles doivent être accompagnées d'une licence qui spécifie clairement les libertés accordées. Il ne doit pas non plus être nécessaire d'acheter des logiciels exorbitants pour pouvoir les lire et les réutiliser. Le moyen de diffusion le plus efficace est bien sûr Internet.

En outre, comme il est impossible de rassembler toutes les données existantes dans une seule base de données géantes, il est nécessaire de favoriser leur indexation par les moteurs de recherche voire de mettre soi-même à disposition du public un moteur de recherche.

Concrètement, des données publiques sont donc libres lorsque~:

\begin{itemize}
\item tout objet numérique élémentaire a un identifiant unique et peut être accessible et cherchable via Internet via des protocoles et formats standards pour un coût nul ou marginal
\item chaque base de donnée est sous une licence permettant sa réutilisation et sa diffusion par tout un chacun et pour tous les buts imaginables
\end{itemize}

\section{Les bases de données sont les nouvelles matières premières}

Dans tous les domaines, les données brutes ne sont pas un but en soi. Elles ont de la valeur si et seulement si elles permettent de mieux agir, maintenant ou à l'avenir. Plus les décisions que l'on prend grâce à elles sont nombreuses et utiles, plus les données elles-mêmes deviennent importantes. C'est donc leur usage qui leur donne de la valeur. Pour valoriser les données publiques, il faut favoriser la création de produits et de services les réutilisant. 

Beaucoup de bases de données ont déjà été produites ou numérisées par les administrations publiques. Les récréer ex nihilo prend énormément de temps et est un gachis de ressources, pourtant c'est parfois ce qui arrive. Par exemple le projet OpenStreetMap recartographie gratuitement sous des licences libres la France entière parce que les cartes produites par l'Institut national de l’information géographique et forestière (IGN) ne peuvent pas être réutilisées librement par les citoyens français. 

Dans le cas de l'IGN, la justification est que c'est à l'IGN de rentabiliser les cartes que les citoyens ont déjà payé avec leurs impôts et pas aux citoyens. Cependant, est-ce que cette exclusivité de l'utilisation commerciale par les administrations est vraiment la meilleure méthode pour créer de la richesse~? Nous allons tout de suite voir que non.

\section{La valeur économique des données libres}

Les données libres créent de la valeur de deux manières distinctes~:

\begin{itemize}
\item Elles permettent à l'administration d'économiser de l'argent en externalisant certains services.
\item Elles permettent à des entreprises de fournir des services performants et innovants à moindre coût.
\end{itemize}

Dans le secteur privé, avoir des données libres permet bien souvent de ne pas perdre de l'argent dans la négociation de licences coûteuses ou la longue et coûteuse création de bases de données. Cela facilite la création de nouveaux services par tout un chacun et favorise donc une saine concurrence tout en évitant aux citoyens de payer deux fois ou plus la création de ces bases de données. Comme beaucoup de services intéressants pour les citoyens sont avant tout locaux, de nouvelles opportunités d'emploi apparaissent. Cela permet aussi aux investisseurs étrangers de disposer facilement d'indicateurs leur permettant d'évaluer le sérieux des administrations locales ou de sonder l'état d'un potentiel marché plus rapidement.

Prenons par exemple le cas des données météorologiques. Dans son étude \livre{Public Information wants to be free}, James Boyle évalue à 9,5 milliards par an l'investissement public européen dans la météo et à 19 milliards celui américain. Selon lui, l'Europe en tire un bénéfice économique d'environ 68 milliards via l'amélioration des décisions des individus et des entreprises tandis que les États-Unis, qui ont décidé de complètement libérer ces données, en retirent 750 milliards par an, c'est-à-dire 39 fois plus.

Un tel taux de multiplication n'est pas seulement possible parce le secteur privé s'est emparé de ces données pour créer des services qui n'auraient sinon jamais existé, mais aussi parce que des volontaires ou des entreprises aident l'administration à mieux organiser ses propres données. Comme le remarque un rapport de l'institut londonienne de libération des données~: 

\begin{quotation}
Nous avons simplement lancé un appel à contribution sur Twitter pour nous aider à libérer les données publiques londoniennes et une petite armée de volontaires nous a aidé à améliorer la qualité de nos données. S'il y a une leçon à retenir, c'est qu'il faut utiliser l'expertise qui est déjà là, prête à aider.
\end{quotation}

Il faut enfin noter que la libération des données publiques créent des opportunités entrepreunariales même pour ceux qui n'ont pas accès à Internet. En Ouganda, la Grameen Foundation a développé la Question Box, une sorte de téléphone qui permet aux Ougandais d'avoir accès par le réseau mobile à de nombreuses informations médicales, agricoles ou éducatives. C'est une sorte de Google pour ceux qui n'ont pas Internet. En Europe, un tel outil pourrait intéresser ceux qui sont laissés de côté par la révolution du numérique à cause de leur manque de familiarité aux ordinateurs ou de leur mauvaise compréhension de la langue, comme les plus âgés ou certains nouveaux immigrants.  

\section{Améliorer l'administration par la libération des données}

Les pays et les villes ont sans cesse besoin de s'améliorer et de se mettre à jour. Ils doivent savoir quels services sont utiles aux citoyens, où et comment. La libération des données permet de satisfaire aux exigences de renouvellement d'une manière assez indolore, car elle permet de personnaliser à moindre coût les services aux usagers. Par exemple, la ville de Rennes a vu fleurir les applications permettant de faciliter l'utilisation des transports en commun dès qu'elle a libéré les plans et horaires des lignes de bus et métro.

Dans le modèle administratif classique, l'administration délivre le même service impersonnel à tous. Mais si l'administration ne fait plus que fournir des données co-créées avec les citoyens et des entreprises, alors les citoyens peuvent adapter à leurs propres besoins des services qui auraient été autrement standards. Cela favorise la participation citoyenne et la confiance dans l'administration et diminue les coûts pour l'administration comme pour l'usager tout en améliorant le service. Cette réalité est certes en partie cynique mais indéniablement elle est pragmatique.

Cependant il faut distinguer cette sorte de restructuration de l'action publique avec une privatisation classique. Dans les dérégulations habituelles, ce qui se passe est que le monopole de l'État est remplacé par un ou plusieurs monopoles privés. La libération des données publiques, au contraire, donne une chance égale à chacun de créer son propre service. Il existe donc une saine concurrence entre les acteurs et le risque de monopolisation d'activités autrefois publiques par des acteurs privés est nettement moins élevé.

\section{Les derniers freins à l'essor d'une nouvelle ère dans les administrations}

Les principales raisons pour lesquelles les données publiques ne sont pas aussi souvent libérées qu'elles devraient l'être sont les suivantes~:

\begin{itemize}
\item La pure et simple ignorance de l'importance du mouvement en cours. Beaucoup d'administrations n'ont toujours pas terminé de numériser correctement leurs données quand bien même les directives gouvernementales l'exigent.
\item Les craintes légales. Dans l'état actuel du droit d'auteur, le statut juridique par défaut de toute œuvre est «~Tous droits réservés~». Lorsque plusieurs acteurs ont participé ensemble à la création d'une base de donnée, il n'est pas toujours évident d'identifier qui est titulaire de quels droits ni de négocier la libération des données.
\item La crainte de publier des données de mauvaise qualité. Parfois les sources des données sont inconnues ou mal identifiées et les administrations ne savent pas elles-mêmes si celles-ci sont fiables ou non.
\item L'argent. Il existe un investissement initial pour libérer des données qui n'est pas toujours négligeable à cause des points ci-dessus. Qui plus est les administrations qui voient les retombées économiques positives de la libération des données, comme le Trésor Public ou les mairies, sont rarement exactement les mêmes que celles qui ont procédé aux investissements. 
\end{itemize}


\chapter{L'enthousiasme du chercheur}\label{cherch}

\section{La recherche de la vérité n'est pas celle du profit financier}
Tout le monde peut comprendre pourquoi le piratage de la musique peut être polémique. Les musiciens gagnent leur vie en vendant leurs œuvres. Partager des copies de leur travaux semble devoir les priver d'une source de revenus. Les pirates y répondent en faisant remarquer qu'il est moins rentable d'être inconnu que d'être piraté. Mais les chercheurs ne gagnent pas leur vie en vendant leurs articles scientifiques. Je n'entrerai donc pas dans le débat du piratage ici. 

Imaginons qu'un groupe d'auteurs ne gagnent pas leur vie en vendant leurs œuvres et en autorise le partage. Ce n'est pas parce qu'ils sont riches qu'ils suivent ce chemin étrange. C'est parce que leurs sujets d'intérêts, leurs motivations personnelles et les circonstances institutionelles les amènent à écrire pour être lus. L'important est l'impact, l'influence. Pas l'argent. Leurs carrières dépendent plus de la taille de leur lectorat que de l'existence de lecteurs payants.

Il n'y a pas beaucoup de journalistes ou de romanciers dans ce groupe. Mais c'est là qu'on trouve de plus en plus de chercheurs des sciences dures ou humaines.

Les chercheurs ne gagnent pas leur vie en vendant leurs articles à des revues académiques où ceux-ci sont vérifiés par leurs pairs. La plupart du temps, ces revues ne versent aucune redevance aux auteurs, mais leur font au contraire payer pour que leur article soit examiné. S'il arrive que les chercheurs veuillent gagner de l'argent en écrivant des livres,  ce n'est clairement pas le cas des articles de revues. De plus en plus de chercheurs ont compris cela et encouragent le partage de leurs articles publiés. 

La plupart des chercheurs travaillent dans des institutions qui les payent pour chercher. Les salaires que celles-ci leur payent leur permettent tout d'abord de se nourrir. Ils leur permettent aussi de ne pas être dépendant du marché du livre. Des musiciens qui n'enregistreraient que quelques morceaux tous les ans qu'ils n'enverraient qu'à quelques amis sélectionnés sur le volet mourraient de faim. Des chercheurs qui peaufinent pendant deux mois un seul article sur un sujet inexploré et inconnu continue à vivre. Ces chercheurs peuvent défendre des opinions peu populaires ou s'intéresser à des sujets spécialisés sans dommage financier. Ils peuvent s'intéresser à ce qui leur paraît vrai sans se préoccuper de savoir si ça se vend.

Ce système fonctionne très bien comme cela. Si les universités ne permettaient pas aux chercheurs de vivre décemment, ceux-ci ne défendraient que des opinions populaires. La recherche de la vérité serait remplacée par la recherche du profit. 

\section{La libre diffusion et réutilisation permet une meilleure reconnaissance}
Les premières revues scientifiques qui furent lancées à Paris et Londres il y a 350 ans supplantèrent rapidement les lettres et les livres utilisés jusque là pour diffuser la recherche. Les revues permettaient d'atteindre un public plus large et étaient imprimées plus vite que les livres. La réputation de certaines revues a permis de hiérarchiser les chercheurs entre eux. C'étaient les chercheurs qui étaient en demande de publication. Cela explique que les revues ne leur payent rien. 

Le système a bien fonctionner jusqu'à ce que le prix des revues finisse par augmenter plus vite que l'inflation à partir de 1970. Ces prix n'ont cessé d'augmenter exponentiellement depuis quarante ans. Même les universités les plus riches sont obligées d'annuler certains abonnements chaque année pour raisons budgétaires. Harvard par exemple a récemment demandé à ses facultés de limiter le nombre de leurs abonnements à cause des prix démentiels. Ainsi Elsevier, un éditeur majeur du marché, a déclaré une marge de 36\% en 2010, alors qu'ExxonMobil ne déclarait \emph{que} 28\%.  Certains prennent cela pour une crise du marché des revues mais c'est bien plutôt une crise de l'accès à la recherche.

C'est pourquoi en 2002 l'Initiative pour le Libre Accès de Budapest a donné un nom au partage libre de la recherche~: Accès libre. Elle a décrit quelques stratégies pouvaient être poursuivies pour améliorer l'accès libre. Voici la définition de l'accès libre selon cette initiative~:


\begin{quotation}Par «~accès libre~» aux articles de revues académiques, nous entendons sa mise à disposition gratuite sur l'Internet public, permettant à tout un chacun de lire, télécharger, copier, transmettre, imprimer, chercher ou faire un lien vers le texte intégral de ces articles, les disséquer pour les indexer, s'en servir de données pour un logiciel ou s'en servir à toute autre fin légale, sans barrière financière, légale ou technique autre que celles indissociables de l'accès et l'utilisation d'Internet. La seule contrainte sur la reproduction et la distribution et le seul rôle du droit d'auteur dans ce domaine devraient être de garantir aux auteurs un contrôle sur l'intégrité de leurs travaux et le droit à être correctement reconnus et cités.
\end{quotation}

En un mot, la seule obligation pour l'utilisation d'un article en accès libre est de citer son auteur. 

Pourquoi donc cette restriction~? L'objectif de l'accès libre est d'abattre les barrières à l'utilisation académique de la littérature académique sans porter préjudice aux chercheurs. Académiquement, réutiliser les travaux des autres sans les citer n'est pas honnête. Les carrières et influences des auteurs dépendent de la bonne attribution des bons travaux aux bonnes personnes. 

Inversement, quel est le but de la suppression des restrictions classiquement imposées par le droit d'auteur au nom du bien de ces auteurs~? La réponse est que partager la connaissance accélère la recherche. Cela aide les chercheurs à trouver les travaux dont ils ont besoin pour progresser et à se faire connaître pour leurs propres travaux. La connaissance a toujours été un bien commun dans le sens théorique du terme~: c'est un bien idéel et non-rival qui peut être consommé sans dommage par tous. Une politique du libre n'est que la traduction concrète de cette idée. 

Ce sont les auteurs qui décident de mettre leurs livres en accès libre. Nous chercheurs n'attendons pas des musiciens et romanciers qu'ils décident de mettre leurs œuvres sous licence libre, même si certains acceptent de céder leurs droits de temps à autre. Ce qui nous intéresse sont les travaux académiques et notre salaire ne dépend pas de la vente de nos articles. 

\section{Les œuvres en accès libre sont en croissance exponentielle}

Il y a deux manières répandues de promouvoir l'accès libre. Soit via les revues elles-même (accès doré dans le jargon), soit via des sites de dépôt (accès vert). 

Les revues d'accès libre se financent de multiples manières, car bien sûr accès libre ne veut pas dire édition gratuite. Les éditeurs sont parfois à but lucratif et parfois non. Certains utilisent une méthode de revue par les pairs classique mais d'autres sont plus innovants. Le répertoire des revues en accès libre compte à présent plus de 8000 revues en accès libre.

Les sites de dépôt sont des simples bases de données de contenu numérique. En France, le site HAL héberge par exemple plus de 220~000 travaux. Certains sites ne diffusent qu'un seul champ, d'autres sont multi-disciplinaires. Certains sont ouverts à tous les chercheurs, d'autres ne publient que les travaux d'une université ou d'un centre de recherche. Plus de 60\% des revues donnent leur accord pour que les articles validés soient déposés dans de tels sites. Ces sites peuvent par ailleurs héberger des images, des présentations, de l'audio, du code source de logiciel, etc… Il en existe plus de 2100 actuellement.

30\% des revues modérées actuelles sont en accès libre. C'est énorme par rapport à il y a dix ans et le taux de conversion des indécis décolle. Mais c'est toujours une petite partie de l'ensemble de l'offre de revues. C'est pourquoi imposer aux chercheurs de publier dans des revues en accès libre leur limiterait nettement le choix.

Cependant les sites d'accès libre vert sont en pleine expansion. La plupart des universités encouragent cette pratique si elles n'y obligent pas. À Harvard ou au MIT, non seulement la faculté oblige à ce dépôt mais impose que tous les travaux à venir le soient et préempte ainsi les contrats des éditeurs. Ainsi les politiques universitaires peuvent obliger à des changements massifs de comportement. Il existe à présent plus de 150 dépôts de travaux en accès libre purement universitaires dans le monde. 

Certains pourraient croire que cette passion pour la publication en accès libre a été imposée par les administrations centrales dans les universités. Rien n'est plus faux. Au MIT et à Harvard, toutes les facultés ont voté à l'unanimité les politiques de ces facultés. 

Les politiques nationales de certains pays comme l'Angleterre ou le Danemark sont plus avancées. Le Danemark a purement et simplement imposé que tous les travaux de recherche financés par l'argent public soient diffusés en accès libre. En Angleterre cela ne saurait tarder, car le Higher Education Funding Council for England a annoncé que tous les travaux publiés dans le cadre du prochain Research Excellence Framework de 2014 seront en accès libre. La Commission Européenne a annoncé que les travaux qu'elle financeraient serviront une politique d'accès libre et a appelé les États-membres à faire de même.

La plupart des gens qui surfent sur Internet ne réalisent pas que ce réseau a initialement été développé par des chercheurs pour partager leurs recherches mais que cet objectif a été laissé de côté avec l'arrivée du commerce. Il n'a cependant pas été laissé de côté et l'accès libre aux publications scientifiques est en train de devenir la norme des communications académiques. 

La prochaine étape de ce processus, déjà en cours, sera la réutilisation de tous ces travaux de recherche nouvellement publiés à des fins industrielles. Les produits et services innovants de l'avenir sont déjà souvent construits à partir de recherches financés par l'État. Augmenter le nombre de travaux librement disponibles pour le grand public grâce à Internet donnera une nouvelle dynamique à ce cercle vertueux. Plus que jamais, la recherche publique remplira alors complètement ses buts~: améliorer nos connaissances et grâce à celles-ci améliorer notre qualité de vie.



\chapter{Le commentaire de l'historien}

\section{15\ieme siècle~: L'invention de l'imprimerie met fin aux moines-copistes}

\subsection{Le prix du livre}
Commençons avec la Peste Noire dans les années 1350 en Europe. L'Europe fut durement touchée, comme le reste. Elle mit 150 ans à s'en remettre. Les gens fuyaient l'Empire Byzantin et amenaient en
Europe la peste.

Les institutions religieuses furent les plus lentes à récupérer. Elles ne furent pas frappées durement seulement parce que les moines vivaient dans des environnements confinés, mais aussi parce qu'ils
ne se reproduisaient pas eux-mêmes, et que les parents manquaient à présent de main d'œuvre.

À l'époque, si vous vouliez un livre, il fallait le demander à un moine. Qui le recopierait à la main.
Aucune copie n'était parfaite. Les moines corrigeaient des fautes tout en en introduisant d'autres.

Seuls certains livres étaient publiés. Non seulement parce que le prix des matières premières était exorbitant (170 peaux de veau ou 300 de mouton)
mais aussi parce que l'Église n'autorisait pas les livres qui contrariaient sa doctrine.

En 1450, les monastères n'étaient toujours pas repeuplés et copier un livre coûtait très cher à cause des matières premières et du manque de main d'œuvre. 

En 1451, Gutenberg parfit la
presse à imprimer, avec des caractères en métal amovible, une impression à base d'encre grasse et de blocs en bois. En même temps, un nouveau type de papier bon marché à fabriquer et résistant est
copié des Chinois. Dans les décennies qui suivirent, les moines-scribes devinrent obsolètes.

\emph{L'invention de la presse a révolutionné la société en permettant de transmettre rapidement, facilement et efficacement l'information.}
\subsection{L'Église perd le contrôle de l'information}
L'Église catholique, qui contrôlait auparavant totalement la diffusion de l'information et avait fait un marché de niche de cette rareté de l'information, éructa. Elle ne contrôlait plus ni
l'information ni les esprits et fit pression sur les rois pour bannir cette technologie.

De nombreux arguments furent inventés à l'occasion pour rétablir l'ancien régime. L'un d'entre eux était \begin{quotation}
     «~Comment paierons-nous les moines-copistes~?~»                                                                                                    
                                                                                                        \end{quotation}

Finalement, l'Église catholique ne réussit pas à empêcher la propagation de l'imprimerie,  laissant la voie libre à la Renaissance et au protestantisme, mais beaucoup de sang coulat pour empêcher la
circulation rapide et efficace des idées, de la culture et de la connaissance.

L'aboutissement des efforts de l'Église peut être marqué par la parution d'une loi en France le 13 janvier 1535 qui ordonnait la fermeture de toutes les librairies et condamnait à mort par pendaison
quiconque utilisait une imprimerie.

Cette loi fut largement inefficace. Les frontières du pays fourmillèrent de librairies et la littérature pirate se répandit en France via des contrebandiers ravitaillant les gens normaux en quête de
nouvelles choses à lire.

\section{17\ieme siècle~: Marie la Sanguinaire censure grâce au droit de copie}
\subsection{Une bâtarde répudiée}
Le 23 mai 1533, la fille de 17 ans qui serait Marie 1\ieme d'Angleterre fut officiellement déclarée bâtarde par un archevêque protestant. Sa mère, Catherine, qui était catholique et une protégée du
Pape,
avait
été mise à la porte par son père Henri, qui s'était converti au protestantisme pour se débarrasser d'elle. Marie tenterait de redresser cette injustice toute sa vie.

Le roi Henri VIII voulait un fils pour lui léguer le trône d'Angleterre mais son mariage n'avait pas réussi. Sa femme Catherine d'Aragon ne lui avait donné qu'une fille. Pire, le Pape ne voulait
pas le laisser divorcer.

La solution d'Henri fut assez radicale mais innovante. Il convertit toute l'Angleterre au protestantisme et fonda l'Église d'Angleterre afin de renier le Pape. Il fit déclarer son mariage nul le 23
mai 1533 puis se maria à plusieurs femmes par la suite. Il eut une deuxième fille avec sa seconde femme puis enfin un garçon avec la troisième. La demi-sœur et le demi-frère de Marie étaient
protestants. 

Édouard succéda à Henri VIII en 1547, à neuf ans. Il mourut avant d'atteindre l'âge adulte. Marie était deuxième dans l'ordre de succession, même si elle était une bâtarde. Elle devint donc reine en
1553.

Elle n'avait pas parlé à son père pendant des années. Elle voulut rendre l'Angleterre au catholicisme. Elle persécuta sans relâche les protestants en en exécutant publiquement des centaines et
s'acquit le surnom de Marie la Sanguinaire\footnote{Bloody Mary en anglais}.
\subsection{Rétablissement du catholicisme}
Marie Tudor partageait les préoccupations de l'Église catholique à propos de la presse. Que le grand public puisse faire rapidement circuler l'information était dangereux pour le rétablissement du
catholicisme à cause de l'existence de textes hérétiques. Vu que la France avait misérablement échoué à bannir l'imprimerie, même en menaçant les contrevenants de pendaison, elle chercha une autre
solution. Une qui serait aussi autant bénéfique pour l'industrie de l'imprimerie que pour elle.

Elle accorda un monopole à la corporation des imprimeurs de Londres en échange du contrôle des ouvrages imprimés. Ce fut un accord très lucratif pour la corporation, qui travailla dur pour maintenir
le monopole en place. 

Cette coopération des pouvoirs corporatistes et gouvernementaux fut très efficace pour réduire la liberté d'expression et étouffer les dissensions politico-religieuses. Aussi
longtemps que rien de politiquement dérangeant ne circulait, tous les divertissements étaient autorisés. L'accord était gagnant-gagnant.

Le monopole fut accordé le 4 mai 1557 à la Compagnie londonienne des Libraires. Il fut appelé copyright (droit de copie en français).

Les Libraires travaillait comme un bureau de censure privée en brûlant les livres interdits, en détruisant les presses clandestines et en empêchant la diffusion de tous les textes politiquement
incorrects. Peu d'affaires remontaient à la Reine. Après quelques consultations, les censeurs savaient ce qu'il fallait censurer.

Marie Tudor mourut en 1558. Sa sœur protestante Élisabeth lui succéda. La tentative de restauration du catholicisme par Marie échoua mais l'invention du copyright lui survécut. 

\subsection{Le droit de copie se maintient grâce aux éditeurs}
À la mort de Marie, ni la Couronne ni la corporation ne désiraient abolir le copyright. L'accord dura 138 ans sans interruption. Élisabeth se servit de cet accord pour empêcher la propagation des
matériaux catholiques.

Pendant le 17\ieme siècle, le Parlement essaya de prendre progressivement le contrôle de la censure des mains de la Couronne. 

En 1641, il abolit dans la foulée du vote de
l'Habeas Corpus l'infâme Chambre étoilée, véritable tribunal d'inquisition qui jugeait les
cas
d'infractions
à la censure. Cela rendit ces infractions inoffensives, de même que la traversée en dehors des clous est tout à fait tolérée concrètement parlant. Dès lors, la créativité en Grande-Bretagne fleurit.

Malheureusement, ce n'était pas du tout ce que le Parlement avait prévu.

En 1643, le monopole du copyright fut restauré par le Licensing Order of 1643 avec une subtilité supplémentaire. Tous
les auteurs, imprimeurs et éditeurs devaient s'enregistrer auprès des Libraires pour demander une licence
d'exercice et de publication pour chaque ouvrage. Les Libraires avaient de plus le droit de détruire les presses et les livres non autorisés, et d'infliger de sévères sanctions aux contrevenants.

Accélérons un peu. Il y eut la Glorieuse Révolution en 1688, et la composition du Parlement changea radicalement. En faisaient désormais majoritairement partie des gens qui avaient été du mauvais côté de la
censure et n'avaient aucune envie de la voir continuer. Le monopole des Libraires fut abrogé en 1695.

À partir de 1695, il n'y eut plus de copyright. La créativité fleurit de nouveau, et les historiens soutiennent que les documents publiés dans ce vent de liberté menèrent à la fondation des
États-unis.

Malheureusement, les Libraires étaient peu ravis d'avoir perdu leur travail et leur monopole lucratif. Ils rassemblèrent leurs familles devant le Parlement en demandant la restauration de leur
monopole.

Le Parlement, qui venait juste d'abolir la censure, n'avait aucune envie de la restaurer. Les Libraires suggérèrent donc que les auteurs devraient «~posséder~» leurs travaux. Ainsi, ils
faisaient d'une pierre trois coups. 

\begin{enumerate}
\item 
Le Parlement s'assurait qu'il n'y aurait pas de censure centralisée. \item
Les éditeurs gardaient le monopole de leurs impressions, et les
auteurs ne pouvaient publier sans eux. \item
Le monopole entrait dans la Common Law, ce qui lui offrait des protections plus fortes que s'il faisait partie des lois purement
jurisprudentielles.\end{enumerate}

Le lobby des éditeurs obtint gain de cause et le nouveau monopole du droit de copie parut en 1709 pour prendre effet au début 1710. Voilà la première grande victoire du lobby de l'édition.

Dès lors, les Libraires continuèrent à détruire et brûler les travaux des autres presses pendant longtemps même si légalement ils n'en avaient plus le droit. Cela dura jusqu'au jugement Entick vs.
Carrington en 1765, qui concernait un de ces raids contre les auteurs «~non licenciés~» (c'est-à-dire non désirés).
Les juges décidèrent que la poursuite des presses et auteurs illégaux était du
ressort de l'État, non des Libraires.

Ce qu'il faut retenir ici est que le copyright n'a jamais été inventé pour profiter aux auteurs. Ce sont des justifications a posteriori. Le copyright bénéficiait avant tout aux censeurs, aux éditeurs
et aux imprimeurs. Personne ne défendit que le copyright était essentiel à
l'écriture.

Rien de nouveau sous le soleil donc. Les toutes premières fondations de la démocratie en Europe de l'Ouest devaient déjà se battre contre les monopoles de l'édition.

\section{19\ieme siècle~: De Jefferson à Hugo}
\subsection{Le copyright américain sert le progrès de la science et des arts utiles}
\begin{quotation}
 Lire sans payer~? C'est du vol !
\end{quotation}

À la fondation des États-Unis, le concept de monopole sur les idées fut apporté dans le nouveau monde, et intensément débattu. Thomas Jefferson s'opposa
fermement à ce monstre~:
\begin{quotation}
Si la nature a rendu une chose moins susceptible que les autres de propriété exclusive, c'est l'action de penser appelée idée, qu'un individu peut posséder pour autant qu'il la garde pour lui-même,
mais qui ne lui appartient plus dès qu'elle est divulguée aux autres, sans que les autres puissent s'en débarrasser. Ce trait tout à fait particulier des idées fait que l'on ne les possède pas moins
si les autres les possèdent, parce que tous les possèdent entièrement. Instruire quelqu'un ne diminue pas mon instruction. Il est illuminé sans me faire pour autant de l'ombre. Que les idées puissent
circuler librement d'un point à l'autre du globe pour l'instruction morale mutuelle des hommes et pour l'amélioration de leur condition semble avoir été volontairement conçu par la nature quand
elle nous a rendu … incapable de les enfermer ou de nous les approprier.
\end{quotation}

Un compromis fut cependant conclu. Les États-Unis ont été les premiers à considérer qu'il y avait une raison à l'existence du copyright et des brevets. Cette raison est très lapidaire dans la loi
américaine~:

\begin{quotation}
 …pour promouvoir le progrès des sciences et des arts utiles…

Article I, Section 8, Clause 8 de la Constitution américaine, dite clause du copyright
\end{quotation}

Il doit être noté que l'intention du monopole n'est pas qu'une profession gagne sa vie, comme celle des distributeurs ou des imprimeurs. Elle est exemplaire dans sa clarté~: la seule justification du
monopole est \emph{la maximisation de la culture et de la connaissance disponibles dans la société}.

Le copyright américain découle donc d'un équilibre entre l'accès du public à la culture et l'intérêt de ce même public à la création de la culture. C'est essentiel. Le public
est la seule aune de l'intérêt du copyright. 

Les détenteurs de monopole, bien que bénéficiant eux aussi du copyright, ne sont pas des intéressés légitimes, et n'ont pas leur mot à dire dans
l'interprétation de la loi, de même qu'un régiment obéit mais ne dicte pas la politique de sécurité nationale. 

Ce point doit être souligné. Beaucoup croient que la Constitution des États-Unis justifie l'existence d'un monopole du droit de copie
pour que les artistes puissent gagner leur vie. Littéralement
parlant, là n'est pas la question ou l'intérêt du copyright.

\subsection{Invention des bibliothèques publiques en Angleterre}
Pendant ce temps au Royaume-Uni, les livres devenaient relativement chers à cause du copyright. Seuls les riches pouvaient collectionner les livres, et certains commencèrent à en louer à d'autres.

Les éditeurs devinrent furieux en s'en rendant compte et firent pression sur le Parlement pour rendre illégale la lecture d'un livre sans en avoir payé son propre exemplaire. Ils essayèrent de rendre
illégales les bibliothèques publiques avant même que celles-ci soient inventées.

Mais le Parlement ne suivit pas leur avis en voyant que la diffusion des livres était bénéfique pour le public. Pour le Parlement, le vrai problème était surtout que les riches pouvaient maintenant
contrôler ce que les pauvres lisaient et qui lisait. Il décida donc de créer des bibliothèques accessibles au public, à tous les publics.

Les défenseurs du copyright fulminèrent en entendant l'idée. \begin{quotation}
  «~Plus personne ne pourra vivre de ses écrits ! Plus personne n'écrira de livre ! Plus aucun livre ne se vendra !~»                                                            
                                                             \end{quotation}
                                                             
Cependant, le Parlement anglais de l'époque fut bien plus avisé que ne le sont les parlementaires européens d'aujourd'hui et prit le pétage de plomb des monopolistes du droit de copie pour ce qu'il
était. En 1849, une loi instituant l'existence de bibliothèques publiques fut passée et la première bibliothèque ouvrit en 1850.

Bien sûr, depuis, aucun livre n'a plus été écrit. Ou alors, les tirades des monopolistes du copyright étaient injustifiées et tout aussi fausses que leurs tirades actuelles.

On peut même noter que dans certains pays européens, depuis le début du 20\ieme siècle, certaines rentes versées aux auteurs et traducteurs sont indexées sur leur succès en bibliothèque.

\subsection{L'industrialisation allemande dépasse l'anglaise}
À la même période, le droit d'auteur ou le droit de copie n'existaient pas en Allemagne. Plusieurs historiens soutiennent
que c'est cette différence qui a expliqué que
l'Allemagne a si rapidement
rattrapé et dépassé industriellement le Royaume Uni. La connaissance se répandait vite et bien. D'une certaine manière, le succès allemand prouvait que les vues de Jefferson ou du Parlement anglais
étaient correctes~: le public a intérêt à la diffusion rapide de la culture.

\subsection{Les droits moraux se répandent sur le continent}
À la fin du 19\ieme, siècle, les renforcements du droit de copie exigés par les éditeurs avaient enterré les possibilités pour les créateurs de gagner leur vie grâce
à leur travail. Basiquement, l'argent allait aux distributeurs et aux éditeurs, et les créateurs vivotaient sans le sou (tout comme aujourd'hui). 

En France, une personne nommée Victor Hugo essaya d'équilibrer les règles du jeu en internationalisant une tradition française nommée droit d'auteur et en l'incluant dans le droit de
copie.

Il essaya aussi d'internationaliser le droit de copie même si l'époque était plutôt aux lois de libéralisation des marchés à travers l'Europe. À l'époque, les droits
restaient nationaux. Un écrivain
français vendait son monopole à un éditeur français mais les éditeurs allemands et anglais faisaient ce qu'ils voulaient de son œuvre.

Curieusement, les monopoles des brevets et du droit de copie furent oubliés dans le mouvement de libéralisation des marchés pour empêcher la «~concurrence déloyale~», réminiscence de l'époque
où les corporations dictaient les prix des produits et les salaires. Aujourd'hui encore, les poursuites judiciaires à coup de brevets rappellent l'époque où les corporations
vandalisaient les commerces non autorisés.

L'idée de Victor Hugo était de compenser les pouvoirs exorbitants des éditeurs en accordant des droits inaliénables aux créateurs, avec l'effet de bord de continuer à priver le public de ses droits.

Il y réussit partiellement, même s'il ne vit pas son succès de son vivant, avec la signature de la convention de Berne en 1886. Cette convention dit que les pays doivent mutuellement respecter les
droits d'auteur et de copie établis dans les autres pays et instaure une autorité de surveillance, la BIRPI (Bureaux internationaux réunis pour la protection de la propriété intellectuelle). Cette
agence est devenue l'Organisation mondiale de la propriété intellectuelle ou OMPI et la convention de Berne a déjà été
plusieurs
fois réécrite depuis 1886.

On peut noter que, dans toutes ses péripéties, les principaux perdants ont toujours été le public et les
créateurs, le plus souvent mal organisés et représentés lors de l'élaboration des lois, à
l'inverse des éditeurs et distributeurs.

\section{Les quatre droits essentiels d'un auteur}
\begin{enumerate}
\item 
\textbf{Le monopole commercial de la reproduction d'une œuvre}. C'est le monopole originellement accordé à la corporation des imprimeurs londoniens pour organiser la censure.
\item
\textbf{Le monopole commercial de la représentation vivante d'une œuvre}. Si quelqu'un met en scène une œuvre pour en tirer de l'argent, le détenteur du monopole a le droit de demander une
rétribution.
\item
\textbf{Le droit moral de la reconnaissance comme auteur}, afin de protéger contre la contrefaçon ou le plagiat.
\item
\textbf{Le droit moral d'interdire une réinterprétation de son œuvre}. Si un artiste pense qu'une mise en scène détruit son œuvre ou son image, il peut refuser sa diffusion.
\end{enumerate}

Les droits moraux, issus du droit d'auteur français défendu par Victor Hugo à travers l'Europe, sont très différents des monopoles commerciaux en ce qu'ils ne sont pas transférables. Ils ne sont pas
du tout justifiés par les raisons données par le Parlement britannique en 1709. 

Il est aussi notable que ces quatre aspects sont souvent délibérément confondus pour défendre le monopole le plus dommageable à la société, le monopole commercial sur la reproduction. 

Vous
entendrez souvent des gens de l'industrie du droit de copie défendre leur monopole en demandant «~Voudriez-vous que quelqu'un prenne votre travail et se l'attribue~?~». Malheureusement, ce troisième
point peu controversé, le droit moral à l'attribution et au crédit, n'a rien à voir avec les monopoles d'exploitation commerciale.

Les États-Unis ne voulurent pas reconnaître les droits moraux et restèrent donc en dehors de la convention de Berne jusqu'à ce qu'elle se révèle utile pour contrer Toyota un siècle plus tard. Nous y
reviendrons.

\section{Années 1930~: l'industrie musicale entre dans la danse}
Pendant la plus grande partie du vingtième siècle, les conflits sur les droits d'auteur et de copie se focalisèrent sur la musique, non sur le livre. La bataille fit rage entre les musiciens et leurs
labels. Au début du vingtième siècle, les musiciens étaient
regardés à juste titre comme les plus légitimes dans ce conflit. Cependant, l'industrie de la musique réussit à mettre sous sa coupe l'essentiel de la production musicale. Cela commença en Italie. 

Dans les années 1930, beaucoup de musiciens perdirent leur emploi à cause de la crise économique et de l'arrivée du cinéma parlant.

Dans cet environnement, deux initiatives furent prises parallèlement. 

De leur côté, les syndicats de musiciens essayèrent de garantir des revenus stables à leurs chômeurs tout en réglementant l'arrivée de la «~musique mécanique~», c'est-à-dire de celle qui
ne demande
pas de musiciens en chair et en os pour être reproduite. La question fut soulevée dans une assemblée de l'Organisation internationale des travailleurs de l'époque.

En même temps, l'industrie musicale essaya aussi de contrôler cette «~reproduction technique~» de la musique en contrôlant la radio et les musiciens. Cependant, le monde politique et économique de
l'époque les regardait comme des intermédiaires vers les musiciens. En faire plus était essayer de courir la faillite vu qu'ils n'étaient pas plus importants que quiconque, sauf pour Mussolini.

\subsection{L'Italie fasciste et la naissance de l'IFPI}

En 1933, l'industrie du phonographe fut invitée à Rome par la  Confederazione Generale Fascista dell’Industria Italiana. Lors d'une conférence entre le 10 et le 14 novembre, la fédération
internationale de l'industrie phonographique fut fondée. Elle serait plus tard connue sous le nom d'IFPI. Il fut conclu que l'IFPI essayerait d'obtenir les mêmes droits pour les producteurs que pour
les musiciens et les artistes qui vendaient habituellement leurs droits aux éditeurs et aux diffuseurs.

L'IFPI continua à se rencontrer dans un pays qui favorisait son agenda corporatiste, c'est-à-dire l'Italie, comme l'année suivante à Stresa en 1935. L'année 1935 et les suivantes furent chaotiques
en Europe mais l'Italie mit en place les droits corporatistes que l'IFPI réclamait dès 1937.

Essayer de négocier un monopole attaché à la convention internationale de Berne et du même style que le droit de copie possédé normalement par les auteurs mais pour les producteurs était trop tentant
pour l'IFPI. 

Vu que l'Italie de 1950 leur était politiquement hostile, elle se réunit dans le Portugal para-fasciste d'alors. La conférence mit sur pied un projet qui donnerait aux producteurs des droits
identiques à ceux du droit de copie, appelés «~droits voisins~», sur les reproductions techniques ou vivantes des œuvres. Ce monopole serait à peu près
identique à celui du monopole
commercial du droit de copie.

Les droits voisins furent ratifiés par l'Organisation mondiale de la propriété intellectuelle d'alors en 1961 dans la convention de Rome, qui donna aux producteurs et aux artistes-interprètes des droits beaucoup
plus importants qu'avant.

Depuis 1961 l'IFPI défend corps et âme le droit de copie même si les producteurs eux-même ne jouissent pas directement du monopole qu'il confère. Ils jouissent seulement des dits «~droits voisins~».

\subsection{Lumières et puissance financière}
L'industrie du divertissement confond tous ces monopoles sur la reproduction des œuvres à dessein. Elle défend «~son droit de copie~» qu'elle n'a en fait pas. Elle parle avec
nostalgie d'une tradition qui remonterait à l'époque des Lumières et aurait été créée par la sagesse des anciens de cette époque [insérez les chatons ici] . Ces monopoles n'ont été créés qu'en Europe
en 1961 et ont toujours été polémiques. Ils n'ont jamais été produits par la sagesse des Lumières.

Si les musiciens avaient été les seuls à réussir à faire reconnaître leurs droits sur les copies enregistrées de leurs performances, les producteurs seraient restés des bureaux
d'intermédiaires comme ils
l'avaient toujours été. Cela aurait été le cas si les gouvernements fascistes ne s'étaient pas mêlés de l'affaire entre temps.

\section{1980~: deuxième piratage des droits de copie et d'auteur, par Pfizer}
La dernière partie du vingtième siècle est marquée par deux faits. 

D'un côté les labels s'effrayent de la possibilité qui s'ouvre au grand public de copier pour des usages non-commerciaux leurs
œuvres même si les droits d'auteur et de copie ont toujours visé les usages commerciaux. Au point de défendre que la copie privée devrait être illégale ou de remettre en cause le droit à la vie
privée pour surveiller nos usages. 

De l'autre côté les monopoles conférés par les droits de copie continuent à s'étendre et à forger le monde. 

Vu que les atteintes à la vie privée sont à la une des
journaux nous nous intéressons à l'expansion des droits accordés par le législateur.
\subsection{Les invasions barbares}


Toyota a touché le cœur de l'âme américaine dans les années 1970 en envahissant le marché américain. Les voitures américaines -~les voitures~! des voitures américaines~!~- n'étaient pas assez
performantes pour les 
Américains eux-mêmes. Tout le monde achetait Toyota. L'apocalypse approchait. Les États-Unis étaient finis, impuissants face à la compétitivité asiatique.

Lorsqu'il est devenu clair que les États-Unis n'étaient plus industriellement concurrentiels et ne pourraient continuer leur domination économique par ce biais, de nombreuses commissions furent formées
pour
répondre à cette question cruciale~: Comment maintenir la domination économique américaine tout en ne produisant rien de concurrentiellement valable~?

\subsection{La réponse pharmaceutique}
La réponse vint d'un endroit inattendu~: Pfizer.

Le président de Pfizer, Edmun Pratt, écrivit un édito fulminant dans le \livre{New York Times} du 9 juin 1982~: «~Voler les esprits~». 

Ce qui le faisait enrager était le vol de propriété
intellectuelle
par le tiers-monde. Il voulait dire par là qu'il lui était inconcevable que les pays du tiers-monde utilisent leurs connaissances locales et leur temps pour produire avec leurs matières premières dans
leurs usines des médicaments pour soigner leurs peuples, qui mourraient fréquemment de maladies bénignes chez nous mais fatales dans les conditions de vie locales. 

Les politiciens y virent une réponse
à leurs interrogations sur l'avenir des États-Unis, et Pratt alla diriger une nouvelle commission. Cette commission fut la magique ACTN~: Commission de conseil sur les négociations commerciales (Advisory Committee on Trade Negotiations).

Ce que l'ACTN recommanda, en suivant la direction de Pfizer, était si provocant que personne n'était tout à fait sûr de la nécessité de son application. Les États-Unis devraient mener de front les
négociations commerciales et la politique extérieure afin de stigmatiser les pays qui n'obéissaient pas aux impératifs du «~libre échange~» à l'américaine, par exemple en étant inscrit sur la liste
de surveillance «~Special 301~». Cette liste serait une liste de pays qui ne respecteraient pas assez les lois sur le copyright. La plupart des pays du monde, Canada inclus, seraient trop tolérants aux
infractions au copyright pour les Américains.

La solution à l'absence de production de valeur ajoutée fut donc la redéfinition des termes «~absence~», «~production~» et «~valeur~» via la diplomatie internationale, au forceps. 

Ça a marché.
Les représentants du commerce extérieur des États-Unis, en faisant diplomatiquement et économiquement pression sur les gouvernements étrangers pour mettre en place des législations favorisant les
intérêts industriels américains et en établissant tout une palanquée d'accords bilatéraux et multilatéraux de «~libre échange~» aux mêmes fins, firent mieux respecter les intérêts américains dans le
monde entier.

De cette manière, les États-Unis pouvaient créer de la valeur en louant des plans de construction et en récupérant des produits finis selon ces plans. C'était à présent un marché équitable selon les
accords de «~libre échange~» qui redéfinissaient artificiellement ce que valeur veut dire.

Les industries américaines des monopoles par droit de copie ou de brevet étaient derrière ce putsch. Ils allèrent tous ensemble à l'Organisation mondiale de la propriété intellectuelle pour y répéter
l'abordage réussi par l'industrie de la musique en 1961, pour y recueillir légitimité et hospitalité sous l'égide d'une nouvelle convention appelée «~Berne Plus~».

\subsection{Naissance de l'OMC}
À un moment, il devint nécessaire aux États-Unis de rejoindre la convention de Berne, puisque l'OMPI surveille l'application de la convention.

Cependant, l'OMPI comprit le plan et les mit plus ou moins dehors. L'OMPI n'a pas été créée pour donner à un pays en particulier ce type d'avantage sur les autres. Les autres membres furent outrés par
la tentative assumée de détourner les lois sur les brevets et le droit d'auteur.

Une autre plate-forme d'accord fut donc créée. Les consortiums américains approchèrent le GATT (acronyme anglais pour Accord général sur les tarifs douaniers et le commerce) pour y gagner en
influence. 

Une procédure fut lancée afin d'obliger plus ou moins directement les membres à adhérer à un nouvel accord qui enterrerait la convention de Berne et renforcerait l'industrie américaine via
la redéfinition des termes «~production~», «~valeur~» et «~absence~». 

Cet accord fut appelé TRIPS en anglais ou ADPIC en français~: Aspects des droits de propriété intellectuelle qui
touchent au commerce. Lors de la ratification de l'ADPIC, le GATT fut renommé «~OMC~»~: Organisation mondiale du commerce. Les 52 pays du GATT qui refusèrent l'OMC furent bientôt dans une position
intenable et seul un pays sur les 129 pays signataires du GATT refusa de joindre l'OMC.  

L'ADPIC a été considérablement critiqué parce qu'il est construit pour enrichir les riches au détriment des pauvres, pour que ceux-ci payent de leur santé ou de leur vie quand ils n'ont pas d'argent.
Il interdit aux pays pauvres de produire leurs propres médicaments pour leur population même si celle-ci en meurt. Quelques aménagements marginaux ont bien été concédés avec le temps mais ils sont rares.

Pour donner une idée de l'importance de l'industrie américaine du droit de copie dans l'affaire, regardons ce qui s'est passé quand la Russie a demandé à entrer pour des raisons incompréhensibles dans l'OMC. Les États-Unis exigèrent que la Russie ferme AllofMP3, un site qui vendait des MP3s et était classé comme radio en Russie en payant une licence pour.

Analysons un peu la situation. Les États-Unis et la Russie siègent à la même table. Ce sont d'anciens ennemis qui gardaient leurs armes nucléaires braquées sur l'autre 24h/24, 7j/7, par tous les
temps. Les États-Unis auraient pu demander n'importe quoi au vaincu et l'auraient obtenu. 

Qu'ont-ils demandé~?

La fermeture d'un bête magasin de vente de musique en ligne.


%Annexes
\chapter*{Sources}
\addcontentsline{toc}{chapter}{Sources}

Les œuvres que j'ai utilisées sont~:

\begin{itemize}
\item le \livre{Manifeste pour le domaine public} disponible à \url{http://www.publicdomainmanifesto.org/french} pour le chapitre \ref{conserv}
\item les billets \livre{Fêter le patrimoine, mais laisser disparaître le domaine public~?} dans le domaine public de Lionel Maurel alias Calimaq pour le chapitre \ref{conserv}.
\item ma traduction du livre \livre{Against Intellectual Property} de Stephan Kinsella disponible à \url{http://contrelaproprieteintellectuelle.sploing.fr} sous licence \href{http://creativecommons.org/licenses/by/2.0/fr/}{CC-BY}  pour le chapitre \ref{proprio}.
\item ma traduction de l'article de journal \livre{My, dzieci sieci} de Piotr Czerski disponible à \url{http://traductions.sploing.fr/politique/2012/04/06/nous-les-enfants-du-siec/} sous licence \href{http://creativecommons.org/licenses/by-sa/2.0/fr/}{CC-BY-SA} pour le chapitre \ref{poete}.
\item ma traduction du billet de blog \livre{Mein Rad} de Marcel-André Casasola Merkle disponible à \url{http://traductions.sploing.fr/politique/2012/05/09/mon-velo/} sous licence \href{http://creativecommons.org/licenses/by-sa/2.0/fr/}{CC-BY-SA} pour le chapitre \ref{cycliste}.
\item ma traduction de \livre{Opening Access to Research} de Peter Suber disponible à \url{http://traductions.sploing.fr/politique/2013/05/20/ouvrir-lacces-a-la-recherche/} sous licence \href{http://creativecommons.org/licenses/by-sa/2.0/fr/}{CC-BY-SA} pour le chapitre \ref{cherch}.
\item ma traduction d'une partie du rapport \livre{Open Data Open Society} de Marco Fioretti disponible à \url{http://traductions.sploing.fr/politique/2013/05/20/pourquoi-liberer-les-donnees-publiques/} sous licence \href{http://creativecommons.org/licenses/by-sa/2.0/fr/}{CC-BY-SA} pour le chapitre \ref{devadmin}.
\item ma traduction du livre du Parti Pirate Suédois \livre{The Case for Copyright Reform} disponible à \url{http://reformedroitauteur.sploing.fr/} sous licence \href{http://creativecommons.org/licenses/by-sa/2.0/fr/}{CC-BY-SA} pour le chapitre \ref{libe}. 
\end{itemize}


\end{document}
